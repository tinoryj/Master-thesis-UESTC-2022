\section{实现}
\label{sec:implementation}

我们使用 OpenSSL 1.1.1g \cite{openssl}、Intel SGX SDK 2.7 \cite{sgx} 和Intel SGX SSL \cite{sgxssl} 在 C++ 中构建了一个 \sysname 原型。我们的原型通过 SHA256 实现明文和密文块的指纹识别操作,并通过 AES256 实现基于块的加密。它包含大约 14.2\,K LoC。

\paragraph{Setup.} 
为了引导密钥安全区,云对云的子密钥 (\S\ref{subsec:enclave-management}) 和盲密钥 (\S\ref{subsec:key-management})进入关键安全区代码。或者,\sysname 也可以在验证密钥安全区后提供两个秘密(使用 SGX \cite{sgx} 的秘密提供功能),但会产生额外的引导开销。密钥安全区使用 SHA256 生成全局密钥并在密钥回归中实现哈希函数。每个 PoW 安全区都在 NIST P-256 椭圆曲线中实现 DHKE,以与云共享一个 PoW 密钥。

\paragraph{Key generation.} 每个客户端都实现了 Rabin 指纹识别 \cite{rabin81} 用于内容定义的分块。我们将 Rabin 指纹识别中的最小、平均和最大块大小分别固定为 4\,KB, 8\,KB 和 16\,KB。为了实现基于 SGX 的推测加密 (\S\ref{subsec:encryption}),我们将 nonce 和计数器分别固定为 12 字节和 4 字节,并使用 HMAC-SHA256 实现 MAC。密钥安全区通过 SHA256 生成每个明文块的 MLE 密钥。

\paragraph{Deduplication.} \sysname 实现了基于源的去重与 PoW 相结合。 PoW enclave 实现了一个 \textit{ proof generation ECall} 来计算密文块的指纹,并使用 AES-CMAC 基于生成的指纹生成签名。云将指纹索引 (\S\ref{subsec:encrypted-dedup}) 实现为基于 LevelDB \cite{leveldb} 的键值存储。为了降低网络和磁盘 I/O 成本,我们将(非重复)密文块存储在 8MB 的容器中,作为传输和存储单位 \cite{lillibridge13}。

\paragraph{优化。} 为了减少上下文切换和 SGX 内存加密/解密开销,每个客户端批量处理多个指纹(默认为 4,096 个)以在具有密钥安全区 (\S\ref{subsec:encryption}) 的安全通信通道中传输.密钥安全区处理每批接收到的指纹。它通过指针访问批处理,而不将内容复制到 enclave \cite{harnik18}。类似地,PoW enclave 按批次(默认为 4,096)处理密文块,无需内容复制。我们还使用多线程来提高性能。每个客户端通过多线程并行处理分块、明文块指纹识别、加密、PoW 和上传,而密钥安全区和云服务于不同线程中的多个连接。