\documentclass[master]{thesis-uestc}

\title{基于可信执行环境的高性能加密重复数据删除研究}{High-Performance Encrypted Data Deduplication System Based on TEE}
\author{任彦璟}{Ren Yanjing}
\setdate[submit]{2022年3月17日}
\setdate[oral]{2022年4月15日}  
\setdate[confer]{2022年6月8日} 
\advisor{李经纬\chinesespace 副教授}{Prof. Dr. Jingwei Li}
\school{计算机科学与工程学院(网络空间安全学院)}{School of Computer Science and Engineering(School of Cyberspace Security)}
\major{计算机科学与技术}{Computer Science and Technology}
\studentnumber{201921080334}
% \ProfessionalDegreeArea{随便学学}
\ClassificationNumber{TP309.2}
\ClassifiedClass{公开}
\UDCNumber{004.78}

\newcommand{\sysnameS}{\textrm{TEEDedup}\xspace}
\newcommand{\sysnameF}{\textrm{FeatureSpy}\xspace}
\newcommand{\prototype}{\textrm{TEEDedup+}\xspace}
% non-hanging indent 
\usepackage{enumitem}
\setenumerate{listparindent=\parindent, fullwidth,itemindent=2em,label=(\arabic*)}
\setitemize{listparindent=\parindent, fullwidth, itemindent=2em}
% \makeglossaries

\begin{document}
\makecover
\originalitydeclaration
\begin{chineseabstract}
    
    \chinesekeyword{}
\end{chineseabstract}

\begin{englishabstract}
    
    \englishkeyword{}
\end{englishabstract}

\thesistableofcontents
% \thesisfigurelist
% \thesistablelist
% \glsaddall
% \thesisglossarylist


% Main body
\chapter{绪\hspace{6pt}论}

\section{研究工作的背景与意义}

\cite{ren2021accelerating}
\section{国内外研究历史与现状}
\section{本文的主要贡献与创新}
\section{本论文的结构安排}
\chapter{相关基础研究}
\label{chapter:background}

本章将介绍普通和加密重复数据删除、可信执行环境等本文利用的基础知识,以及加密重复数据删除系统中存在的性能瓶颈和安全性隐患。

\section{重复数据删除}
\label{sec:background-deduplication}

本文专注于基于数据块的重复数据删除,它以称为数据块的小型数据单元为粒度运行。基于数据块的重复数据删除比基于文件的重复数据删除更精细,因此通常具有更高的存储效率(节省更多的存储空间)。基于数据块的重复数据删除有以下两种基本的数据块分块方法:

\begin{itemize}[leftmargin=0em]
    \item \textbf{固定大小的数据分块(Fixed-size chunking)},通常将文件划分为固定长度的数据块,具有简单快速的特点,但文件中小范围修改(例如:增加或删除1字节内容)将从修改位置开始影响后续所有数据块的内容(即,边界漂移问题\citing{muthitacharoen2001low}),将严重降低重复数据删除的空间节省效果。
    \item \textbf{可变大小的数据分块(Variable-size chunking)},也称为\gls{cdc}。通常采用内容相关的方式指定数据块的边界(例如,通过Rabin指纹\citing{rabin1981fingerprinting}在特定内容模式出现时进行分块)。因此,在文件发生小范围修改时,产生的大多数数据块仍可保持不变,使得重复数据删除系统存储效率得到保障。
\end{itemize}

在多数备份系统工作负载\citing{zhu2008avoiding,lillibridge2009sparse}下,可变大小的数据分块方案通常可以获得更优的存储效率,但在某些特定工作负载(例如,VM备份数据集\citing{jin2009effectiveness})下,固定大小的数据分块方案却更加有效。本文的工作可兼容固定大小和可变大小的数据分块方法产生的数据块。

\begin{figure}[!htb]
    \small
    \centering
    \includegraphics[width=\textwidth]{pic/background/chunk-based-dedup-arch.pdf}
    \caption{基于数据块的重复数据删除的工作流程概览}
    \label{fig:chunk-based-dedup-flow}
\end{figure}

图\ref{fig:chunk-based-dedup-flow}总结了基于数据块的重复数据删除工作流程。具体地,重复数据删除系统首先通过数据分块过程将客户端的文件(例如,备份文件)分割为逻辑数据块,根据每一个逻辑数据块的内容,使用哈希算法计算得到其对应的唯一标签(又称为指纹)。如果两个数据块具有相同的指纹,则认为两个数据块内容相同(不同逻辑数据块计算得到相同指纹的概率可忽略不计\citing{black2006compare});若两个逻辑数据块指纹不一致,则认为两逻辑数据块不同。重复数据删除系统仅存储相同逻辑数据块的唯一副本(称为物理数据块),并且每个相同的逻辑数据块仅通过一个空间开销较小的索引指向相同的物理数据块。此外,基于数据块的重复数据删除系统记录文件所拥有的所有逻辑数据块的信息作为该文件的元数据,用于文件读取、删除等操作。

重复数据删除技术根据重复数据删除操作发生的位置可分为源端重复数据删除及目标端重复数据删除\citing{IDC2010Data}:

\begin{itemize}[leftmargin=0em]
    \item \textbf{源端重复数据删除(Source-based Deduplication)}由客户端计算目标数据块的哈希值,并由服务端检查该哈希值是否存在于索引表中。如果哈希值存在(即服务端已有目标数据块的副本),则通知客户端无需传输目标数据块。
    \item \textbf{目标端重复数据删除(Target-based Deduplication)}强制客户端传输所有密文数据块,并在服务端对所有收到的密文数据块进行重复数据删除。
\end{itemize}

源端重复数据删除技术可有效节省网络流量资源,可显著降低云服务商提供存储服务的成本,但泄露了“其他客户端是否已经存储相应密文数据块”的侧信道信息(参见\S\ref{subsubsec:intro-problem-security});目标端重复数据删除具有更高的隐私保护能力,但产生大量网络资源浪费,并显著增加了服务端计算开销。本文关注网络资源开销较小的源端重复数据删除技术,并通过TEE技术解决其存在的性能和安全性问题。

\section{安全重复数据删除}
\label{sec:background-enc-deduplication}

加密重复数据删除解决了外包环境(例如,云存储)中的数据块机密性保障问题,同时保持了重复数据删除的有效性。出于安全因素考虑,用户希望将自己的数据加密后再进行外包存储,以确保个人数据隐私性。传统对称加密算法算法为每个客户端或每个逻辑数据块分配独立的加密密钥,使得来自不同(或相同)客户端的相同的逻辑数据块被加密为不同的密文数据块,服务器无法感知这些密文数据块所对应的明文数据块内容是否一致,使得针对外包数据的重复数据删除完全失效。

\begin{figure}[!htb]
    \small
    \centering
    \includegraphics[width=\textwidth]{pic/background/chunk-based-enc-dedup-arch.pdf}
    \caption{基于数据块的加密重复数据删除的工作流程概览}
    \label{fig:chunk-based-enc-dedup-flow}
\end{figure}

如图~\ref{fig:chunk-based-enc-dedup-flow}所示,基于内容的密钥生成方法给加密重复数据删除提供了新的处理思路。最基本的基于内容的密钥生成方法的实例化是基于明文数据的内容(例如,基于重复数据删除中的逻辑数据块)导出其对称加密密钥,并使用该密钥加密明文数据以形成对应的密文数据(例如,加密的逻辑数据块)。因此,使用基于内容的密钥生成方法为任意逻辑数据块产生对称加密密钥可以确保将相同的明文数据块加密为相同的密文数据块。最后,存储系统从每个密文数据块中导出其对应的指纹并执行重复数据删除。相较于普通重复数据删除,加密重复数据删除需额外保存用户文件所包含数据块的密钥元数据,并由各个客户端进行加密以防止服务端取得相应数据块的明文内容。

\subsection{数据块加密技术}
\label{subsec:background-encrypted-deduplication-key}

\gls{mle}\citing{bellare2013MLE}将基于内容的密钥生成方案等加密原语形式化用于加密重复数据删除。它指出了如何从明文数据块的内容导出相应的对称加密密钥(称为消息锁加密密钥,简称为MLE密钥)。以最广泛使用的消息锁加密技术:\gls{ce}\citing{douceur2002reclaiming}为例,CE使用明文数据块的安全哈希作为MLE密钥,使得相同的明文数据块可被加密为相同的密文数据块,从而令重复数据删除技术在密文数据块之上仍然可行。除此之外,基于CE的消息锁加密方案还包括:

\begin{enumerate}[leftmargin=0em]
    \item \textbf{\gls{hce}}\citing{douceur2002reclaiming}与CE具有相同的MLE密钥产生规则,但基于明文哈希值计算指纹用于重复数据删除。作为对比的,CE使用密文的指纹进行重复数据删除检查。
    \item \textbf{\gls{rce}}\citing{douceur2002reclaiming}使用随机密钥加密明文以产生非确定的密文,但基于明文的哈希值进行重复数据删除检查。
    \item \textbf{\gls{cd}}\citing{li2016cdstore}使用明文哈希值作为秘密共享(Secret Sharing)的输入种子,并对产生的多个秘密共享分别计算哈希并进行重复数据删除,在兼容重复数据删除的基础上提高了密文存储的可靠性。
\end{enumerate}

然而,CE容易受到\textbf{离线暴力破解攻击}的威胁。这是由于CE的MLE密钥(即明文数据块的哈希)可以自由产生。具体来说,攻击者通过枚举所有可能的明文数据块的MLE密钥来从目标密文数据块(其加密密钥未知)推断输入的明文数据块,以检查是存在某个明文数据块被加密到目标密文数据块。

服务器辅助消息锁加密\textit{(Server-aided MLE)}\citing{bellare2013DupLESS}是最先进的加密原语,可增强加密重复数据删除对离线暴力攻击的安全性。它为消息锁加密中的MLE密钥生成步骤部署了一个专用的\textbf{密钥服务器(Key server)}。为了加密明文数据块,客户端首先将明文数据块的指纹发送到密钥服务器,密钥服务器通过指纹和密钥服务器维护的\textbf{全局秘密(Global secret)}返回MLE密钥。如果全局秘密是安全的,则攻击者无法发起离线暴力攻击;如果全局秘密被泄露,则其安全性会降低到原始消息锁加密的安全性。服务器辅助消息锁加密建立在如下两种安全机制之上:

\begin{itemize}[leftmargin=0em]
    \item \textbf{\gls{oprf}}\citing{naor2004Number}是一种包含服务端和客户端的安全多方计算协议。其中,服务端持有密钥$k$,客户端持有输入$x$,协议双方联合计算函数$f_k(x)$,并确保最终仅由客户端得到所计算的目标函数值。OPRF协议采用双盲方式执行,即计算过程中可确保服务端无法获得客户端输入且客户端无法获得服务端密钥。
    \item \textbf{速率限制(Rate-limiting)}\citing{bellare2013DupLESS}阻止特定操作频率超出某些限制。在大型系统中,速率限制通常用于保护底层服务和资源。
\end{itemize}

基于无记忆伪随机函数,密钥服务器可依据客户端发送明文数据块的“盲化指纹”,进而基于盲指纹和全局秘密产生对应数据块的“盲化密钥”,阻止了密钥服务器了解明文数据块信息;并使得客户端可在不了解密钥服务器所拥有的全局秘密的条件下获得目标明文数据块的MLE密钥。对来自客户端的密钥生成请求进行速率限制,进一步防止恶意客户端向密钥服务器发出海量密钥生成请求以获得目标MLE密钥,限制了攻击者暴力破解攻击的速度。

\subsection{数据所有权证明技术}
\label{subsec:background-encrypted-deduplication-pow}

为了节省宝贵的网络带宽资源,现有加密重复数据删除系统普遍采用源端重复数据删除,以便在客户端删除重复数据块,而无需上传到服务端(\S\ref{sec:background-deduplication})。但是,源端重复数据删除导致“目标数据块是否已在服务端存储”的侧信道信息,使得某些客户端存在恶意时,源端重复数据删除很容易受到侧信道攻击\textit{(Side-channel attack)}\citing{harnik2010side,halevi11}的影响。

一种典型的侧信道攻击被称为\textbf{伪造所有权攻击}:恶意客户端未授权访问其他客户端存储的数据块\citing{harnik2010side,mulazzani11}。具体来说,由于服务端仅能基于收到的数据块哈希值判断客户端是否拥有对应的数据块,攻击者可使用任意目标密文数据块的指纹来说服服务端其是该目标数据块的所有者,进而获得该数据块的完全访问权限(\S\ref{subsec:intro-background});另一种侧信道攻击被称为\textbf{推测内容攻击}:恶意客户端将目标密文数据块指纹发送到服务端以检查该数据块是否存在(例如,目标密文数据块对应于某个可能的密码\citing{harnik2010side}),以此识别来自其他客户端的敏感信息(\S\ref{subsubsec:intro-problem-security})。

现有防御机制尚无法在执行源端重复数据删除且不大量增加额外网络带宽资源开销的前提是防御推测内容攻击。而为了防止伪造所有权攻击,源端加密重复数据删除增加了\gls{pow}机制\citing{halevi11},要求客户端额外向服务端提交目标密文数据块的所有权证明(Proof),且仅在所有权证明成功的条件下进行重复数据删除。所有权证明机制可使得服务端只针对客户端真实拥有(即具有完整访问权限)的密文数据块执行重复数据删除,避免了攻击者非法访问其他客户端已在云服务端存储的内容。现有基于Merkle树的所有权证明机制(称为POW-MT)\citing{xu2013weak}使用纠删码对数据块进行编码,随后在纠删码编码结果的基础上建立Merkle树用于所有权证明。而另一种基于\gls{uh}的所有权证明机制(称为POW-UH)\citing{halevi2011proofs} 相较于基于Merkle树的所有权证明机制拥有更高的效率,但降低了安全性假设。

本文关注网络和服务端计算资源开销较小的源端重复数据删除技术,并通过TEE技术解决其存在的密钥生成和所有权证明效率问题,以及易受到推测内容攻击的安全性问题。

\section{可信执行环境}
\label{sec:background-tee}

\gls{tee}\citing{tee}是一种在计算平台上由软硬件协同构建的安全区域。其设计目标是确保安全区域中的程序按照预期执行,保证程序初始状态和运行时的机密性、完整性。基于GlobalPlatform组织\citing{GP}对TEE的标准规范,各个软硬件厂商结合自身基础架构构建的TEE存在显著差异,但各类TEE均具有以下主要特性:

\begin{itemize}[leftmargin=0em]
    \item \textbf{隔离性}:
          TEE将其中的目标程序、数据等与外部环境隔离,使得TEE外软硬件均无法获得其内部的机密信息。最早的隔离机制出现于Intel 80286处理器,其提供了两种具有不同安全权限的CPU运行模式,并在之后的X86架构处理器上衍生出各类具有不同权限的特权界别。近年来,Intel提出了安全区域范围更小的可信执行环境SGX(Intel Software Guard Extensions)\citing{sgx};同样的,ARM提出了TrustZone技术通过时分复用的方式为CPU提供了安全世界\textit{(Secure World)}与非安全世界\textit{(Normal World)}的隔离。
    \item \textbf{软硬协同性}:
          标准定义下允许厂商单独使用软件或硬件机制实现TEE,但在实际开发的产品中,厂商往往选择结合软件及硬件的方案设计TEE。
    \item \textbf{富表达性}:
          与传统安全芯片或纯软件的密码学隐私保护机制相比,TEE支持更丰富的上层应用。软件开发者仅需根据业务需求合理划分程序的隐私与非隐私部分,且TEE不会对其中运行的程序的算法逻辑、程序语言等方面进行额外限制。即,TEE内部是图灵完备的。同时,TEE内部的数据无需进行复杂的密态运算,使得其中的应用可支持更多复杂的算法。
\end{itemize}

\subsection{Intel SGX}
\label{subsec:background-tee-sgx}

\begin{figure}[!htb]
    \small
    \centering
    \includegraphics[width=0.5\textwidth]{pic/background/sgx-example.pdf}
    \caption{Intel SGX框架}
    \label{fig:sgx-arch}
\end{figure}

Intel Software Guard Extensions (Intel SGX)\citing{sgx,sgx2},是一组内置于现代Intel CPU中用于增强应用程序代码和数据安全性的扩展指令。开发者可利用SGX技术将应用程序的安全操作封装在称为“安全区\textit{(Enclave)}”的硬件保护环境中,保障用户关键代码和数据的机密性和完整性。Intel SGX最关键的优势在于将应用程序以外的软件栈如操作系统(OS)和基本输入输出系统(BIOS)都排除在了\gls{tcb}之外,一旦程序和数据在安全区之中,即便是操作系统也无法影响安全区内的代码和数据,安全区的安全边界只包含CPU和它本身(如图\ref{fig:sgx-arch}所示)。Intel SGX具有隔离、证明和密封三项主要安全功能。

\textbf{隔离(Isolation)}:安全区代码和数据被放置在被称为\textit{Enclave page cache (EPC)}的硬件内存保护区域中,该内存区域使用内存加密引擎(MEE)进行加密,以防止泄露任何内存数据到本安全区之外。EPC以大小为4\,KiB的页面为基本单位管理安全区内存,任意安全区内应用程序最多可占用96\,MiB内存空间\citing{harnik2018SGX}。当安全区内存使用超过上限时,须将未使用的内存页面加密并逐出到未受保护的主存,并在将逐出的页面加载回安全区内存时解密并验证完整性,导致巨大的分页开销\citing{arnautov2016SCONE,dinhngoc2019Everything}。

此外,SGX为安全区和未受保护的内存之间的交互提供了两个接口,应用程序可以通过安全区内部调用(ECalls)进入安全区以执行安全区内部函数,并且在ECall中,程序可通过安全区外部调用(OCalls)暂时退出安全区并调用不受保护的内存中的不受信任的函数。但安全区内部调用和外部调用会产生大量的CPU上下文切换开销\citing{harnik2018SGX},严重影响SGX应用程序性能。

\textbf{认证(Attestation)}:SGX安全区支持本地认证\textit{(Local Attestation)}和远程认证\textit{(Remote Attestation)},以确保安全区内运行的程序未被修改。特别的,在远程认证过程(参见\citing{SGX-RA}提供的端到端远程认证示例)中,远程实体(例如,服务端)需要联系Intel运营的证明服务来检查目标安全区提供的安全区相关信息的完整性和正确性。随后,远程实体通过将其收到的安全区信息与目标安全区中预期的可信信息进行比较来验证目标安全区是否未被修改。然而,由于远程认证需要Intel服务器的支持,远程认证过程通常导致较高且不可控的时间开销。

\textbf{密封(Sealing)}: SGX安全区通过密封在安全区内容需要在安全区外持久化存储时进行保护。它使用目标安全区专用的密封密钥(\textit{Sealing key})在数据被移出安全区之前进行加密。密封密钥可以从安全区测量哈希(\textit{Measurement hash},即安全区内容的SHA-256哈希)或安全区的创建者提供的签名身份派生。基于前者产生的密封密钥仅有目标安全区本身可以导出,而基于后者产生的密封密钥可有同一开发者创建的各个不同的安全区导出。因此,仅有相应的安全区才能获得正确的密封密钥并解密密封数据。

\subsection{ARM TrustZone}
\label{subsec:background-tee-tz}

ARM TrustZone\citing{trustzone}是ARM公司为Cortex-A微架构\citing{cortex-a}设计的一种TEE解决方案,并在最新的指令集架构中扩展至Cortex-M微架构\citing{cortex-m}。TrustZone技术通过对原有ARM处理器硬件架构进行修改,将处理器划分为两个具有不同安全权限的运行环境(两运行环境通过时分复用共享处理器资源):安全世界(Secure World)和非安全世界(Normal World),并防止运行于非安全世界的软件直接访问安全世界资源。

\begin{figure}[!htb]
    \small
    \centering
    \begin{tabular}{@{}c@{}c@{}c}
        \includegraphics[width=0.49\textwidth]{pic/background/ARM-TZ-A.pdf} &
        \hspace{5pt}
        \includegraphics[width=0.49\textwidth]{pic/background/ARM-TZ-M.pdf}   \\
        \mbox{\small (a) Cortex-A微架构中的TrustZone}                       &
        \mbox{\small (b) Cortex-M微架构中的TrustZone}                         \\
    \end{tabular}
    \caption{ARM TrustZone技术}
    \label{fig:ARM-TZ-base}
\end{figure}

如图~\ref{fig:ARM-TZ-base}(a)所示,ARM引入了一种称为监视模式的处理器模式,该模式负责在世界过渡时保留处理器状态,两个世界可以通过称为\gls{smc}的特权指令进入监视模式并实现彼此切换。

与Cortex-A相同的是,Cortex-M依旧将处理器运行状态划分为安全世界和非安全世界,并阻止运行于非安全世界的软件直接访问安全资源。不同的是,Cortex-M针对更快的上下文切换和低功耗应用进行了优化。具体来说,Cortex-M中世界之间的划分是基于内存映射的,并且转换是在异常处理代码中自动发生的(如图~\ref{fig:ARM-TZ-base}(b)所示)。这意味着,当从安全内存运行代码时,处理器处于安全世界,而当从非安全内存运行代码时,处理器处于非安全世界。Cortex-M中的TrustZone技术排除了监视模式,也不需要任何安全的监视软件,这大大减少了安全世界与非安全世界的切换延迟,使得世界之间的转换为更高效。总体而言,基于TrustZone技术,处理器厂商、开发者可以搭建一个可信执行环境,并在TEE内运行可信操作系统\textit{TEE OS}(例如,高通QSEE\citing{qsee}、OP-TEE\citing{op-tee})。

\section{本章小结}

本章介绍了普通与加密重复数据删除,现有加密重复数据删除中数据块加密技术和数据所有权证明技术导致的性能问题,以及可信执行环境的相关内容。

\chapter{基于Intel SGX的高性能重复数据删除}
\chapter{基于可信执行环境(TEE)对推测内容攻击的主动检测}

\section{简介}
\label{sec:featurespy-intro}

现有加密后重复数据删除中的数据所有权证明机制不足以完全击败恶意客户端(参见\S\ref{subsubsec:intro-problem-security})。即使通过数据块级的所有权证明保护源端重复数据删除,攻击者仍可通过枚举可能的数据内容发起推测内容攻击(Learning-content attack)\cite{harnik2010side, zuo2018mitigating}。具体来说,攻击者先验的了解数据文件的大部分格式化内容(例如,工资单),并旨在识别其他客户端拥有的文件的私有部分(例如,工资单中的实际工资和奖金)。攻击者通过枚举私有内容的可能取值来伪造大量文件,并对每个文件执行源端重复数据删除,最终在某些文件无需上传任何数据块时推断目标私有信息的取值。

本文扩展了以前的研究\cite{harnik2010side, zuo2018mitigating},证明在实践中推测内容攻击的严重性。具体来说,通过一个案例研究(参见\S\ref{sec:featurespy-attack}),本文说明针对具有较低信息熵(例如,包含数千个枚举取值)的文件,推测内容攻击可在本地局域网络环境中以不超过2分钟的时间开销完成攻击,而在真实云环境下也仅需8分钟左右完成。本文认为数据所有权证明无法阻止推测内容攻击,因为攻击者可以访问伪造文件的全部内容并且能向云服务端证明其对这些文件的所有权。

本章提出了\sysnameF,它通过主动检测每个客户端中的推测内容攻击来增强加密后重复数据删除的安全性。本文研究发现发起推测内容攻击的客户端具有较短时间内处理大量相似数据块(来自大量伪造的文件)的行为特征,且这些相似数据块中的信息修改仅限于针对数据块内容的小范围修改。另一方面,本文的研究表明,正常工作负载(未发生推测内容攻击)中的连续数据块(可能一起处理)彼此之间存在显着差异。因此,\sysnameF 在客户端可信执行环境(参见\S\ref{sec:background-tee})中检查所处理数据块的相似性,若客户端在短时间内处理了大量相似数据块,则判定为推测内容攻击可能发生,最终实现针对恶意客户端的主动侦测。

\sysnameF 建立在本文提出的基于TEE的数据所有权证明(参见\S\ref{chapter:sgxdedup})方案的基础之上,该方案可确保重复数据删除仅发生在经过安全区处理的加密数据块上。\sysnameF 基于加密数据块检测相似性,并将检测过程与安全区中的数据所有权证明耦合,以防止恶意客户端绕过检测过程。具体来说,\sysnameF 提出了相似性保留加密(Similarity-preserving
encryption, SPE),相似性保留加密为消息锁加密增加了加密后数据的相似性保留能力,除了消息锁加密密钥,其额外根据原始数据的内容特征推导出特征密钥(Feature key)。由于相似的原始数据很可能具有相同的特征密钥,相似性保留加密使用特征密钥加密一小部分原始内容(称为指标(Indicator))以保留相似性,并使用消息锁加密方案加密剩余的大部分内容以确保数据安全性。由于相似数据仅在少数区域不同,\sysnameF 通过比较使用特征密钥加密的小部分来检测原始数据是否相似。

本章实验分析表明\sysnameF 可以有效地检测推测内容攻击(例如,在本文的案例研究中,检测率最低为98.6\%),而产生误判的概率很低(例如,\sysnameF 的默认配置条件下为误判率为零)。此外,本文将\sysnameF 部署到基于TEE的高性能加密重复数据删除系统\sysnameS 中(关于\sysnameS 的相关设计参见\S\ref{chapter:sgxdedup}),以提高其应对推测内容攻击的安全性。本章提出的\sysnameF 的原型实现(即\prototype)相较于\sysnameS 增加的额外性能开销在基于真实世界数据集的性能测试中低至8.8\%(上传)和0.8\%(下载)。
\section{背景和问题}
\label{sec:featurespy-background}

\subsection{加密后重复数据删除}
\label{subsec:featurespy-basics}

\paragraph*{重复数据删除。}
重复数据删除是一种冗余消除技术,可以有效节省存储空间 \cite{wallace12, meyer11}。它将每个输入文件划分为固定大小或可变大小的 {\em 块}。每个块由相应内容的加密哈希(称为 {\em 指纹})标识,并且假设不同块映射到同一指纹的概率可以忽略不计 \cite{black2006compare}。重复数据删除存储系统仅在其指纹与现有存储指纹唯一时才存储块,以实现存储效率。本文重点介绍 {\em 外包环境}(例如云存储)中的重复数据删除,其中许多客户将数据外包到云中。云对来自相同或不同客户端的数据执行重复数据删除,以节省维护成本 \cite{harnik10}。


\paragraph*{加密后重复数据删除。}
加密后重复数据删除在外包环境中增强了重复数据删除的安全性。具体来说,一个威胁是受感染的云可以窃听外包数据并了解敏感信息。为了击败受损的云,加密后重复数据删除在将预先重复数据删除的块(称为 {\em 明文数据块})外包给云之前执行 {\em 消息锁定加密 (MLE)} \cite{bellare2013MLE, bellare2013DupLESS}。具体来说,每个明文数据块都使用从明文数据块的内容派生的对称密钥(称为 {\em MLE 密钥})加密,这样跨不同客户端的相同明文数据块总是映射到相同的加密块(称为 {\em ciphertext chunks}) 可以通过重复数据删除删除。 MLE 的一个实例是 {\em 收敛加密 (CE)},它使用每个明文数据块的加密哈希作为 MLE 密钥 \cite{douceur02}。

另一个威胁是恶意客户端可以发起 {\em side-channel attack} \cite{harnik10, halevi11} 以从其他客户端获得对数据的未经授权的访问。具体来说,为了执行 {\em 源端重复数据删除} \cite{harnik10},客户端将每个密文数据块的指纹提交给云,云维护一个 {\em 指纹索引} 以跟踪现有存储的密文数据块。如果目标密文数据块的指纹已经存储在指纹索引中(即重复),则客户端不需要传输密文数据块。否则(即不重复),云通知客户端传输密文数据块。一种侧信道攻击 \cite{mulazzani11, halevi11} 是恶意客户端可以使用任何目标密文数据块的指纹来欺骗云,使其成为相应的块,从而获得未来下载块 \cite{mulazzani11} 的全部权利。


加密后重复数据删除将源端重复数据删除与 {\em 所有权证明 (PoW)} \cite{halevi11} 相结合,以防止上述边信道攻击。具体来说,除了区块指纹外,客户端还需要上传一个{\em proof},云端会根据这个证明来验证客户端是否完全{\em 持有对应的密文区块}(而不仅仅是一个指纹) .云只有在相应密文数据块的所有权被成功验证时才会响应,这样恶意客户端就无法识别其他客户端拥有的密文数据块。

\subsection{内容学习攻击(Learning-content Attack)}
\label{subsec:featurespy-attack}
除了虚假所有权之外,加密的重复数据删除还面临另一种侧通道攻击(仍然由恶意客户端发起),称为 {\em learning-content attack} \cite{harnik10, zuo2018mitigating},它利用了 {\em 重复数据删除的泄漏模式}(即,块是否由任何其他客户端上传)。学习内容攻击假设攻击者先验地知道某些受害者客户端拥有一个单独的文件,其内容遵循公开的内容模式(即已知部分)。它的目标是识别文件的私有部分。具体来说,攻击者会枚举私有部分的所有可能值,并生成许多假文件。它上传每个假文件,如果被告知不要传输某个文件的任何块(即假文件是整个副本),则推断目标文件。

我们认为 PoW \cite{halevi11} 不足以防止学习内容攻击,因为对手枚举了块的全部内容并且能够说服其对它们的所有权。换句话说,PoW 无法检测到这些块是完全由客户拥有还是只是一些伪造品。

\paragraph*{案例研究。}
我们通过案例研究强调学习内容攻击的严重性。我们认为 Alice 和 Bob 是一所大学的高年级学生,他们租用云(启用跨用户重复数据删除)来备份注册学生的计算机。求职后,Alice 和 Bob 都会收到一家公司的 offer,这些 offer 最初存储在各自的计算机中,然后自动备份到云端。
我们假设 Alice 是旨在推断 Bob 的报价信息的对手。


在这里,我们通过 Google 的 offer letter \cite{GoogleOffer} 模拟 offer,并更改 \textit{姓名},\textit{年薪}(假设是 6\,K \cite{harnik10} 的倍数,介于204\,K 和 804\,K) 和 \textit{签到红利}(假设是 10\,K 的倍数并且介于 300\,K 和 600\,K 之间)生成 Alice 和 Bob 的报价,每个大约需要 18.5\,KiB。此外,我们实施了一个加密的重复数据删除存储系统,以将优惠存储在云中。具体来说,客户端将录取通知书作为输入,执行分块、加密和源端重复数据删除,并且仅将非重复的密文数据块传输到云(\S\ref{subsec:featurespy-basics})。最初,我们将 Bob 的报价存储在云中,这样 Alice 可以通过学习内容攻击推断 Bob 的实际工资和签约奖金。具体来说,Alice 建立在自己的 offer 之上,枚举 Bob 的所有可能的年薪和签约奖金,并在存储一些虚假 offer 时,如果没有转移任何块,则推断 Bob 的 offer。


我们随机生成 Bob 的工资和签约奖金,并评估两个 LAN 中的学习内容攻击(在我们的本地测试平台中部署客户端和云,请参阅 \S\ref{subsec:featurespy-evaluation-performance} 以了解测试平台配置) 和云(将客户端部署在我们的本地测试平台和云在阿里云 \cite{Alibaba},请参阅 \S\ref{subsec:featurespy-evaluation-performance} 以了解云配置)测试平台。表~\ref{tab:featurespy-attack} 显示了识别正确报价的平均成本。 Alice 需要上传大约 841 个虚假报价,这些报价转换为消耗 7.4\,MiB 网络流量(包括非重复密文数据块和元数据的传输)。也就是说,在局域网和云测试台上分别推断出 Bob 的隐私信息只需要 105.0\,s 和 475.5\,s。


\begin{table}
  \centering
    \small
  \begin{tabular}{|c|c@{\hspace{.2em}}|@{\hspace{.2em}}c@{\hspace{.2em}}|@{\hspace{.2em}}c@{\hspace{.2em}}|}
    \hline
    {\bf Testbeds} & {\bf Upload Attempts} & {\bf Network Traffic} & {\bf Time}\\
    \hline
    \hline
    LAN & \multirow{2}{*}{841.0 $\pm$ 608.3} & \multirow{2}{*}{7.4 $\pm$ 5.4\,MiB} & 105.0 $\pm$ 76.1\,s \\
    \cline{1-1}\cline{4-4}
    Cloud & & & 475.5 $\pm$ 339.8\,s   \\
    \hline
  \end{tabular}
  \caption{学习内容攻击的平均成本。 我们在 10 次运行中评估结果,并包括来自 {\em Student's t-Distribution} 的 95\% 置信区间.}
  \label{tab:featurespy-attack}
  \vspace{-6pt}
\end{table}

\section{Setting}
\label{sec:setting}


This paper presents \sysnameF to augment encrypted deduplication by {\em proactively detecting the learning-content attack} in a client-side {\em trusted execution environment (TEE)}, so as to fully defeat against the malicious client.

\paragraph{Trusted execution.} We build on {\em Intel Software Guarded Extensions (SGX)} \cite{sgx} to realize the TEE, since SGX is widely supported by today's commodity computers. In addition to SGX, our design (\S\ref{sec:design}) can be extended to other trusted computing technologies \cite{amd-sev, pinto19} that support TEE.

SGX extends Intel CPU with a set of security-related instructions to realize a TEE. It allocates an {\em enclave} in a hardware-guarded memory region (called {\em enclave page cache (EPC)}), in order to host (in-enclave) contents with confidentiality and integrity guarantees. After an enclave is created, SGX provides {\em remote attestation} to authenticate the enclave via a remote entity (e.g., the cloud), as well as {\em secret provisioning} to share a secret key between the (authenticated) encalve and the cloud. Also, SGX provides two interfaces for the interaction between the enclave and  unprotected memory. A program can enter the enclave to execute in-enclave functions via {\em enclave calls (ECalls)}. Within an ECall, it can temporarily exit the enclave and call untrusted functions in unprotected memory via {\em outside calls (OCalls)}.



\paragraph{Deployment scenario.} Figure~\ref{fig:model} presents the scenario of encrypted deduplication (\S\ref{sub:basics}). To deploy \sysnameF, we first compile the enclave code into a shared object \cite{sgx}, and distribute the shared object along with a signature for integrity verification to each client. The cloud  hosts the shared object to validate the enclave of each client.
Specifically, the client initializes \sysnameF to create the corresponding enclave by loading the shared object, and the cloud authenticates each enclave via remote attestation \cite{sgx} to ensure that the correct code is loaded into the enclave.

In the upload procedure, \sysnameF processes plaintext chunks (that are generated by the client), and computes corresponding ciphertext chunks and fingerprints for source-based deduplication. A non-compromised client only transfers non-duplicate ciphertext chunks to the cloud, while \sysnameF reports a malicious client if the client is caught to launch the learning-content attack.

\begin{figure}
  \centering
  \includegraphics[width=3.25in]{pic/featurespy/deployment.pdf}
  \vspace{-6pt}
  \caption{Deployment of \sysnameF.}
  \label{fig:model}
  \vspace{-6pt}
\end{figure}


% \paragraph{Challenges of design with SGX.} Although SGX supports plenty of security features, it is necessary to judiciously make design decisions, which affect the security  and performance of \sysnameF. Specifically, the {\em trusted computing base (TCB)} of an SGX enclave includes the Intel CPU, in-enclave code and data, and (enclave) interface function calls. A large TCB facilitates design, yet increasing security risks since a lot of interface calls may behave in unpredictable ways \cite{lie05}.

% Also, when the in-enclave content has a larger size than EPC (whose size is limited by 128\,MiB), SGX encrypts unused EPC pages and evicts them to unprotected memory, leading to high {\em EPC paging overhead} \cite{dinhngoc19}. Although the latest version of SGX supports a larger EPC size up to 1\,TiB \cite{sgx2}, it loses the integrity tree protection \cite{feng21}.

% Furthermore, both ECalls and OCalls add  context-switch overhead \cite{arnautov16}. The code that is just running in the enclave (i.e., without accessing unprotected memory) even incurs performance drop over that is running in unprotected memory \cite{harnik18}.

%



% The foremost issue of designing \sysnameF is to decide detecting the learning-content attack in either the cloud or the client side.
% A cloud-side detection system can view the uploads of all clients, and monitor their behaviors. However, the uploaded contents include chunk fingerprints that are random by nature, and non-duplicate ciphertext chunks that form just a partial view of each client's behavior. Such information is insufficient to reason about the learning-content attack.

% Thus, we choose to design \sysnameF for client-side deployment, while running the detection process in a {\em shielded environment} to protect against a malicious client that  may tamper the system to escape detection (\S\ref{sub:threat}).



% \subsection{Intel SGX}
% \label{sub:sgx}

% \paragraph{SGX basics.} SGX extends Intel CPU with a set of security-related instructions to realize a shielded execution environment. It allocates an {\em enclave} in a hardware-guarded memory region (called {\em enclave page cache (EPC)}), in order to host (in-enclave) contents with confidentiality and integrity guarantees. After an enclave is initialized, SGX provides {\em remote attestation} to authenticate the enclave via a remote entity (e.g., the cloud), as well as {\em secret provisioning} to share a secret key between the (authenticated) encalve and the cloud. Also, SGX provides two interfaces for the interaction between the enclave and  unprotected memory. A program can enter the enclave to execute in-enclave functions via {\em enclave calls (ECalls)}. Within an ECall, it can temporarily exist the enclave and call untrusted functions in unprotected memory via {\em outside calls (OCalls)}.



% \paragraph{Challenges of design with SGX.} Although SGX supports plenty of security features, it is necessary to judiciously make design decisions, which affect the security  and performance of \sysnameF. Specifically, the {\em trusted computing base (TCB)} of an SGX enclave includes the Intel CPU, in-enclave code and data, and (enclave) interface function calls. A large TCB facilitates design, yet increasing security risks since a lot of interface calls may behave in unpredictable ways \cite{lie05}.

% Also, when the in-enclave content has a larger size than EPC (whose size is limited by 128\,MiB), SGX encrypts unused EPC pages and evicts them to unprotected memory, leading to high {\em EPC paging overhead} \cite{dinhngoc19}. Although the latest version of SGX supports a larger EPC size up to 1\,TiB \cite{sgx2}, it loses the integrity tree protection \cite{feng21}.

% Furthermore, both ECalls and OCalls add  context-switch overhead \cite{arnautov16}. The code that is just running in the enclave (i.e., without accessing unprotected memory) even incurs performance drop over that is running in unprotected memory \cite{harnik18}.



\paragraph{Threat model.} Our major security goal is to enhance encrypted deduplication (\S\ref{sub:basics}) with security against a {\em malicious} client. Like encrypted deduplication \cite{bellare13a}, we consider a compromised cloud that aims to eavesdrop the original content from any stored ciphertext chunk. In addition, we consider a malicious client that aims to learn the original plaintext chunks of other non-compromised clients. Specifically, the malicious client can access its compromised  plaintext chunks and keys, and arbitrarily fake new plaintext chunks to launch the learning-content attack (\S\ref{sub:basics}). Also, it can tamper with the  contents and operations in unprotected memory, in order to escape the capture of \sysnameF.
Our threat model makes the following assumptions.
\begin{itemize}[leftmargin=*]
\item The communication channel between each client and the cloud is protected by SSL/TLS against eavesdropping.
\item
  The SGX enclave is trusted and authenticated (e.g., via remote attestation when it is first bootstrapped), so as to honestly perform attack detection against tampering. Also, like the previous works \cite{shinde20, ren21}, the SGX enclave preserves confidentiality for only a cryptographic key (i.e., the proof key in \S\ref{sec:implementation}) rather than all in-enclave contents; this is important in light of the side-channel attacks against SGX (e.g., see \cite{fei21} for a survey).

%   trusted, and authenticated by remote attestation (\S\ref{sub:sgx}) when it is first bootstrapped.
% It assumes the confidentiality of SGX only in one lemma, i.e., the secrecy of a cryptographic key. This is an important design choice in light of the side-channels
%  Although previous studies  report the security vulnerabilities of SGX, they are orthogonal
 % to \sysnameF, which can be enchanced with  future improvements of SGX.
% \item \sysnameF augments PoW \cite{halevi11, ren21}, and naturally addresses the threat that uses a fingerprint to obtain the ownership of the corresponding chunk (\S\ref{sub:basics}).
\item A malicious cloud may corrupt outsourced data to compromise integrity. \sysnameF does not address the threat, yet it is compatible with {\em proof data prosession (PDP)} \cite{ateniese07} and {\em proof of retrievability (PoR)} \cite{juels07} to perfrom periodically integrity verification of outsourced data, as well as fault-tolerant storage to recover data from corruption \cite{li15}.
\end{itemize}

\section{FeatureSpy Design}
\label{sec:design}

We design \sysnameF with the following goals in mind.

\begin{itemize}[leftmargin=*]

\item {\bf Attack detection.} It reliably detects the learning-content attack against tampering by the client (i.e., robustness). Also, it achieves a high probability for catching the attack, while the number of misjudgements is small.
\item {\bf Confidentiality and bandwidth/storage efficiency.} It preserves source-based encrypted deduplication, so as to achieve  data confidentiality and bandwidth/storage efficiency. Also, it mitigates the information leakage due to key compromise.
\item {\bf Low performance overhead.} It detects the learning-content attack in the write path, and incurs low performance overhead in encrypted deduplication deployment.
\item {\bf Limited trusted computing base (TCB).} It manages minimal trust portions and function call interfaces for the SGX enclave; this is a necessary design goal in light of abusing interface function calls \cite{lie05}.
\end{itemize}

We first present a strawman (insecure) design to demonstrate how we detect the learning-content attack by monitoring  content processing. Then, we present \sysnameF, which prevents a malicious client from bypassing the attack detection procedure.



\subsection{Strawman Design}
\label{sub:basic}

\paragraph{Overview.} Our insight is that the adversary that launches the learning-content attack enumerates many {\em similar} chunks, which follow identical content patterns with information changes in a few  regions. In other words, such chunks are likely to share the same {\em content features} (that can be generated based on the corresponding chunk contents via {\em N-transform} \cite{shilane12}, see a latter part ``features extraction'' of this subsection), and be  processed within a small time window. This leads to a {\em skew feature distribution} for each time window in the attack procedure, since some features are shared by most of the chunks that are processed together.

On the other hand, we argue that the feature distribution of consecutive chunks (that are processed together \cite{zhu08}) in real-world untampered storage workloads is generally {\em uniform} (i.e., different features correspond to the same number of chunks).
Specifically, we analyze two real-world datasets (see \S\ref{sub:datasets} for dataset details), and partition the stream of chunks in each dataset into multiple non-overlapped windows, such that each window includes $W$ consecutive chunks. We extract three ordered features for each chunk via N-transform \cite{shilane12}. For each window, we count the {\em normalized difference} (i.e., the absolute difference divided by the total number of chunks in the window) between the maximum and minimum numbers of chunks that share an identical $i$-th feature ($i=1, 2$ and $3$). A lower normalized difference of a window indicates that the corresponding feature distribution is more uniform.


Figure~\ref{fig:featureDistribution} presents the normalized differences of all windows (with the sizes of $W$ = 1\,K, 5\,K, and 10\,K, respectively) based on their first features (i.e., $i$ = 1). The results for the other features are similar, since features across different orders are unlikely to be identical (due to the principle of N-transform); hence we omit them here.
We observe that although the distribution of the normalized differences across different windows is skew, each window only has a small value (e.g., up to 0.035 in Linux and 0.005 in CouchDB) of normalized difference. When $W$ = 1\,K, 5\,K and 10\,K, the features are even shared by an identical number of chunks (i.e., uniform feature distribution) in 91.5\%, 58.3\% and 29.9\% CouchDB windows, respectively.


\begin{figure}
  \centering
  \begin{tabular}{cc}
    \includegraphics[width=1.55in]{pic/featurespy/plot/featureDistribution/featureDistributionLinux.pdf} &
                                                                                            \includegraphics[width=1.55in]{pic/featurespy/plot/featureDistribution/featureDistributionCouchbase.pdf} \\
    {\small (a) Linux} & {\small (b) CouchDB} \\
    \end{tabular}
  \vspace{-6pt}
  \caption{Normalized differences across different windows in two real-world datasets. The windows in the x-axis are sorted by their normalized differences.}
  \label{fig:featureDistribution}
  \vspace{-6pt}
\end{figure}


Thus, the strawman design detects the learning-content attack by {\em differentiating feature distributions}. Figure~\ref{fig:architecture}(a) presents the architectural workflow of the strawman design.
Since \sysnameF is co-located with the client that may tamper with unprotected memory, it protects the feature extraction and attack detection procedures via the SGX enclave, so as to honestly report the attack event. Specifically, it enters the enclave to extract the content features of each plaintext chunk and reports an attack if many chunks match identical content features (i.e., similar). If no attack is announced, it continues to perform source-based encrypted deduplication (\S\ref{sub:basics}). In the following, we address the design details.

%(note that we do not choose to protect all client-side operations via the enclave, due to the huge TCB

% Thus, we propose to differentiate normal and attack cases by examining feature distributions. Specifically, we monitor the processing of chunks, compare the content features of each chunk, and report an  attack event if the client inputs many  chunks that have identical features for processing.

%to be processed in a small time window.

\paragraph{Features extraction.}
We extract the content features of each plaintext chunk via {\em N-transform} \cite{shilane12}, which is widely used to
detect chunk-level resemblances. N-transform is defined based on $N$ (e.g., 12 by default) pairs of coefficients $(a_i, m_i)$, where $N$ indicates how many {\em sub-features} (based on which N-transform generates features, see below) to be extracted for each chunk. Specifically, for each plaintext chunk $M$, it uses Rabin fingerprinting \cite{rabin81} to compute many fingerprints over the 32-byte sliding windows of chunk data, and transforms the Rabin fingerprint $fp$ in each sliding  window as:
\begin{eqnarray}
  \label{eq:feature}
  \pi_i = a_i * fp + m_i \mod 2^{32},
\end{eqnarray}
where $i$ = 1, 2,\ldots, $N$. It derives the $i$-th sub-feature of $M$ as $fp$, if $fp$ leads to the {\em maximum} $\pi_i$ among other Rabin fingerprints. The rationale is that changes in small regions are likely to affect some Rabin fingerprints, among which only a few (as the sub-features) generate the maximum values of $\{\pi_i\}$. Thus, the majority sub-features remain stable for similar chunks.

N-transform computes a feature by combining multiple (e.g., four by default) consecutive sub-features together. It characterizes each chunk by a set $S$ (e.g., by default the number of features $|S| = 3$) of features, in order to mitigate the computational overhead of comparing many sub-features for similarity detection. Specifically, the more common features two chunks have, the more likely to be similar they are.


\paragraph{Attack detection.}
We manage a hash table in the enclave to track how many plaintext chunks share identical content features. Each entry of the hash table maps a feature to the number of times (four bytes) that the feature occurs across different chunks.
 In addition,
we define a window size $W$ (e.g., 5\,K by default) and periodically clear all table entries per processing $W$ chunks, such that the hash table only keeps the {\em recent} occurrences of features in a short time. Note that the hash table is up to 5\,K $\times$ 3 $\times$ (16 bytes + 4 bytes) $\approx$ 300\,KiB and adds negligible memory overhead to the SGX enclave, where each chunk has three features, and each feature is concatenated from four sub-features that takes four bytes each according to the default configuration of N-transform.

To examine each plaintext chunk, we query the hash table based on {\em each} of its content features. If a feature does not exist, we add the feature  into the hash table, and initialize the corresponding occurrence with one; Otherwise if the feature exists, we increment the corresponding occurrence by one. We report an attack if the occurrence reaches a pre-defined ratio $T$ (e.g., 3\% by default) of the window size $W$.




\subsection{FeatureSpy Overview}
\label{sub:secure_design}

One security limitation of the strawman design is that it is vulnerable to {\em bypassing} the attack detection procedure (Figure~\ref{fig:architecture}(a)). Specifically, the malicious client can directly inject its self-constructed chunks to be processed by the unprotected operations (e.g., key generation), in order to launch the learning-content attack (\S\ref{sub:basics}).


%The question is how to design the SGX-based architecture of \sysnameF, such that it not only protects the attack detection operation via SGX, but also prevents the adversary from bypassing the detection procedure.


\begin{figure}
  \centering
  \begin{tabular}{c}
    \includegraphics[width=3.25in]{pic/featurespy/naive.pdf} \\
    {\small (a) Strawman design} \vspace{5pt}\\
    \includegraphics[width=3.25in]{pic/featurespy/architecture.pdf} \\
    {\small (b) Secure design}
  \end{tabular}
  \vspace{-6pt}
  \caption{Architectural workflows of designs. The strawman design is vulnerable to bypassing the detection procedure. \sysnameF couples the attack detection procedure with SGX-based PoW \cite{ren21} in the enclave,  in order to prevent the bypassing.}
  \label{fig:architecture}
  \vspace{-6pt}
\end{figure}


\sysnameF builds on {\em SGX-based PoW} \cite{ren21} to prevent the client from escaping the detection procedure. Specifically, the SGX-based PoW approach \cite{ren21} takes each ciphertext chunk as input and enters the SGX enclave to compute the fingerprint of the ciphertext chunk, as well as a signature of the fingerprint. The cloud proceeds to check if the fingerprint corresponds to a stored ciphertext chunk (i.e., deduplication) only when the signature is successfully verified. Since the cloud responds based on {\em authenticated} fingerprints that are processed by the SGX enclave, a client cannot identify the existence of the chunks that it is unauthorized to access.

Thus, a na\"{i}ve approach is to enlarge the enclave in the strawman design to totally perform detection, key generation, encryption, and SGX-based PoW in the enclave. In this case, only the chunks processed by the procedures in the SGX enclave are accepted by the cloud for deduplication (due to SGX-based PoW), and the malicious client cannot bypass any procedure.
However, the approach incurs a huge TCB and introduces potential performance \cite{arnautov16, harnik18, dinhngoc19} and security \cite{lie05} issues.


\sysnameF proposes to {\em perform attack detection based on ciphertext chunks}, and couples only the detection procedure with SGX-based PoW in the enclave (Figure~\ref{fig:architecture}(b)), so as to reduce the size of TCB.

%SGX-based PoW is performed only on the ciphertext chunks, whose features are examined (in the attack detection procedure), such that a malicious client cannot bypass the detection procedure.

% The cloud responds only when the ownerships of the corresponding ciphertext chunks are successfully verified, such that a compromised client cannot identify the ciphertext chunks owned by other clients
% Since the deduplication patterns are only returned based on  the {\em examined} chunks by the detection procedure in the enclave, a compromised client cannot directly perform source-based deduplication based on
%  its self-constructed chunks  for bypassing the detection procedure.
% computes the fingerprint of each ciphertext chunk and generates a {\em signature} based on the fingerprint in the enclave, such that
% We  propose a {\em secure design} , which  couples
% Specifically, we now detect the learning-content attack based on ciphertext chunks (see below).
% If no exception is reported, the



\paragraph{Challenge.}
However, a new challenge is raised for {\em detecting similarity based on ciphertext chunks}, since the enclave now only views the encrypted data.  It is impossible to detect similarity after MLE, as MLE keys are derived from the {\em whole contents} of plaintext chunks (\S\ref{sub:basics}). This maps similar (but distinct) plaintext chunks to totally different ciphertext chunks, thereby destroying similarity.

% Recently, Wu {\em et al.} \cite{wu21} propose

% An alternative approach is {\em feature-based encryption (FBE)}, which performs encryption/decryption based on a key (called {\em feature key}) derived from the content features of each plaintext chunk. Since similar chunks only differ in a few data regions, they are likely to have the same feature key, and a number of common data blocks (each of which takes 16 bytes in AES) from the beginning of contents.
% Considering that encrypted deduplication \cite{douceur02, shah15} necessitate a fixed {\em initialization vector (IV)} (to preserve deterministic encryption),
% the first a few blocks in the corresponding ciphertext chunks are identical due to the {\em block-chaining property} of some popular block cipher modes (e.g., CBC and CFB \cite{dworkin01}), and similarity detection can be performed by examining such initial blocks (of each ciphertext chunk).


An alternative cryptographic primitive is {\em feature-based encryption (FBE)}, which performs encryption/decryption based on a key (called {\em feature key}) derived from the content features of each plaintext chunk.
Since similar chunks only differ in a few data regions, they are likely to  have the same feature key, and first a few data blocks (each of which takes 16 bytes in AES). Considering that  encrypted deduplication \cite{douceur02, shah15} necessitates a fixed {\em initialization vector (IV)} (to preserve deterministic encryption), FBE preserves the equality of the first a few data blocks of similar chunks (due to the {\em block-chaining encryption} property of the block cipher modes, such as {\em cipher block chaining (CBC)} and {\em cipher feedback (CFB)} \cite{dworkin01}), and similarity detection can be performed by comparing such data blocks in each ciphertext chunk.



% FBE supports to detect similar original plaintext chunks by comparing the first a few data blocks in the corresponding ciphertext chunks.
% FBE is likely to generate identical feature keys for similar plaintext chunks, and preserves similarity after encryption.
% For example, suppose that FBE uses the {\em ciphertext feedback  (CFB)} mode \cite{dworkin01} as the underlying encryption function. In addition to the symmetric key, the CFB mode encrypts the first block of each plaintext chunk based on an {\em initialization vector (IV)}, and each following block based on the ciphertext of the corresponding previous block. Considering that encrypted deduplication implementations \cite{douceur02, shah15} necessitate a fixed IV to preserve deterministic encryption, if two similar plaintext chunks (i.e., assigned with the same feature key) have $i$ common blocks from the beginning of contents, then the first $i$ blocks in the corresponding ciphertext chunks under the CFB mode are the same. It is feasible to detect similar plaintext chunks by examining the initial $i$ blocks of corresponding ciphertext chunks.


However, FBE is vulnerable to key compromise, since a feature key corresponds to a set of chunks that have identical content features. A malicious client can use its compromised feature keys to fully decrypt many chunks, even some of which are {\em beyond its access scope}.

% FBE is likely to generate identical feature keys for similar plaintext chunks, and preserves similarity after encryption.
% For example, suppose that FBE uses the {\em ciphertext feedback  (CFB)} mode \cite{dworkin01} as the underlying encryption function. In addition to the symmetric key, the CFB mode encrypts the first block of each plaintext chunk based on an {\em initialization vector (IV)}, and each following block based on the ciphertext of the corresponding previous block. Considering that encrypted deduplication implementations \cite{douceur02, shah15} necessitate a fixed IV to preserve deterministic encryption, if two similar plaintext chunks (i.e., assigned with the same feature key) have $i$ common blocks from the beginning of contents, then the first $i$ blocks in the corresponding ciphertext chunks under the CFB mode are the same. It is feasible to detect similar plaintext chunks by examining the initial $i$ blocks of corresponding ciphertext chunks.
% %the common initial {\em blocks} (each of which takes 16 bytes for AES) of such chunks under some block cipher modes.
% %For example, in addition to the symmetric key,



This poses a dilemma in choosing a proper cryptographic primitive: MLE is robust against key compromise (i.e., a compromised MLE key cannot  be used to decrypt other chunks except the corresponding one) but destroying similarity, while FBE preserves the similarity of original chunks but vulnerable to key compromise.



\subsection{Similarity-preserving Encryption}
\label{sub:spe}


We propose {\em similarity-preserving encryption (SPE)}, which builds on the MLE key and the feature key, simultaneously, in order to mitigate key compromise via the MLE key while preserving similarity via the feature key.
Specifically, due to the limited content differences of similar chunks, SPE samples a small part (e.g., the first 32 bytes that take 0.4\% of an 8\,KiB chunk) from each plaintext chunk, called the {\em similarity indicator}, such that the similarity indicators of similar chunks are likely to be the same. It encrypts the similarity indicator of each plaintext chunk with the corresponding feature key, while the remaining large part (that takes 99.6\% of an 8\,KiB chunk) of chunk content with the MLE key. Then, we can  detect similarity based on ciphertext chunks by verifying if the corresponding (encrypted) similarity indicators (e.g., the first 32 bytes of each ciphertext chunk) are identical.
Note that SPE does not degrade the storage efficiency of cross-user encrypted deduplication (\S\ref{sub:basics}), since identical chunks share the same content features (and hence feature keys) and MLE keys.

We argue that SPE  mitigates the information leakage of FBE against key compromise. The insight is that the feature key of a compromised chunk $M$ can only be used to decrypt the similarity indicators of the  chunks similar to $M$. However, since such chunks are likely to have the same similar indicator with $M$, SPE does not incur additional information leakage. Even a rare chunk (that is similar to $M$) has a different similarity indicator, the information leakage is dramatically reduced compared to FBE, and subject to the size of the similar indicator.
In the following, we present the design details of how we generate the feature key for SPE.




% SPE encrypts a small part (called {\em similarity indicator} that 32 bytes or 0.3\% of an 8\,KiB chunk) of each plaintext chunk with the feature key while the remaining large part (called {\em trimmed chunk} that takes 99.6\% of an 8\,KiB chunk) of the chunk content with the MLE key.
% Due to the limited content differences of similar chunks,  are likely to also have the same indicator due to their limited differences on chunk contents.
% it is feasible to
% Since similar chunks have a large fraction of duplicate content, they are likely to have the same indicator in addition to the feature key.
% since similar chunks only differ in a few regions,  we can sample
% Thus, we
% We can detect similarity based on ciphertext chunks by verifying if the corresponding (encrypted) indicators are identical. Specifically,
%  \sysnameF now tracks the occurrences of indicators (rather than features in the strawman design in \S\ref{sub:basic}), and reports an exception if the occurrence of some indicator exceeds a pre-defined ratio of the window size (\S\ref{sub:basic}). In addition, since the remaining large part is protected by the MLE key, the information leakage due to key compromise is subject to the size of the indicator.
% Note that SPE does not degrade the storage efficiency of cross-user encrypted deduplication (\S\ref{sub:basics}), since identical chunks share the same content features (and hence feature keys) and MLE keys.
% In the following, we address how the feature key is derived for SPE.




%We propose {\em similarity-preserving encryption (SPE)}, which builds on the MLE key and the feature key, simultaneously, such that it can mitigate key compromise via the MLE key while preserving similarity via the feature key.
%In the following, we present how we generally derive the feature key, followed by the SPE design.
% which augments MLE with similarity preservation while mitigating the key compromise of FBE. Our idea is to  Specifically, since similar chunks only differ in a few regions,  we can sample a small part  in a fixed position (e.g., in the beginning of chunk content), called , such that the indicators of similar chunks  are likely to be the same. Thus, we encrypt the indicator of each plaintext chunk with its feature key, while
%  Specifically,
%  \sysnameF now tracks the occurrences of indicators (rather than features in the strawman design in \S\ref{sub:basic}), and reports an exception if the occurrence of some indicator exceeds a pre-defined ratio of the window size (\S\ref{sub:basic}). In addition,
%preserves similarity via the feature key while builds on the MLE key to mitigate the leakage of sensitive information against key compromise.



\paragraph{Feature key generation.}
Recall that we extract three content features for each plaintext chunk via N-transform (\S\ref{sub:basic}), and a basic key generation approach (called {\tt allFeature}) is to concatenate all features, and compute the feature key based on the cryptographic hash of the concatenation result. However, {\tt allFeature} fails to generate identical feature keys for many similar chunks. Specifically, if the content difference of the similar chunks is exactly in the sliding window that yields a sub-feature (\S\ref{sub:basic}), the chunks will have distinct feature keys, and it is impossible to detect the similarity between them even if the remaining original contents are identical.

We build on the {\em sampling-based approach} \cite{bhagwat09, dong11, qin17} to relax the key generation criteria. The idea is to sample a {\em representative} feature for each plaintext chunk, and compute the feature key based on the sampled feature. This tolerates content differences, and generates identical feature keys for a wide range of plaintext chunks.

We consider two ways to choose the representative feature. Specifically, {\tt firstFeature} generates a feature key based on the {\em first} feature of each plaintext chunk. The advantage is that we do not need to compute subsequent sub-features and features (\S\ref{sub:basic}), and mitigate the computational overhead \cite{zhang19} of N-transform.

We also consider {\tt minFeature}, which builds on {\em Broder's theorem} \cite{broder97} to generate the feature key based on the {\em minimum} feature (i.e., have the minimum value in all features) of each plaintext chunk. Specifically, Broder's theorem states that if two plaintext chunks have many common features (i.e., similar with a high probability), they are likely to share the same minimum feature. That is:
\begin{eqnarray}
  \label{eq:broder}
 \Pr[\min(S_1) = \min(S_2)] = \frac{|S_1 \cap S_2|}{|S_1 \cup S_2|},
\end{eqnarray}
where $S_i$ is the set of features for a plaintext chunk $M_i$, and $\min(S_i)$ returns the minimum value of the features in the set $S_i$ ($i = 1, 2$). Thus, {\tt minFeature} tends to generate identical feature keys for the highly similar chunks. In \S\ref{sec:evaluation}, we will study the trade-off of different key generation approaches of SPE.

Note that in addition to working on features, an alternative design is to generate a feature key based on the first or the minimum {\em sub-feature}. We do not choose this design, due to the low entropy of a sub-feature. According to the original configuration \cite{shilane12} of N-transform, each sub-feature has only four bytes, and an adversary can efficiently launch the brute-force attack by  enumerating all possible sub-features/keys (up to $2^{32}$). On the other hand, in \S\ref{sec:implementation}, we re-configure N-transform to ensure that the key space based on features is large enough (e.g., $2^{256}$) against the brute-force attack.


% \paragraph{SPE design.}
% The basic idea is to sample a small part (e.g., 32 bytes or 0.39\% of an 8\,KiB chunk), as the {\em similarity indicator}, of each plaintext chunk, and encrypt the small part with the feature key while the remaining large part (called {\em trimmed chunk} that takes 99.61\% of an 8\,KiB chunk) of chunk content with the MLE key. It is feasible to detect similarity based on ciphertext chunks by verifying if the corresponding (encrypted) indicators are identical.
% Our insight is that if some bytes (e.g., public content patterns) are shared by many chunks, then such bytes are likely to be {\em non-sensitive}. Thus, the idea of SPE is to build on a few non-sensitive bytes to expose the similarity of original plaintext chunks, while preserving high security (i.e., against key compromise) of the remaining sensitive contents.
% Basically, it encrypts a small part , called the {\em indicator}, of each plaintext chunk with the feature key, while the . It can
% However,
% Also, since the remaining large part is protected by the MLE key, the information leakage due to key compromise is subject to the size of the indicator.
% Figure~\ref{fig:encryption} presents the design of SPE. Instead of using the output of the key generation instances (e.g., {\tt allFeature}, {\tt firstFeature} and {\tt minFeature}) to encrypt the indicator, it computes the feature key as the cryptographic hash of the concatenation of the and the indicator. It then
% Specifically, we sample a small part (e.g., 32 bytes or 0.39\% of an 8\,KiB chunk) from each plaintext chunk in a fixed position (e.g., in the beginning of the chunk content), called the {\em indicator}, and protect the indicator of each plaintext chunk with its feature key, while the remaining large part (that takes 99.61\% of an 8\,KiB chunk) of the chunk content with the MLE key. We can detect similarity based on ciphertext chunks by verifying if the corresponding (encrypted) indicators are identical.
% \begin{figure}
%   \centering
%   \includegraphics[width=3in]{pic/featurespy/encryption.pdf}
%   \vspace{-6pt}
%   \caption{Design of SPE. $\mathbf{H}(\cdot)$ and $\mathbf{E}(\cdot)$ are the hash and encryption functions, respectively.}
%   \vspace{-6pt}
%   \label{fig:encryption}
% \end{figure}
% since similar chunks only differ in a few regions,  such that the indicators of similar chunks  are likely to be the same (i.e., non-sensitive). Thus, we
% , which augments MLE with similarity preservation while mitigating the key compromise of FBE. Our idea is to build on the MLE key and the feature key, simultaneously, such that we can mitigate key compromise via the MLE key while preserving similarity via the feature key. Specifically,
%  \sysnameF now tracks the occurrences of indicators (rather than features in the strawman design in \S\ref{sub:basic}), and reports an exception if the occurrence of some indicator exceeds a pre-defined ratio of the window size (\S\ref{sub:basic}). In addition, since the remaining large part is protected by the MLE key, the information leakage due to key compromise is subject to the size of the indicator.
% Note that SPE does not degrade the storage efficiency of cross-user encrypted deduplication (\S\ref{sub:basics}), since identical chunks share the same content features (and hence feature keys) and MLE keys.
% In the following, we address how the feature key is derived for SPE.







%In \S\ref{subsub:secure_details}, we show how we can further enhance the protection of the indicator against key compromise.
%  presents the architecture of \sysnameF. It processes plaintext chunks in unprotected memory to extract features, generate keys and perform chunk-based encryption. In the SGX enclave, it performs attack detection, followed by PoW, such that the source-based deduplication is applied on the chunks that have been processed by the enclave.
% Figure~\ref{fig:architecture}(b) presents the architectural workflow of the secure design, which generates feature keys and MLE keys to perform SPE. In In the following,
% \paragraph{Similarity-preserving encryption.}
% We choose the indicator as the first $L$ bytes of each plaintext chunk.
% Figure~\ref{fig:encryption} presents the workflow of the similarity-preserving encryption, which combines FBE and MLE to preserve similarity after encryption (\S\ref{subsub:secure_idea}). Instead of directly encrypting the indicator with the feature key, we now hash the concatenation of the indicator and the feature key to form a hash key, and further use the hash key to encrypt the indicator. Our rationale is to ensure that a compromised feature key is insufficient to decrypt the indicator. Also,
% Specifically, the the feature key and the MLE key as input.

% \begin{figure}
%   \centering
%   \includegraphics[width=3in]{pic/featurespy/encryption.pdf}
%   \vspace{-5pt}
%   \caption{Similarity-preserving encryption. $\mathbf{H}(\cdot)$ is a cryptographic hash function, and $\mathbf{E}(\cdot)$ is an encryption function.}
%   \vspace{-5pt}
%   \label{fig:encryption}
% \end{figure}


\subsection{Security Discussion}
\label{sub:security}

We discuss the confidentiality of \sysnameF (in particular SPE, which is the underlying encryption primitive), followed by its robustness against a malicious client.

\paragraph{Confidentiality against a compromised cloud.}
We have discussed the security improvement of SPE over FBE when keys are compromised (\S\ref{sub:spe}), and now
focus on the case that the keys remain secure. Our goal is to show that SPE preserves the security of encrypted deduplication against a compromised cloud. Specifically,
since SPE performs encryption via both MLE and FBE, its security is reduced to those of MLE and FBE. The previous work \cite{bellare13a} has shown
that MLE ensures that any  polynomial-time adversary cannot distinguish the ciphertext of a plaintext chunk from a random value (i.e., PRV\$-CDA), when the plaintext chunk is
drawn from a large space such that it is difficult to predict any chunk chosen from the space (i.e., {\em unpredictable}). In the following, we (informally) show that FBE preserves the security  of MLE if features are unpredictable.


We consider FBE as a generic form of MLE. If we treat the sampled feature(s) (e.g., the representative feature for {\tt firstFeature} and {\tt minFeature}, and all features for {\tt allFeature}) as a message $M$, then FBE actually uses the cryptographic hash of $M$ to encrypt the corresponding expansion $f(M)$ (i.e., plaintext chunk), where $f(\cdot)$ is an expansion function. Since each $M$ is unpredictable, the secret key  remains confidential against a polynomial-time adversary. Then, an adversary cannot distinguish a ciphertext from a random value, if the underlying symmetric encryption is secure.

% In addition, SPE performs encryption via both FBE and MLE, and hence its security relies on the primitives. Specifically, assuming the existence of an adversary that breaks SPE  with an advantage $\epsilon$, by triangle inequality, one can construct either an adversary that breaks either FBE  or MLE with an advantage at least $\epsilon/2$.

%{\bf TODO: Pls add informal security analysis here}
% First we note that the FBE is slightly different from MLE, in the sense that it uses the hash value of the sampled feature rather than the plaintext as a secret key, while this does not spoil the privacy due to the following reason. Since the plaintext chunks can be treated as (expanded) information on the feature, an FBE is essentially an MLE encrypting related information on a message (which is the feature in our case) embedded into the secret key, rather than encrypting the message itself. One can see that the MLE instantiated by Bellare et al. \cite{bellare13a} can do this without loosing privacy: as long as the feature is unpredictable, the secret key remains hidden and thus an adversary cannot distinguish a ciphertext from randomness due to the security of the underlying symmetric encryption. Therefore, the adversary learns little information on the plaintext chunks encrypted by the FBE.
% %Assuming that the feature is unpredictable, any polynomial-time adversary learns little information on the feature key and thus unable to distinguish a ciphertext of the FBE scheme from randomness.
% Moreover, the rest plaintext chunks encrypted by the MLE are only dependent of the MLE ciphertexts and the feature keys, while neither benefits the adversary due to the security of the MLE scheme and the unpredictability of the feature.
% As a result, the resulting SPE remains confidential under the security of the underlying FBE and MLE.


% Notice that in our case, we use the MLE scheme in a slightly different way, in the sense that we sometimes use the hash value of the sampled feature as a secret key, while this does not spoil the privacy. One can see that the sampled feature is unpredictable and the plaintext chunks can be treated as (expanded) information on the feature. Therefore, our scheme remains confidential as long as the MLE scheme can encrypt information on a message embedded into the secret key (rather than the message itself) without loosing privacy. This is the case for the instantiations in \cite{bellare13b}. The security proof proceeds in exactly the same way as the original one in [Appendix B]\cite{bellare13b} except for changing "$\mathcal{SE}(\mathbf{L}[i],\mathbf{M}[i])$" to "$\mathcal{SE}(\mathbf{L}[i],f(\mathbf{M})[i])$" and letting the message distribution $\mathcal{M}$ additionally outputs $f(\mathbf{M})$, where $f$ is an arbitrary polynomial-size function and $f(\mathbf{M})$ can be treated as arbitrary information on the hashed message. We refer the reader to [Appendix B]\cite{bellare13b} for the detailed proof.

% We further argue that while a sampled feature key might be dependent of  other plaintext chunks encrypted in a "standard way" by using the original MLE scheme, an adversary obtains no additional advantage on this, since the feature is unpredictable and thus its ciphertext reveals no information on other chunks.

Note that our informal analysis relies on the security assumption that the features are unpredictable, while we can mitigate the unpredictable assumption via server-aided key generation \cite{bellare13b} (see \S\ref{sec:implementation} for how we deploy \sysnameF in an existing encrypted deduplication system that addresses the unpredictable assumption).

%In \S\ref{sec:implementation}, we show how we integrate our design into an existing encrypted deduplication system \cite{ren21}.


% \subsubsection{Confidentiality}
% The confidentiality of our scheme is protected by the underlying MLE scheme instantiated with a symmetric encryption and a hash function as in \cite{bellare13b}. Recall that in an MLE scheme, a secret key is a hash value of the plaintext. The privacy (against chosen-distribution attacks) of an MLE scheme says that any polynomial-time adversary cannot distinguish honestly generated ciphertexts of unpredictable plaintexts from randomness, which provides us with confidentiality. Assuming that the hash function is modeled as an ideal one (i.e., a random oracle), for any adversary $\mathcal A$, there must exist polynomial-time adversaries $\mathcal B_1$ and $\mathcal B_2$ such that
% $$
% \mathbf{Adv}_{\mathcal A}^{prv-cda}\leq qm\cdot \mathbf{Adv}_{\mathcal B_1}^{kr}+2\cdot \mathbf{Adv}_{\mathcal B_2}^{ror}+\frac{4m^2}{2^k}+\frac{qm}{2^\mu}.
% $$
% Here, $q$ is the number of queries made to the hash function, $\mu$ is the min-entropy of the plaintext, $m$ is the number of plaintexts, $\mathbf{Adv}_{\mathcal A}^{prv-cda}$ is the advantage of ${\mathcal A}$ in breaking the privacy of the MLE scheme,
% and $\mathbf{Adv}_{\mathcal B_1}^{kr}$ and $\mathbf{Adv}_{\mathcal B_2}^{ror}$ are respectively the advantages of ${\mathcal B_1}$ and ${\mathcal B_2}$ in breaking the key recovery (KR) security and the one the one-time real-or-random (otROR) security of the symmetric key encryption scheme. Hence, if the underlying symmetric encryption scheme is simultaneously KR and otROR secure, which is widely believed for the instantiations given in \cite{bellare13b}, our resulting protocol remains confidential.


\paragraph{Robustness against a malicious client.}
We have discussed that an adversary cannot bypass the detection procedure (\S\ref{sub:secure_design}), and now focus on the robustness against other malicious actions (\S\ref{sec:setting}).


\begin{itemize}[leftmargin=*]
\item  {\bf Case 1: Tampering with unprotected procedures.}
  \sysnameF protects the detection procedure via SGX, yet a malicious client can tamper with the unprotected procedures. It may manipulate the features, similarity indicators, and keys, in order to cheat \sysnameF.
  We argue that these manipulations do not help learn contents from the other honest clients that follow our design, since the manipulations lead to a distinct ciphertext chunk with that produced by SPE (applied by an honest client), even if the original plaintext chunks are the same. This {\em prohibits deduplication}, and the learning-content attack that relies on the leakage of deduplication patterns is impossible.



%  deduplication is now {\em disabled} between the target ciphertext chunk (that is generated by the honest client) and any tampered one. The learning-content attack




  % key generation and encryption procedures. It may manipulate the similarity indicators of plaintext chunks (e.g., swap the similarity indicators of different plaintext chunks) or randomize ciphertext chunks (e.g., perform encryption via random keys), in order to cheat \sysnameF.
  % Specifically, a honest client should perform SPE on real plaintext chunks, thereby prohibiting deduplication between the target ciphertext chunk of the honest client and any manipulated ciphertext chunk. Thus, the learning-content attack is impossible.
  % In other words, the malicious client cannot exploit the existence of ciphertext chunk, and the learning-content attack is impossible. The reason is that the manipulation of the above contents prohibits
% while the compromised client uses its artificial keys or contents. This prohibits deduplication  between the target ciphertext chunk of the honest client and any manipulated ciphertext chunk  generated by tampered operations, and hence the learning-content attack is impossible.

% \item {\bf Case 2: Crafting feature distributions.}
%   Since \sysnameF examines the feature distribution of each window of chunks, an adversary may craft chunks to make the feature distribution seem realistic. Specifically, to shield a group of $x$ similar faked chunks, it creates another $W/x - 1$ chunk groups, such that each group includes $x$ chunks that share identical features and the content features of the chunks across different groups are  totally different, where $W$ is the number of chunks in a processing window. It randomly places all (i.e., $W/x$ groups in total) chunks in the window, in order to flatten the overall feature distribution.
%   Our design  generally detects this action, since the maximum number of  chunks (that have identical features) in each group is subject to a threshold $T$ (\S\ref{sub:basic}). To escape detection, the only way is to only submit a few chunks  (i.e., $x < T$) that have identical features in each processing window, yet slowing down the attack.

\item {\bf Case 2: Tampering with data processing.}
  Since \sysnameF monitors data similarity in the granularity of each processing window, a malicious client may tamper with an in-processing data stream, and carefully insert similar chunks (for attack), such that each window just includes a small fraction of  the similar chunks.  Although \sysnameF cannot detect the attack in this case, we argue that it slows down the learning-content attack. For example, without \sysnameF, the malicious client can full fill each window with similar chunks (i.e., the number is $W$) for attack, yet \sysnameF ensures that it can submit up to $W \times T$ similar chunks (otherwise it is detected). The attack cost is reduced by $1 / T$ times. By configuring a small $T$, we can make the learning-content attack too expensive to be impractical; the trade-off is that the possibility of misjudgement (i.e., report an attack in a normal case) is increased.


  %Also, it is feasible to still detect the learning-content attack by configuring a small $T$, yet introducing possibilitis of false positives.
\end{itemize}



\paragraph{Limitation.}
\sysnameF detects the learning-content attack from each {\em independent} client. However, an adversary may compromise multiple clients to cooperatively launch the learning-content attack. Now the number of similar chunks uploaded by each client is greatly reduced, and \sysnameF may not effectively detect the exception. We pose the defense against the cooperative learning-content attack as our future work.


% can schedule each compromised client to upload a subset of faked chunks, and
%  \sysnameF prevents abusing deduplication patterns, but it cannot protect against the leakage of deduplication patterns. In other words, when performing source-based deduplication, an adversary can still use its fully accessible file to learn the information that if someone has uploaded the same file.
%  We pose the full protection of deduplication patterns as our future work.


% Our rationale is to ensure that only the ciphertext chunks examined by \sysnameF can be processed by source-based deduplication, so as to
% prevent the malicious client from  directly submitting the fingerprints that are generated by itself. Also, this
% mitigates the threat of using a fingerprint to convince the ownership of the corresponding ciphertext chunk \cite{halevi11, ren21}.



% We design \sysnameF to perform attack detection {\em based on ciphertext chunks} that will be directly processed by source-based deduplication, such that the adversary cannot bypass the detection procedure. However,

% \paragraph{Design challenge.}

% \paragraph{Main idea.}





% \paragraph{Na\"{i}ve approach.}





% \paragraph{Our approach.}

% To address this issue, our idea is to generate the secret key of each chunk {\em based on its features} (rather than the whole chunk content in MLE \cite{bellare13a}), such that similar chunks are likely to have identical secret keys. To perform encryption, we use a special {\em bloCk cipher mode} \cite{dworkin01} (e.g., the {\em ciphertext feedback (CFB) mode} by default, see \S\ref{sub:detection} for detailed properties), such that the $i$-th ($i>1$) ciphertext block of the resulting ciphertext chunk only depend on the secret key and each previous/current $j$-th original plaintext block, where $j$ = 1, 2, $\ldots$, $i$. In other words, if two similar plaintext chunks have $i$ common plaintext blocks in the beginning of contents, then the first $i$ ciphertext blocks in the  corresponding ciphertext chunks are identical (assuming that the similar plaintext chunks have the same key). Thus, \sysnameF examines a number of {\em prefix} bytes of each ciphertext chunk, and deduces that two plaintext chunks are similar if their ciphertext chunks share the same prefix bytes.
% Since similar chunks differ in a few regions that are unlikely to be just in the beginning of each chunk, \sysnameF can detect most of similar chunks and successfully report the attack with a high probability.




% \paragraph{Architecture.}
% Figure~\ref{fig:architecture}(b) presents the architecture of \sysnameF.

% Here, we adopt the {\em SGX-based PoW} approach \cite{ren21} to further process ciphertext chunks if no exception is detected. Specifically,


% \subsection{Key Generation}
% \label{sub:keygen}

% \sysnameF builds on N-transform \cite{shilane12} to extract features from plaintext chunks. We introduce N-transform, followed by how we generate keys based on the N-transform features.




% \paragraph{Key generation schemes.}


% \subsection{Attack Detection}
% \label{sub:detection}

% \sysnameF detects the learning-remaining-content attack by examining the ciphertext chunks that are encrypted by the {\em ciphertext-feedback (CFB) mode} \cite{dworkin01}.

% %We introduce the chaining-based block cipher mode, followed by our design details.


% \begin{figure}
%   \centering
%   \subfigure[Encryption]{
%     \includegraphics[height=.71in]{pic/featurespy/encryption.pdf}
%   }
%   \subfigure[Decryption]{
%     \includegraphics[height=.71in]{pic/featurespy/decryption.pdf}
%   }
%   \vspace{-5pt}
%   \caption{Encryption and decryption of CFB mode. The encryption of each block recursively depends on previous blocks. The last block that does not need to fit in the block size can still be processed without padding.}
%   \vspace{-5pt}
%   \label{fig:cfb}
% \end{figure}

% \paragraph{CFB mode.}
% The CFB mode extends block cipher to encrypt data that includes multiple blocks (e.g., each block has a fixed size of 16 bytes). Figure~\ref{fig:cfb} presents the encryption and decryption procedures of the CFB mode, which originally uses a freshly random {\em initialization vector (IV)} to process each chunk. We follow previous systems \cite{douceur02, shah15} to use a fixed IV, so as to enable encrypted deduplication.

% To encrypt a chunk $M$, the CFB mode first divides $M$ into $n$ blocks $M_1, M_2, \ldots, M_n$. To encrypt each block $M_i$, if $M_i$ is the first block (i.e., $i = 1$), it encrypts IV and XORs the encryption result with $M_1$: $C_1 = \mathbf{E}(K, IV) \oplus M_1$; otherwise if $M_i$ is not the first block (i.e., $i > 1$), it encrypts the previous ciphertext block $C_{i-1}$, and computes  $C_i = \mathbf{E}(K, C_{i-1}) \oplus M_i$, where $\mathbf{E}(\cdot)$ is the encryption function, $K$ is the feature-based key (\S\ref{sub:keygen}), and $\oplus$ is the bitwise XOR operation. To perform decryption, it recovers the first plaintext block as $M_1 = \mathbf{E}(K, IV) \oplus C_1$, and each following block as $M_i = \mathbf{E}(K, C_{i-1}) \oplus C_i$ ($i>1$). Since the encryption of each block in the CFB mode {\em recursively depends} on its previous blocks, we can
% examine the prefix bytes of ciphertext chunks to find similar original plaintext chunks (\S\ref{sub:overview}). Note that the {\em cipher block chaining (CBC)} and {\em output feedback (OFB)} modes also have the recursive dependency property, and fit our design.




% In addition to recursive dependency, our design choice is due to the {\em padding free} property of the CFB mode.
% If a chunk cannot be evenly dividible into blocks, the CFB mode does not need to pad the last block of the chunk to fit the block size. Instead, it cuts the encryption result (i.e., $\mathbf{E}(K, C_{n-1})$) of the penultimate ciphertext block into the same size of $M_n$ (Figure~\ref{fig:cfb}(a)), so as to enable the bitwise XOR operation (see above). This mitigates the management overhead of paddings, especially when processing  chunks that have varying sizes.

% \paragraph{Detailed design of attack detection.}



% % With a small $L$, it detects similar chunks even when the changes occur near the beginning of contents, but introduces more false positives, since the ciphertexts of unsimilar chunks may randomly share common prefix in a few bytes. With a large $L$, it achieves a high precision, yet cannot detect similar chunks, which differ in the first a few bytes. In Exp\#XXX, we will study the impact of different $L$.


% % examines similarity based on ciphertext chunks in the PoW enclave. Here, we choose to inspect the {\em $L$-byte prefix} (i.e., the first $L$ bytes, where $L$ = XXX by default) of each ciphertext chunk, and report an attack if many ciphertext chunks share the same $L$-byte prefix. The design rationality  is from two sides. On the one hand, similar chunks are likely to be encrypted with identical keys (see above), and their ciphertext chunks share the same $L$-byte prefix with a high probability (e.g., when the changes between similar chunks occur after the first $L$ bytes).
% % On the other hand, we cannot examine the subsequent blocks (in contrast to the prefix) of ciphertext chunks only, since they depend not only on the corresponding plaintext blocks and MLE key, but also on the previous plaintext/ciphertext blocks (e.g., the CBC encryption mode).
% % To detect the pattern-based attack, the PoW enclave

% \subsection{Security Discussion}
% \label{sub:security}

\section{Implementation}
\label{sec:implementation}
我们使用 OpenSSL 1.1.1l \cite{openssl}、SGX SDK 2.15 \cite{sgxsdk} 和 SGX SSL \cite{sgxssl} 实现 \sysnameF,并将其部署到现有的基于 SGX 的加密重复数据删除系统 {\em SGXDedup} \cite{ren21} 来提高其针对学习内容攻击的安全性。具体来说,SGXDedup除了每个客户端和云端之外,还维护一个{\em key server}来管理一个全局secret,并根据chunk指纹和全局secret生成每个明文chunk的MLE key,从而对 {\em 离线暴力攻击} 具有鲁棒性(即解决了不可预测的假设)\cite{bellare13b}。为了加速基于源的加密重复数据删除,它在密钥服务器中部署了一个 SGX 安全区,并执行 {\em 推测加密} \cite{eduardo19} 以减轻(服务器辅助)密钥生成的在线计算开销。此外,它还部署了一个客户端安全区来执行基于 SGX 的高效 PoW (\S\ref{sub:secure_design})。请注意,SGXDedup 无法抵御学习内容攻击(\S\ref{sub:attack}),我们的新原型(称为 \prototype)是为了增强 SGXDedup 以抵御学习内容攻击的安全性。目前,\prototype(包括底层的 SGXDedup)由 C++ 中的 16\,K LoC 组成。下面,我们重点介绍与 \prototype 相关的实现细节。


\paragraph{设置。}
初始化后,\prototype 遵循 SGXDedup,通过 NIST P-256 椭圆曲线中的 {\em Diffie-Hellman 密钥交换 (DHKE)} 在云和每个客户端 enclave 之间共享一个 {\em proof key}。证明密钥在基于 SGX 的 PoW 中用于生成和验证签名(见下文)。


\paragraph{密钥生成。}
\prototype 适用于通过 Rabin 指纹 \cite{rabin81} 生成的可变大小明文块,最小、平均和最大大小分别为 4\,KiB, 8\,KiB 和 16\,KiB。为了在特征键 (\S\ref{sub:spe}) 上提供足够的熵,它将滑动窗口大小和 N 变换中的模数分别配置为 64 字节和 $2^{64}$。它生成 12 个子特征(每个 8 个字节),并将四个子特征的串联散列以形成每个特征(32 个字节)。它根据来自密钥服务器(如 SGXDedup \cite{ren21})的每个明文块的指纹请求 MLE 密钥,以及基于采样特征的特征密钥(根据 {\tt firstFeature},{\ tt minFeature} 或 {\tt allFeature})。


\paragraph{安全区操作。}
加密后(通过 AES-CFB-256 实现),\prototype 将 4,096 个密文块分批到 enclave 中进行处理,以减轻 SGX 上下文切换开销 \cite{arnautov16}。它跟踪哈希表中相似性指标的出现,并将窗口大小 $W$ 和比率阈值 $T$ 的默认值分别配置为 5\,K 和 3\%。

\prototype 遵循 SGXDedup 将基于源的重复数据删除与基于 SGX 的 PoW 相结合。它通过证明密钥基于 4,096 个指纹(密文块)的串联生成签名(通过 AES-CMAC 实现),以便云使用相同的密钥验证这些指纹的真实性。云通过LevelDB\cite{leveldb}实现指纹索引(\S\ref{sub:basics}),并通知客户端只传输非重复密文块。

\paragraph{存储管理。}
云端以 8\,MiB {\em 容器} 为单位管理非重复密文块,以减轻磁盘 I/O 开销。对于每个文件,它管理一个 {\em file recipe},其中列出了密文块的指纹,以及相应的 MLE 密钥和特征密钥。为保护机密性,每个客户端在将其外包到云端之前,都会使用单独的主密钥对文件配方进行加密。

要下载文件,客户端首先从云端检索文件配方,并使用相应的主密钥对其进行解密。然后,客户端根据文件配方检索密文块,并根据 MLE 密钥和功能密钥对其进行解密。

\paragraph{优化。}
我们应用标准方法来提高 \prototype 的性能。每个客户端在不同线程中提取多个明文块的内容特征,并在管道中并行处理分块、密钥生成、加密、PoW和上传。此外,为了提高下载性能,云会在内存中维护一个最近最少使用的缓存 (1\,GiB) 来保存最近恢复的容器。对于每个下载请求,它首先在缓存中搜索容器,并且仅当容器不在缓存中时才从磁盘中检索容器。
\section{实验分析}
\label{sec:featurespy-evaluation}

本文进行了广泛的评估来研究\sysnameF 对推测内容攻击的检测的有效性及相关影响因素(\S\ref{subsec:featurespy-evaluation-detection});原型系统\prototype 在合成数据集(\S\ref{subsec:featurespy-syn})和真实世界数据集工作负载下的性能(\S\ref{subsec:featurespy-real})。本文将主要实验结果总结如下:

\begin{itemize}
  \item \sysnameF 根据相应的密文数据块可有效地找到相似的明文数据块。例如,当明文数据块具有中等程度的差异时,它可检测到多达80.2\%的相似数据块。
  \item 即使攻击者将枚举的伪造文件混合于正常文件间,\sysnameF 仍可有效地检测到推测内容攻击。同时,它对真实世界数据集(正常文件)的误判率极低(例如,默认配置下为0)。
  \item \prototype 在处理大规模真实世界数据时,相较于\sysnameS (不支持\sysnameF)仅产生有限的上传(例如,8.8\%)和下载(例如,0.8\%)性能开销。
  \item 与现有的\cite{harnik2010side, li15}方案相比,\prototype 可以在安全抵御推测内容攻击的同时,实现最高的网络带宽效率(例如,节省高达98.9\%的网络流量)。
\end{itemize}

\subsection{数据集简介}
\label{subsec:featurespy-datasets}

\paragraph*{合成数据集。}

本文考虑三种类型的合成数据集进行评估。首先,本文基于受控随即修改数据块的方式生成第一个合成数据集SYNChunk$(x, y)$。具体来说,本文首先创建一个8\,KiB的基本数据块$M$。随后在该数据块$M$中随机选择$x$个位置,并在每个位置连续修改$y$个字节(随机修改内容)以创建一个与基本数据块相似的修改数据块$M'$。重复10\,K次受控随机修改后,本文得到随机相似数据块数据集SYNChunk$(x, y)$(包括从同一个基本数据块$M$修改得到的10\,K个相似数据块)。本文使用SYNChunk$(x, y)$研究数据块相似性检测的有效性(Exp\#1和Exp\#2)。

此外,本文创建随机文件数据集SYNFile$(x, y)$,其中包含指定个数个大小为4\,KiB的相似文件。本文假设SYNFile$(x, y)$中的文件包含$x$个未知变量(其值将被作为推测内容攻击中的攻击目标),并且每个变量的值是从大小为$y$的信息空间中随机选择的。本文使用SYNFile$(x, y)$作为目标文件来研究\sysnameF 在不同情况下对推测内容攻击的检测有效性(Exp\#4)。本文不考虑更大的目标文件,这是因为攻击者需要枚举更多相似的数据块(用于发起攻击)并且更有可能被\sysnameF 捕获。

此外,本文产生一组大小为2\,GiB的非重复随机文件数据集SYNUnique,其中每个文件中的数据块在数据集中均为非重复数据块。本文使用SYNUnique对\prototype 的性能进行压力测试(\S\ref{subsec:featurespy-syn})。

\paragraph*{真实世界数据集。}本文采用四个真实世界数据集,其特征总结在表~\ref{tab:featurespy-datasets}中: 

\begin{enumerate}
    \item \textbf{FSL}\cite{fsl},其中包括795个2013年1月22日至6月17日之间9名学生主机的home目录快照。该数据集仅包含数据块的元数据(例如,数据块指纹、大小、进入快照的顺序等)。
    \item \textbf{MS}\cite{meyer2011deduplication},其中包括 143个windows文件系统快照,每个快照的逻辑大小约为100\,GiB。该数据集仅包含数据块的元数据(例如,数据块指纹、大小、进入快照的顺序等)。
    \item \textbf{Linux}\cite{linux},其中包括产生自Linux源代码稳定版本(v2.6.11到v5.13之间)的84个快照。
    \item \textbf{CouchDB}\cite{couchdb},其中包括83个CouchDB的 docker镜像,包含通用、社区和企业三个发行版,且版本号介于v2.5.2和v6.6.2之间。
\end{enumerate}

\begin{table}
  \centering
  \small
  \begin{tabular}{cccc}
    \toprule
    {\bf 数据集} & {\bf 快照} & {\bf 去重前总数据量} & {\bf 重复数据删除系数} \\
    \midrule
    FSL & 795 & 56.2\,TiB & 140.4 \\
    MS & 143 & 14.4\,TiB & 6.0 \\
    Linux & 84 & 44.9\,GiB & 1.3 \\
    CouchDB & 83 & 22.9\,GiB & 1.5 \\
    \bottomrule
  \end{tabular}
  \caption{真实世界数据集的特征(重复数据删除率定义为重复数据删除前数据大小与重复数据删除后数据大小之比,更高的重复数据删除率意味着相应的数据集包含更多的冗余)}
  \label{tab:featurespy-datasets}
\end{table}

\subsection{检测效果分析}
\label{subsec:featurespy-evaluation-detection}

\paragraph*{Exp\#1(明文数据块的相似性检测)。}本文首先验证通过比较内容特征可有效地找到相似的明文数据块。本文考虑多个SYNChunk$(x, y)$数据集,每个SYNChunk$(x, y)$ 包括10\,K个从基本数据块修改得到的相似数据块($x$定义了随机修改位置的数量,$y$定义了每个修改位置处连续修改的字节数,参见\S\ref{subsec:featurespy-datasets})。 本文为SYNChunk$(x, y)$中的每个明文数据块提取四个内容特征,如果这些数据块与基本数据块分别共享一到四个相同的特征,则认为成功识别相似数据块。本文评估每个SYNChunk$(x, y)$数据集中成功识别的与基本数据块相似的数据块占整个数据集所有数据块的比例。

\begin{figure}[!htb]
    \centering
    \includegraphics[width=0.5\textwidth]{pic/featurespy/plot/detection/syn/fixed_pq_legend.pdf}
    \vspace{5pt}\\
    \begin{tabular}{@{\ }c@{\ }c}
        \includegraphics[width=0.45\textwidth]{pic/featurespy/plot/detection/syn/fixed_p_4.pdf} &
        \includegraphics[width=0.45\textwidth]{pic/featurespy/plot/detection/syn/fixed_q_16.pdf}  \\
        \mbox{\small (a) 固定$x=4$,改变$y$}                                                    &
        \mbox{\small (b) 固定$y=16$,改变$x$}                                                     \\
    \end{tabular}
    \caption{(Exp\#1) 明文数据块的相似性检测}
    \label{fig:featurespy-expDetectionSynSim}
\end{figure}

图~\ref{fig:featurespy-expDetectionSynSim}(a)显示了当本文固定$x$=4并变化$y$(图~\ref{fig:featurespy-expDetectionSynSim}(a)),以及固定$y$=16并变化的$x$(图~\ref{fig:featurespy-expDetectionSynSim}(b))时产生的多个SYNChunk$(x, y)$数据集的检测结果。一般来说,识别到相似数据块的成功率随着$x$或$y$的增大而降低,这是因为修改范围越大,越有可能使数据块产生不同的内容特征。具体来说,成功率受$x$的影响比$y$更大,这是因为增加$x$将改变大量滑动窗口(\S\ref{subsec:featurespy-basic})的Rabin指纹。另一方面,本文可以通过检查数据块是否共享任意一个或两个相同的特征来有效地检测大部分(例如,至少78.9\%)的相似数据块。

\paragraph*{Exp\#2(密文数据块的相似性检测)。}
本文扩展Exp\#1来研究\sysnameF 基于密文数据块的相似性检测能力。实验采用本文提出的相似性保留加密方法(SPE)产生密文数据块。具体来说,本文对每个明文数据块执行特征密钥生成(\S\ref{subsec:featurespy-spe}),检查为每个SYNChunk$(x, y)$中所有数据块生成的密钥相同的比例,以验证相似性保留加密为相似数据块产生相同特征密钥的能力(标记为KeyGen)。然后,本文执行SPE,使用特征密钥加密对应数据块的特征指标,并统计加密后的相似性指标与每个数据集中基本数据块的相似性指标相同的比例作为密文数据块相似性检查能力的结果。这里,本文关注三个SPE实例SPE$(1)$、SPE$(2)$和SPE$(4)$,它们将1个、2个和4个加密块(长度分别为16、32和64个字节)作为每个数据块的相似性指标。

\begin{figure*}[!htb]
    \centering
    \includegraphics[width=0.4\textwidth]{pic/featurespy/plot/detection/syn/synBarPlotDetect_legend.pdf}\\
    \begin{tabular}{@{}c@{}c}
        \includegraphics[width=0.49\textwidth]{pic/featurespy/plot/detection/syn/syn-p1-q4-detect.pdf}  &
        \includegraphics[width=0.49\textwidth]{pic/featurespy/plot/detection/syn/syn-p2-q8-detect.pdf}    \\
        \mbox{\small (a) $\textrm{SYNChunk}(1, 4)$}                                                     &
        \mbox{\small (b) $\textrm{SYNChunk}(2, 8)$}                                                       \\
        \includegraphics[width=0.49\textwidth]{pic/featurespy/plot/detection/syn/syn-p4-q16-detect.pdf} &
        \includegraphics[width=0.49\textwidth]{pic/featurespy/plot/detection/syn/syn-p8-q32-detect.pdf}   \\
        \mbox{\small (c) $\textrm{SYNChunk}(4, 16)$}                                                    &
        \mbox{\small (d) $\textrm{SYNChunk}(8, 32)$}                                                      \\
    \end{tabular}
    \caption{(Exp\#2) 密文数据块的相似性检测}
    \label{fig:featurespy-expDetectionSynDetect}
\end{figure*}

图~\ref{fig:featurespy-expDetectionSynDetect}展示了具有最小(SYNChunk$(1,4)$)、小(SYNChunk$(2,8)$)、中(SYNChunk $(4,16)$)和大(SYNChunk$(8,32)$)差异的相似数据块产生的密文数据块的相似性检查成功率。与\textit{allFeature}相比,实例\textit{firstFeature}和\textit{minFeature}为仅为更相似的数据块生成相同的特征密钥,尤其是当修改量很大时。例如,\textit{firstFeature}和\textit{minFeature}分别为45.7\%和59.3\%的数据块产生相同密钥,但\textit{allFeature}仅为8.1\%的数据块生成了相同特征密钥。此外,\textit{minFeature}为差异较大的相似数据块产生相同特征密钥的能力优于\textit{firstFeature},这可能是因为最小内容特征对数据块内容的随机变化更不敏感。此外,本文观察到相似性保留加密在加密后保留了较高的相似性。具体来说,通过检查密文数据块中的相似性指标,本文在SYNChunk$(1,4)$、SYNChunk$(2,8)$、SYNChunk$(4,16)$和SYNChunk$(8,32)$,分别检测到至多95.2\%、90.4\%、80.2\%和58.3\%的相似数据块。

\paragraph*{Exp\#3(推测内容攻击检测案例研究)。}
本文扩展了\S\ref{sec:featurespy-attack}中的案例研究,以研究 \sysnameF 如何检测推断工资和签约奖金的推测内容攻击。本文用一个固定报告阈值$T$=3\%和一个大小为两个加密块(等效32字节)的相似性指标来配置 \sysnameF。参考\S\ref{sec:featurespy-attack},攻击者需要伪造101$\times$31=3131个文件,其中年薪和签约奖金分别有101和31个可能值。为了模拟在正常数据集中混合伪造文件(否则更容易被检测到)的攻击者,本文将所有伪造文件随机插入到每个Linux/CouchDB(由于FSL和MS数据集不含实际数据,因此不予考虑)快照中的文件之间,并对每个快照中的文件(及伪造的文件)单独执行数据分块\cite{fsl, meyer2011deduplication},并在产生的数据块之上应用相似性保留加密。本文以检测率(\sysnameF 成功检测到的此类攻击快照的数量与使用的快照总数的比率)作为评价标准。此外,本文使用\sysnameF 处理每个原始快照中的所有文件(不包含伪造的文件),评估\sysnameF 的误判率(即\sysnameF 误判的快照数量与原始快照总数的比率)。本文展示了\sysnameF 如何检测推测内容攻击,并给出整体的检测结果。

\begin{figure*}[!htb]
    \centering
    \includegraphics[width=0.8\textwidth]{pic/featurespy/plot/detection/overall/prefixDistribution_legend.pdf}\\
    \begin{tabular}{cc}
        \includegraphics[width=0.472\textwidth]{pic/featurespy/plot/detection/overall/prefixDistribution-1000-Linux-first.pdf} &
        \includegraphics[width=0.472\textwidth]{pic/featurespy/plot/detection/overall/prefixDistribution-1000-CouchDB-first.pdf} \\
        \mbox{\makecell[c]{\small (a) Linux:\textit{firstFeature}实例}}                                                        &
        \mbox{\makecell[c]{\small (b) CouchDB:\textit{firstFeature}实例}}                                                        \\        \includegraphics[width=0.472\textwidth]{pic/featurespy/plot/detection/overall/prefixDistribution-1000-Linux-min.pdf} &
        \includegraphics[width=0.472\textwidth]{pic/featurespy/plot/detection/overall/prefixDistribution-1000-CouchDB-min.pdf}   \\
        \mbox{\makecell[c]{\small (c) Linux:\textit{minFeature}实例}}                                                          &
        \mbox{\makecell[c]{\small (d) CouchDB:\textit{minFeature}实例}}                                                          \\
        \includegraphics[width=0.472\textwidth]{pic/featurespy/plot/detection/overall/prefixDistribution-1000-Linux-all.pdf}   &
        \includegraphics[width=0.472\textwidth]{pic/featurespy/plot/detection/overall/prefixDistribution-1000-CouchDB-all.pdf}   \\
        \mbox{\makecell[c]{\small (e) Linux:\textit{allFeature}实例}}                                                          &
        \mbox{\makecell[c]{\small (f) CouchDB:\textit{allFeature}实例}}                                                          \\
    \end{tabular}
    \caption{(Exp\#3)攻击检测示例:给出了使用\textit{firstFeature}、\textit{minFeature}和\textit{allFeature}方案分别处理包含伪造文件(即攻击,Attack)和不包含伪造文件(即不发生攻击,Raw)的Linux/CouchDB最后一个版本的快照时每个窗口(大小固定为$W$=1\,K)中最高频相似性指标的出现频率。}
    \label{fig:featurespy-expDetectionOverall}
\end{figure*}

图~\ref{fig:featurespy-expDetectionOverall}显示了\sysnameF 的三种实例在检测原始Linux快照(v5.13)和CouchDB快照(v6.6.2)以及插入伪造文件后的快照时的结果。具体来说,x轴为处理窗口(大小固定为$W$=1\,K)的顺序编号,y轴为每个窗口中具有相同相似性指标的数据块占该窗口中所有数据块的比例。本文观察到所有三个实例都可以有效地检测到推测内容攻击,因为它们在攻击快照(Attack)中至少有一个窗口的最高频率超过了阈值$T$=3\%(即,具有相同相似性指标的密文数据块的比例大于$T$=3\%)。另一方面,\textit{minFeature}方案在Linux快照中发生了误判,这是由于其至少有一个正常的窗口(例如,Raw的第65个窗口)中,具有相同相似性指标的密文数据块的比例大于阈值$T$=3\%。

\begin{figure}[!htb]
    \centering
    \includegraphics[width=0.5\textwidth]{pic/featurespy/plot/detection/overall/effectiveness-falsePositive_legend.pdf}
    \vspace{5pt}\\
    \begin{tabular}{@{\ }c@{\ }c}
        \includegraphics[width=0.6\textwidth]{pic/featurespy/plot/detection/overall/effectivenessLinux.pdf} &
        \includegraphics[width=0.3\textwidth]{pic/featurespy/plot/detection/overall/falsePositiveLinux.pdf}   \\
        \mbox{\small (a) 检测率}                                                                            &
        \mbox{\small (b) 误判率}                                                                              \\
    \end{tabular}
    \caption{(Exp\#3) 每个Linux/CouchDB快照中的总体检测率和误判率}
    \label{fig:featurespy-expDetectionOverallFalsePositive}
\end{figure}

图~\ref{fig:featurespy-expDetectionOverallFalsePositive}显示了本文将\sysnameF 的检测窗口大小分别配置为1\,K,5\,K和10\,K时Linux数据集中的总体检测率和误判率。在各个窗口大小设置下,\sysnameF 均实现了很高的检测率(例如,至少98.6\%)。然而,较大的窗口会略微降低检测率,这是因为大窗口中仅有相对较小部分的密文数据块具有相同的相似性指标。此外,\sysnameF 仅在窗口大小为1\,K时才会产生误判(图~\ref{fig:featurespy-expDetectionOverallFalsePositive}(b)),这是因为原始快照(Raw)当中本身包含较多的相似数据块。但即使在这种情况下,\textit{allFeature}也没有任何误判,这是因为它只能检测到少量相似度较高的数据块。另一方面,可检测到更多相似数据块(Exp\#2)的\textit{minFeature}方案会产生28.6\%的误判。

除了Linux数据集,本文也在CouchDB数据集中评估\sysnameF,并发现所有\sysnameF 的实例均可成功检测到所有混合在CouchDB数据集的快照中的推测内容攻击(即具有100\%的检测率),且没有引入任何误判。可能的原因是每个CouchDB快照中几乎不含有相似数据块,当注入许多对伪造文件产生的相似数据块时,\sysnameF 可以立即检测到(相似数据块的)频率分布变化。

%TODO
\paragraph*{Exp\#4(目标文件熵对攻击检测的影响)。}

本文扩展研究\sysnameF 在检测具有特定信息熵的目标文件时的检测率,即针对具有不同信息熵的目标文件SYNFile$(x, y)$($x$是文件中未知变量的数量,$y$是每个变量的可能取值的数量,参见\S\ref{subsec:featurespy-datasets})的检测能力。本文枚举了目标文件的$x\times y$个可能值,将它们随机插入到Linux数据集最后一个快照(v5.13,大小为985.9\,MiB)包含的文件序列中,并评估100次随机生成的目标文件数据集下\sysnameF 的检测率(阈值设置为固定的$T$=3\%)。

\begin{figure}[!htb]
    \centering
    \includegraphics[width=0.5\textwidth]{pic/featurespy/plot/detection/trade-off/trade_off_legend.pdf}
    \vspace{5pt} \\
    \begin{tabular}{@{\ }c@{\ }c@{\ }c}
        \includegraphics[width=0.32\textwidth]{pic/featurespy/plot/detection/trade-off/varyWindow_linux.pdf}    &
        \includegraphics[width=0.32\textwidth]{pic/featurespy/plot/detection/trade-off/varyModifyPos_linux.pdf} &
        \includegraphics[width=0.32\textwidth]{pic/featurespy/plot/detection/trade-off/varyFileNumber_linux.pdf}  \\
        \makecell[c]{\small (a) 改变窗口大小$W$                                                                   \\ \small 检测SYNFile$(6,2048)$数据集} &
        \makecell[c]{\small (b) 固定窗口大小$W$=5\,K                                                              \\ \small 检测SYNFile$(\cdot,2048)$数据集} &
        \makecell[c]{\small (c) 固定窗口大小$W$=5\,K                                                              \\ \small 检测SYNFile$(6,\cdot)$数据集} \\
    \end{tabular}
    \caption{(Exp\#4)针对不同伪造文件集合的检测率}
    \label{fig:featurespy-expDetectionTradeOff}
\end{figure}

图~\ref{fig:featurespy-expDetectionTradeOff}(a)展示了当改变窗口大小$W$时检测到对SYNFile$(6,2048)$目标文件的攻击的检测率结果。\textit{firstFeature}和\textit{minFeature}的检测率基本不受到窗口大小$W$的影响(例如,始终保持在93\%以上),但\textit{allFeature}的检测率随着窗口大小增大而下降到34\%。其原因是\textit{allFeature}只能检测到少量相似度较高的数据块(Exp\#2),当窗口大小$W$很大时,这些数据块在窗口内的占比很低。图~\ref{fig:featurespy-expDetectionTradeOff}(b)展示了改变目标文件中未知变量$x$的数量时的检测结果。本文观察到,当$x$=9时,\textit{allFeature}的检测率再次急剧下降到21\%,这是因为它无法检测到具有更多差异区域的相似数据块。

图~\ref{fig:featurespy-expDetectionTradeOff}(c)展示了当改变目标文件中每个变量的可能值的数量$y$时的结果。当$y$为512时,所有三个实例的检测率都较低(例如,仅为13\%)。其原因是目标文件的信息熵较低,攻击者只需要构建少量相似的内容,导致\sysnameF 难以在混在的数据块流中找到相似数据块。一种可能的解决方案是配置一个较小的阈值$T$或窗口大小$W$以在检测到少数几个相似数据块时报告攻击行为,但这会增加在不同工作负载中误判的可能性。对此,本文提出了一项未来的研究方向,即如何自动平衡检测对低信息熵文件的攻击和在处理不同工作负载时最大限度地减少误判之间的权衡。
\subsection{Performance Evaluation}
\label{sub:evaluation-performance}


\paragraph{Setup.} We configure two testbeds for performance evaluation. %to evaluate the performance of \prototype.
\begin{itemize}[leftmargin=*]
\item {\bf LAN.} It includes three machines, each of which is equipped with an eight-core 2.9\,GHz Intel Core i7-10700 CPU, a 4\,TB 7200 RPM Seagate Exos SATA hard disk drive and 32\,GB RAM.  All machines are connected via a 10\,Gbps switch, and run Ubuntu 20.04.3.

\item {\bf Cloud.} It is deployed in {\em Alibaba Cloud} \cite{alibaba}. We rent several {\em ecs.g7t.3xlarge} virtual machines (VMs) in two regions to run the cloud, key server and multiple clients, respectively. The VMs in the same and across different regions are connected with 10\,Gbps and 100\,Mbps networks, respectively.
  Each VM is equipped with a 12-core 3.5\,GHz CPU virtualized by an Intel Xeon (Ice Lake) Platinum 8369B and 48\,GiB memory, and installed with  Alibaba Cloud Linux 3.2104. We additionally mount the cloud machine with {\em Alibaba General-Purpose NAS} as the storage backend. The NAS can achieve up to 15K\,IOPS for 4\,K random reads and writes.
  %with 150\,MBps bandwidth.
\end{itemize}
  % Besides the LAN testbed, we also configured a Cloud testbed to evaluate the multiple-client performance in under both synthetic and real-world workloads. We select which supports Intel SGX in two different regions.  All machines run. The highest network bandwidth between instances in the same region is 10Gbps, while the network bandwidth between instances in the two regions is 100 Mbps. And, we use

We make the following default configuration. For the underlying \sysnameF, we fix $W$ = 5\,K (i.e., about 300\,KiB EPC usage), $T$ = 3\%, the size of the similarity indicator at 32\,bytes (i.e., two blocks), and the number of features extracted via N-transform as three. For \prototype, we configure three threads to extract the content features of plaintext chunks (except Exp\#5 that microbenchmarks the performance of \prototype in a single thread, and Exp\#6 that evaluates the performance of \prototype using a varying number of threads to extract content features), and a 1\,GiB container cache to improve the download performance (\S\ref{sec:implementation}).

\subsubsection{Synthetic Workloads}
\label{subsub:syn}

We evaluate \prototype using SYNUnique, in which each 2\,GiB file includes globally unique chunks (\S\ref{sub:datasets}).
To avoid disk I/O, we load all data into each client's memory before each test, and let the cloud store all received data in memory (as opposed to Exp\#8 that enables disk I/O to study real-cloud deployment).
Our goal is to understand the maximum achievable performance of \prototype without the impact of deduplication and disk I/O, and show that \prototype incurs limited performance overhead over SGXDedup \cite{ren21}.
We average the results of each experiment over 10 runs, and include the 95\% confidence intervals from {\em Student's t-Distribution} into bar charts (for brevity, we exclude them from line charts).

\paragraph{Exp\#5 (Microbenchmarks).}
We start with microbenchmark evaluation by deploying a client, a key server, and a cloud in distinct machines in the LAN testbed. We let the client upload the same 2\,GiB file in SYNUnique twice, and evaluate the processing time of different upload steps in a single thread. The considered steps include: (i) {\em chunking}, which partitions the input file into variable-size plaintext chunks; (ii) {\em feature generation}, which extracts the content features of each plaintext chunk and samples the target feature(s) based on \sysnameF instances; (iii) {\em fingerprinting}, which computes the fingerprint of each plaintext chunk; (iv) {\em key generation}, which generates both feature key and MLE key; (v) {\em encryption}, which encrypts each plaintext chunk; (vi) {\em detection}, which detects the learning-content attack based on ciphertext chunks; (vii) {PoW}, which proves the ownership of each ciphertext chunk; (viii) {\em deduplication}, in which the cloud detects duplicate chunks; (ix) {\em transfer}, which transmits non-duplicate ciphertext chunks and the file recipe.

% We present the time breakdown of \sysnameF to study the performance of different steps. (i) {\em chunking}, which divides the input file into non-overlapping plaintext chunks; (ii) {\em feature extraction} extracts the corresponding minFeature/firstFeature/allFeature from each chunk for feature-based key generation; (iii) {\em fingerprinting}, calculates the fingerprint of each chunk for MLE key generation; (iv) {\em key generation}, uses the key server to generate MLE key and feature-based key based on MLE hash and features; (v) {\em encryption}, use feature-based key to encrypt the indicator, and MLE key to encrypt the remaining content of each chunk; (vi) {\em detection}, the enclave launch attack detection based on the indicators; (vii) {\em SGX-based PoW}, which prove the ownership of the ciphertext chunk to the cloud; (viii) {\em deduplication}, in which the cloud detect duplicate chunks and inform the client; (ix) {\em transfer}, which uploads the non-duplicate ciphertext chunks and the recipe.

\begin{table}[t]
    \small
    \centering
    \setlength{\tabcolsep}{5pt}
    \renewcommand{\arraystretch}{1.05}
    \setlength{\tabcolsep}{0.006\textwidth}{
    \begin{tabular}{|c|c|c|c|c|}
        \hline
        \multicolumn{2}{|@{\,}c|}{\textbf{Procedure/Step}} & \multicolumn{1}{l|}{\hspace{.5em}\textbf{firstFeature}} &
        \multicolumn{1}{c|}{\textbf{minFeature}} &
        \multicolumn{1}{c|}{\textbf{allFeature}} \\ \hline \hline
        \multicolumn{2}{|c|}{Chunking} & \multicolumn{3}{c|}{$2.12\pm0.006$} \\ \hline
        \multicolumn{2}{|c|}{\makecell[c]{Feature generation}} &
        \makecell[c]{$4.34 \pm 0.01$} & \makecell[c]{$9.93 \pm0.04$} & \makecell[c]{$9.85 \pm0.02$} \\ \hline
        \multicolumn{2}{|c|}{\makecell[c]{Fingerprinting}}&
        \multicolumn{3}{c|}{$1.81 \pm 0.002$} \\ \hline        \multicolumn{2}{|c|}{\makecell[c]{Key generation}}&
        \multicolumn{3}{c|}{$0.73 \pm 0.02$ ($0.49 \pm 0.01$)} \\ \hline
        \multicolumn{2}{|c|}{Encryption} & \multicolumn{3}{c|}{$1.22 \pm 0.001$} \\ \hline
        \multirow{2}{*}{In Enclave}
        & Detection &
        \multicolumn{3}{c|}{$0.04   \pm 0.005$} \\ \cline{2-5}
        & PoW &
        \multicolumn{3}{c|}{$1.86   \pm 0.004$} \\ \hline
        \multicolumn{2}{|c|}{Deduplication}  &
        \multicolumn{3}{c|}{$0.55 \pm 0.02$}  \\ \hline
        \multicolumn{2}{|c|}{Transfer}  & \multicolumn{3}{c|}{$1.16 \pm 0.03$ ($0.04 \pm 0.001$)}\\ \hline
    \end{tabular}
    }
    \caption{(Exp\#5) Time breakdown per 1\,MiB of synthetic file data processed (unit: ms). Except explicitly specified in parentheses, the consumed time of each step in the second upload is identical with that in the first upload.}
    \label{tab:evaluation-syn-system-breakdown}
    \vspace{-6pt}
\end{table}

Table~\ref{tab:evaluation-syn-system-breakdown} presents the results (per 1\,MiB file data processed). Since \prototype builds on SGXDedup \cite{ren21}, it inherits the performance benefits to offload the computational overhead of key generation (\S\ref{sec:implementation}) in the second upload, as well as avoid the transfer of duplicate chunks. The detection step is efficient, and takes up to 0.4\% of the overall time in the upload procedure. In addition, the feature generation step is expensive due to the computational overhead of N-transform. For example, {\tt firstFeature} takes 31.4\% of the overall time in the first upload; the consumed time fractions further increase to  51.1\% and 50.9\% for
{\tt minFeature} and {\tt allFeature}, respectively, since they extract all three features (as opposed to {\tt firstFeature} that only extracts the first feature). However, we argue that we can mitigate the performance overhead of feature generation via multi-threading (see below).




% Here we perform two consecutive uploads of the same synthetic file (2GiB random file) to verify the difference in system performance when uploading non-duplicate and duplicate data. Since \sysnameF is developed based on {\em SGXDedup}, we have retained all the features in SGXDedup. Therefore, in the second upload, the key generation step's time computation decreased by $49\%$ (this is due to the speculative encryption in SGXDedup could off-line generate encryption/decryption mask for future key generation requests). At the same time, since the second upload does not require any ciphertext chunks to be uploaded, the time cost of uploading only the relevant metadata is only $3.45\%$ of the first upload.

% The detection step is very efficient, which only brings the overhead of $2\%$ to the enclave. However, feature extraction requires a lot of time (up to $51.13\%$ and $50.93\%$ of total time consumption with minFeature and allFeature, $31.38\%$ with firstFeature in the first upload), which will become the performance bottleneck of \sysnameF. Since the minFeature and allFeature schemes need to calculate a total of 3 features and then extract the target feature, but the firstFeature scheme only needs to calculate one feature, so it only needs $43.7\%$ of time to be completed. In the \sysnameF system, we use multiple threads to perform feature extraction calculations in parallel to reduce the overall system overhead.

\paragraph{Exp\#6 (Single-client performance).}
We consider a single client, and compare the performance of \prototype with the base system SGXDedup. We let the client upload the same file twice (like Exp\#5) and further download the file.
%We evaluate the upload (download) speed as the ratio of the file size (i.e., 2\,GiB) to the time the client finishes the upload (download).

Figure~\ref{fig:singleClientThroughput}(a) and Figure~\ref{fig:singleClientThroughput}(b) present the upload speeds of the first and second uploads, respectively, when we vary the number of threads to extract features (\S\ref{sec:implementation}). The speed of SGXDedup keeps as 297.1\,MiB/s in the first upload, and 304.3\,MiB/s in the second upload, since it does not extract features. The upload speed of \prototype first increases with the number of threads (e.g., 265.3\,MiB/s for {\tt firstFeature}, 261.3\,MiB/s for {\tt minFeature} and 262.6\,MiB/s for {\tt allFeature} when three threads are used to extract features in the first upload), and then decreases (e.g., at about 220\,MiB/s in the first upload and 225\,MiB/s in the second upload for all three instances) due to resource contention. By exploiting multi-threading, \prototype only incurs the performance overhead over SGXDedup at 8.0-12.0\% in the first upload, and 6.6-7.5\% in the second upload.
In addition, we  observe few performance differences between the first and second (that does not need to transfer duplicate data) uploads, since our LAN testbed has high  bandwidth for transferring data. We argue that source-based deduplication brings significant performance gains in real-cloud deployment (Exp\#8), and reduces the network traffic for processing real storage workloads (Exp\#10).
Figure~\ref{fig:singleClientThroughput}(c) compares the download speed. \prototype incurs 1.3\% performance drop, since it decrypts each chunk with both the MLE key and the feature key.


% We consider a single client, and compare the upload and download performance of \prototype with the baseline system: SGXDedup, which adopts SGX-based server-aided MLE generation and SGX-based PoW to support source-based deduplication with high security and performance. Note that, we build \prototype based on the baseline SGXDedup system. Therefore, we retained all the features of SGXDedup, especially speculative encryption which could boost the key generation performance when the system is not used for the first time. We evaluate the upload and download speeds in three cases: (i) a client first uploads a 2\, GiB file;  (ii) the client restarts and then uploads the identical 2\, GiB file again; and (iii) the client downloads the file. We first analyze the effect of multi-threaded feature extraction optimization through the first round of upload, then analyze the optimization effect of SGXDedup's design on duplicate data upload through the second round of upload, and finally analyze the impact of encrypting indicators separately on download performance. To analysis the theoretical performance of \prototype, we conduct the performance test in the ramdisk and run the test 10 times to give the confidence interval in student-t distribution.

\begin{figure}[t]
    \centering
    \includegraphics[width=0.35\textwidth]{pic/featurespy/plot/performance/LANSyn/legend.pdf}\\
    \vspace{1pt}
    \begin{tabular}{@{\ }c@{\ }c@{\ }c}
        \includegraphics[height=.78in]{pic/featurespy/plot/performance/LANSyn/upload_thread_line.pdf}&
        \includegraphics[height=.78in]{pic/featurespy/plot/performance/LANSyn/upload_thread_2nd_line.pdf}&
        \includegraphics[height=.78in]{pic/featurespy/plot/performance/LANSyn/download_bar.pdf}\\
        \makecell[c]{\small (a) 1st upload} &
        \makecell[c]{\small (b) 2nd upload} &
        \makecell[c]{\small (c) Download}\\
    \end{tabular}
    \vspace{-6pt}
    \caption{(Exp\#6) Single-client performance in the LAN testbed. In download, all \prototype instances achieve the same speed, and we compare them (orange) with SGXDedup (green).}
    \vspace{-6pt}
    \label{fig:singleClientThroughput}
\end{figure}


\paragraph{Exp\#7 (Multi-client performance).}
We evaluate the performance when multiple clients issue uploads/downloads concurrently. We use the cloud testbed to consider an increasing number of clients, and deploy all clients, the key server and the cloud in the same region. We measure the {\em aggregate upload (download) speed} as the ratio of the total uploaded (downloaded) data size to the total time all clients finish the uploads (downloads).


\begin{figure*}[t]
    % \hspace{-5pt}
    \begin{minipage}[t]{0.58\textwidth}
        \centering
        \includegraphics[width=0.7\linewidth]{pic/featurespy/plot/performance/multiClient/legend.pdf}\\
        \vspace{1pt}
        \begin{tabular}{@{\ }c@{\ }c@{\ }c}
            \includegraphics[width=0.32\linewidth]{pic/featurespy/plot/performance/multiClient/upload_1st_line.pdf}&
            \includegraphics[width=0.32\linewidth]{pic/featurespy/plot/performance/multiClient/upload_2nd_line.pdf}&
            \includegraphics[width=0.32\linewidth]{pic/featurespy/plot/performance/multiClient/download_line.pdf}\\
            \makecell[c]{\small (a) 1st upload} &
            \makecell[c]{\small (b) 2nd upload} &
            \makecell[c]{\small (c) Download}\\
        \end{tabular}
        \vspace{-11pt}
        \captionof{figure}{(Exp\#8) Multi-client performance. The download speeds of all \prototype instances are identical, and we compare it (orange) with SGXDedup (green).}
        \vspace{-5pt}
        \label{fig:expMultiClientThroughput}
      \end{minipage}
          \hspace{6pt}
          \begin{minipage}[t]{0.4\textwidth}
               \vspace{-6pt}
               \centering

               \setlength{\tabcolsep}{0.01\textwidth}{
                 \small
          \begin{tabular}{|c|c|c|c|}
                \hline
                {\bf Approach} & {\bf 1st Upload} & {\bf 2nd Upload} & {\bf Download} \\
                \hline
                \hline
                Transfer & \multicolumn{3}{c|}{11.8 $\pm$ 0.04} \\
                \hline
                \hline
                \makecell[c]{\tt firstFeature} & \multirow{3}{*}{11.5 $\pm$ 0.006} & 204.4 $\pm$ 10.06 & \multirow{3}{*}{11.5 $\pm$ 0.004} \\
                \cline{1-1}\cline{3-3}
                \makecell[c]{\tt minFeature} &  & 184.7$\pm$ 7.4 &  \\
                \cline{1-1}\cline{3-3}
                \makecell[c]{\tt allFeature} &  & 185.0$\pm$ 6.4 &  \\
                \hline
                SGXDedup & 11.5 $\pm$ 0.009 & 233.2 $\pm$ 8.4 & 11.5 $\pm$ 0.004 \\
                \hline
            \end{tabular}
        }
        \hspace{15pt}
        \vspace{2pt}
        \captionof{table}{(Exp\#7) Real-cloud deployment (unit: MiB/s). We use {\tt scp} to upload a 2\,GiB file from the client to the cloud to provide a transfer benchmark in the environment.}
        \label{tab:expCloudTest}
      \end{minipage}
\end{figure*}

Figure~\ref{fig:expMultiClientThroughput} presents the results with up to 10 clients. We do not consider more clients, since the aggregate upload speed has  generally become stable. Specifically, the speeds of both the first and the second uploads  increase with the number of clients, and then gradually decrease due to the write contention at the cloud. We observe that SGXDedup reaches the peak performance (e.g., at 924.9\,MiB/s for four clients in the first upload) earlier than \prototype (e.g., up to 882.2\,MiB/s for six clients in the first upload), since it achieves high performance and the cloud is easy to be saturated. Similarly, the download speeds of all approaches decrease beyond three (for SGXDedup that achieves the peak speed of 1148.7\,MiB/s) or four (for \prototype that achieves the peak speed of 1141.3\,MiB/s) clients due to the read contention in the cloud.



% for at most 12 clients. Here, we present: (a) the aggregated first-round upload speed of the \prototype's three schemes and SGXDedup; (b) the second round aggregated upload speed of uploading duplicate data; (c) the aggregated download speed of SGXDedup and \prototype. We found that similar to Exp\#7, in the case of only one client, in the first round of upload, firstFeature, minFeature, and allFeature based \prototype brought $26.9\%$, $27.9\%$ and $27.3\%$ overheads to SGXDedup, respectively. While in the second round, they brought $25.1\%$, $25.6\%$, and $25.7\%$ overhead compared with SGXDedup, which is slightly lower than the overhead in the first round upload. For firstFeature-based scheme, the key generation is the bottleneck of \prototype, while SGXDedup only needs half of the key generation cost which will not become a bottleneck (See Exp\#11 for detailed Intranet deployment breakdown). Note that we set the key enclave to optimize second-round key generation for 12 clients. So, only about $\sim20\%$ of the data of each client can be accelerated by the SGX-based speculative encryption, which can only slightly reduce the overhead, but key generation is still the bottleneck of firstFeature-based \prototype. At the same time, minFefature and allFeature based \prototype's performance is severely restricted by feature generation.


% We consider multiple clients for upload and download operations. Same as Exp\#6, we test its peak performance in ramdisk. To collect the results of more clients, we deployed the key server and the cloud and all clients in the same region in the Alibaba Cloud and used the intranet for testing, which bandwidth is 10Gbps. We start all clients at the same time and count the time required for the last completed client to calculate the total system throughput of multiple clients.

% \begin{figure*}[t]
%     \centering
%     \includegraphics[width=0.4\textwidth]{pic/featurespy/plot/performance/multiClient/legend.pdf}\\
%     \vspace{1pt}
%     \begin{tabular}{@{\ }c@{\ }c@{\ }c}
%         \includegraphics[width=0.33\textwidth]{pic/featurespy/plot/performance/multiClient/upload_1st_line.pdf}&
%         \includegraphics[width=0.33\textwidth]{pic/featurespy/plot/performance/multiClient/upload_2nd_line.pdf}&
%         \includegraphics[width=0.33\textwidth]{pic/featurespy/plot/performance/multiClient/download_line.pdf}\\
%         \makecell[c]{(a) 1st round upload} &
%         \makecell[c]{(b) 2nd round upload} &
%         \makecell[c]{(c) Download}\\
%     \end{tabular}
%     \vspace{-5pt}
%     \caption{(Exp\#8) Multi-client uploads and downloads. (a)Compares the aggregated first-round upload speed of the \prototype's three schemes and SGXDedup with different number of clients. (b) Compares the aggregated speed of uploading duplicate data in the second round with different number of clients. (c) Compares the aggregated download speed of SGXDedup and \prototype}
%     \vspace{-5pt}
%     \label{fig:expMultiClientThroughput}
% \end{figure*}


% With the increase in the number of clients, the aggregated throughput of the two rounds of uploads both rose rapidly and then slightly declined. In the first round of upload, the peak of SGXDedup appeared at 4 clients ($924.9$\,MiB/s), and the peak of \prototype appeared when there are 6 clients (average $875.6$\,MiB/s for three schemes), this is because the client performance of \prototype is lower than SGXDedup, and the server write contention aggravate late. In the second round, the peak of SGXDedup appeared when there were 6 clients ($1644.4$\,MiB/s), and \prototype's maximum speed appeared with 9 clients (average $1581.7$\,MiB/s for three schemes). The number of clients at peak performance has increased. It is because the second round does not need to upload duplicate data, which reduced the cloud's write contention, and the system throughput is only limited by deduplication query. For downloading, the \prototype reached peak performance ($1141.3$\,MiB/s) with 4 clients, and SGXDedup reached peak performance ($1148.7$\,MiB/s) with 3 clients. And the download throughput of SGXDedup under each client number is slightly higher than \prototype, this is because the client pressure of \prototype is slightly higher than that of SGXDedup during downloading. After reaching the peak performance, both SGXDedup and \prototype is affected by cloud read contention and decline quickly.




\paragraph{Exp\#8 (Real-cloud deployment).}
We extend Exp\#6 to study the single-client performance in the cloud testbed. We deploy the client and the key server in two VMs in the same region, and a cloud in a different region (as opposed to Exp\#7, in which all entities are deployed in the same region), so as to simulate the scenario that Internet is connected between  the client and the cloud. Also, we let the client read the file from the local disk (i.e., SSD that achieves about 270\,MiB/s for reads/writes) for uploads, and the cloud store received data in the attached NAS. We use {\tt scp} to benchmark the transfer speed between the client and the cloud.



%read the file from the disk and upload it (for comparison, Exp\#6 load the file into the ramdisk and then upload it), and let the cloud store the received data in the NAS. We also use {\em scp} to upload the same 2\,GiB file as the benchmark of the Internet environment.

% based on Exp\#6, we deploy \prototype and SGXDedup in a real cloud environment. To avoid the impact of fluctuations in the local network environment, we deploy cloud at region A and deploy client and key server at region B to simulate real data outsourcing scenarios. The network bandwidth between the two Cloud (\S\ref{sub:evaluation-performance}) region is 100 Mbps.

% \begin{table}[t]
%     \small
%     \centering
%     \renewcommand{\arraystretch}{1.05}
%     \begin{tabular}{|c|c|c|c|}
%     \hline
%     {\bf Approach} & {\bf First Upload} & {\bf Second Upload} & {\bf Download} \\
%     \hline
%     \hline
%     Transfer & \multicolumn{2}{c|}{11.84 $\pm$ 0.04} & \multicolumn{1}{c|}{11.56 $\pm$ 0.19}  \\
%     \hline
%     \hline
%     \makecell[c]{firstFeature} & \multirow{3}{*}{11.46 $\pm$ 0.006} & 204.44 $\pm$ 10.057 & \multirow{3}{*}{11.45 $\pm$ 0.004} \\
%     \cline{1-1}\cline{3-3}
%     \makecell[c]{minFeature} &  & 184.71$\pm$ 7.415 &  \\
%     \cline{1-1}\cline{3-3}
%     \makecell[c]{allFeature} &  & 184.96$\pm$ 6.387 &  \\
%     \hline
%     SGXDedup & 11.51 $\pm$ 0.009 & 233.22 $\pm$ 8.358 & 11.54 $\pm$ 0.004 \\
%     \hline
%     \end{tabular}
%     % \vspace{-1pt}
%     \caption{(Exp\#7) Real-cloud upload and download (unit: MiB/s).}
%     \label{tab:expCloudTest}
%     \vspace{-6pt}
% \end{table}

Table~\ref{tab:expCloudTest} shows the results. In the first upload, the performance of all approaches is bounded by the transfer speed. In the second upload, the performance of SGXDedup and {\tt firstFeature} is bounded by chunking (Table~\ref{tab:evaluation-syn-system-breakdown}), while {\tt firstFeature} incurs 12.3\% performance overhead over SGXDedup, since it additionally extracts the first feature for key generation. The performance of both {\tt minFeature} and {\tt allFeature} (in the second upload) is bounded by features extraction. Note that the second upload speeds of all approaches are slower than those (Exp\#6) in the LAN testbed for three reasons. First, we now process on-disk files and enable disk I/O. Also, the VMs in the cloud testbed are virtualized from physical machines, and may incur performance drops when handling computational intensive operations.
Furthermore, the high latency of Internet slows down the  transferring of fingerprints in deduplication.
In the download, the performance of all approaches is bounded by the transfer speed, and \prototype incurs 0.6\% overhead when compared with SGXDedup.


% In the first round of upload, limited by the Internet bandwidth, the speeds of \prototype and SGXDedup are $11.46\pm 0.006$, and $11.51\pm 0.009$\,MiB/s, respectively. In the second round of uploading, \prototype (firstFeature) reached 204.44 $\pm$ 10.057\,MiB/s, while minFeature and allFeature are slightly slower than firstFeature scheme with 184.71$\pm$ 7.415\,MiB/s and 184.96$\pm$ 6.387\,MiB/s, respectively. For comparison, SGXDedup reached 233.22 $\pm$ 8.358\,MiB/s. Here, the overhead of \prototype firstFeature scheme compared with SGXDedup has reached $12.34\%$, which is higher than $6.48\%$ in Exp\#6. This is because the deduplication query (bound with the detection and SGX-based PoW) becomes the bottleneck of both SGXDedup and firstFeature-based \prototype, but firstFeature-based \prototype has additional data exchange overhead with multi-threaded feature generation. Meanwhile, the performance of minFeature and allFeature schemes is limited by feature generation (Refer to the breakdown for Internet deployment in Exp\#11), and the overhead compared with SGXDedup has reached $20.8\%$ and $20.7\%$ respectively. In download, \prototype and SGXDedup are restricted again by the Internet bandwidth, and the speeds are $11.47\pm 0.01$\,MiB/s and $11.54\pm 0.01$\,MiB/s respectively. Note that \prototype's download performance is slightly lower than SGXDedup by $0.6\%$, this is because the recipe in \prototype contains the feature-based key, which requires more processing time before download chunks.'



\subsubsection{Real-world Workloads}
\label{subsub:real}
We evaluate \prototype using FSL and MS, in order to understand its performance when processing real-world large-scale data.

\paragraph{Exp\#9 (Trace-driven performance).}
We evaluate the upload and download performance in the LAN testbed. We choose ten snapshots from FSL and MS each as follows. For FSL, we pick the weekly snapshots from the same user to have high cross-snapshot redundancies; For MS, we pick the snapshots that have the most intra-snapshot redundancies. The chosen FSL and MS snapshots take 407.5\,GiB and 902.5\,GiB of pre-deduplicated data, respectively. Since our snapshots only contain chunk fingerprints and sizes (\S\ref{sub:datasets}), we reconstruct each plaintext chunk by repeatedly writing its fingerprint into a spare chunk with the corresponding specified size. We first upload the snapshots one by one, and then download them in the same order of upload. Note that the original SGXDedup \cite{ren21} does not have the container cache (\S\ref{sec:implementation}); for fair comparison, we implement an in-memory cache for SGXDedup to buffer the most recently restored containers, and configure it with the same size (1\,GiB) as that of \prototype.


\begin{figure}[t]
    \centering
    \includegraphics[height=0.15in]{pic/featurespy/plot/performance/LANTrace/trace_legend_upload.pdf}\\
    \includegraphics[height=0.15in]{pic/featurespy/plot/performance/LANTrace/trace_legend_download.pdf}\\
    \vspace{3pt}
    \begin{tabular}{@{\ }c@{\ }c}
        \includegraphics[height=.82in]{pic/featurespy/plot/performance/LANTrace/trace_fsl.pdf}&
        \includegraphics[height=.82in]{pic/featurespy/plot/performance/LANTrace/trace_ms.pdf}\\
        \mbox{\small (a) FSL} &
        \mbox{\small (b) MS}\\
        % \includegraphics[height=.82in]{pic/featurespy/plot/performance/LANTrace/trace_linux.pdf}&
        % \includegraphics[height=.82in]{pic/featurespy/plot/performance/LANTrace/trace_couch.pdf}\\
        % \mbox{\small (c) Linux} &
        % \mbox{\small (d) CouchDB}\\
    \end{tabular}
    \vspace{-6pt}
    \caption{(Exp\#9) Trace-driven performance.}
    \vspace{-6pt}
    \label{fig:traceDrivenThroughput}
\end{figure}

Figure~\ref{fig:traceDrivenThroughput} presents the results. After the first FSL snapshot (e.g., 224.8\,MiB/s, 223.9\,MiB/s, 214.9\,MiB/s and 216.9\,MiB/s for SGXDedup, {\tt firstFeature}, {\tt minFeature} and {\tt allFeature}, respectively), both SGXDedup and \prototype achieve high performance (e.g., at least 298.9\,MiB/s, 266.8\,MiB/s, 246.4\,MiB/s  and 248.8\,MiB/s for SGXDedup, {\tt firstFeature}, {\tt minFeature} and {\tt allFeature}, respectively), since they do not need to transfer the cross-snapshot redundancies that take a large fraction in FSL. The download speed is generally steady (e.g., 88.7-102.6\,MiB/s for SGXDedup,  and 88.0-100.2\,MiB/s for \prototype). On average, compared to SGXDedup, {\tt firstFeature}, {\tt minFeature} and {\tt allFeature} slow down the upload performance by 8.8\%, 15.7\% and 15.0\%, respectively, and the download performance by 0.8\%.

Compared to FSL, the upload  performance in MS generally drops by about 21\%, since MS includes many unique chunks and leads to a large fingerprint index (that is implemented via LevelDB \cite{leveldb}). This aggravates the overhead of querying the fingerprint index for the existence of ciphertext chunks for deduplication. Also, the download speed in MS fluctuates across snapshots, since some snapshots have more non-duplicate chunks and may be stored in the consecutive regions (i.e., less fragmented \cite{lillibridge13}) that can be quickly accessed via sequential reads.



% (a) shows the upload and download speeds across FSL snapshots. Both SGXDedup and \prototype can achieve high upload speeds. When uploading the first snapshots, SGXDedup achieves $224.8$\, MiB/s upload speed, while \prototype achieves average speed at $218.6$\, MiB/s, and then when uploading the remaining snapshots, due to a large number of duplicate chunks, the upload speeds of SGXDedup and \prototype are significantly improved. For \prototype and SGXDedup, the average performance of $257.9$\, MiB/s and $300.1$\, MiB/s has bEen reached respectively. On average, \prototype incurs an upload performance drop of $13.18\%$ compared to SGXDedup. Note that the overhead is slightly larger than that ($5.26\%\sim7.23\%$) in our synthetic performance evaluation (Exp\#6). The reason is that chunking is now disabled in trace-driven evaluation, the bottleneck of SGXDedup has changed from chunking to SGX-based PoW, and the upper-performance limit has been improved, but the bottleneck of \prototype is still the feature extraction.  The download speed decreases from $102.6$\,MiB/s to $88.7$\,MiB/s for SGXDedup, and from $100.2$\,MiB/s to $88.0$\,MiB/s for \prototype, mainly due to the decrease in cache hit rate caused by chunk fragmentation \cite{lillibridge13}.

% Figure~\ref{fig:traceDrivenThroughput}(b) shows the upload and download speeds across MS snapshots. Both systems achieve lower upload speeds than in the FSL snapshots since the MS dataset has larger data volume and increased the access overhead of the fingerprint index. On average, \prototype incurs upload overhead of $9.65\%$ compared to SGXDedup. Note that the download speed of the MS dataset fluctuates greatly, because the deduplication ratio between each snapshot of MS is low and the difference is obvious, the hit rate of our container cache fluctuates greatly.

\paragraph{Exp\#10 (Network traffic analysis).}
We analyze the network traffic of \prototype, and compare it with three deduplication approaches that are secure against the learning-content attack: (i) {\em target-based deduplication} \cite{harnik10}, which forces the client to transfer all ciphertext chunks to the cloud; (ii) {\em random-threshold deduplication} \cite{harnik10}, which performs target-based deduplication if the number of uploads of each chunk is smaller than a pre-defined threshold, or source-based deduplication (\S\ref{sub:basics}) otherwise; (iii) {\em two-stage deduplication} \cite{li15}, which performs source-based deduplication on the ciphertext chunks from the same client, followed by target-based deduplication on those across different clients. Here, we follow the previous work \cite{harnik10} to choose the upper and lower bounds of the threshold in randomized-threshold deduplication at 20 and 2, respectively. We focus on FSL and MS. For FSL, we merge the snapshots of each user on a daily basis, and store them in the order of time. For MS, we consider that each snapshot is from an individual client, and store the snapshots in the order of the snapshot ID.
We do not consider the bandwidth overhead due to file recipes.


% Here we only consider the network traffic generated by the fingerprint (32\,Byte) required for source-based deduplication and the chunks that need to be uploaded.

% We analyze the network traffic of \prototype during upload. We consider three schemes that can prevent learning-content attacks for comparison. They are all based on source-based deduplication (i.e., the client performs deduplication and only upload non-duplicate chunks to the cloud) and/or target-based deduplication (i.e., the client uploads all chunks to the cloud, which performs deduplication on the received chunks) schemes: (i) {\em two-stage deduplication} \cite{li15}, which applies source-based deduplication on each client, followed by target-based deduplication across clients; (ii) {\em randomized-threshold deduplication} \cite{harnik10}, which performs either source-based deduplication or target-based deduplication based on a randomly chosen threshold. Here, we  For FSL, as  For MS, Linux and CouchDB, we . Note that, CouchDB contains three different distributions, we upload the common version first, then the community version, and finally the enterprise version. For {\em two-stage deduplication}, we consider each different ID (in MS dataset) or version number (in Linux and CouchDB datasets) as an independent user.

\begin{figure}[t]
    \centering
    \includegraphics[height=0.2in]{pic/featurespy/plot/bandwidth/upload_traffic_legend.pdf}     \vspace{3pt} \\
    \begin{tabular}{@{\ }c@{\ }c}
        \includegraphics[width=0.23\textwidth]{pic/featurespy/plot/bandwidth/upload_traffic_fsl.pdf} &
        \includegraphics[width=0.23\textwidth]{pic/featurespy/plot/bandwidth/upload_traffic_ms.pdf} \\
        {\small (a) FSL} & {\small (b) MS} \\
        % \includegraphics[width=0.23\textwidth]{pic/featurespy/plot/bandwidth/upload_traffic_linux.pdf} &
        % \includegraphics[width=0.23\textwidth]{pic/featurespy/plot/bandwidth/upload_traffic_couch.pdf} \\
        % {\small (c) Linux} & {\small (d) CouchDB}
    \end{tabular}
    \vspace{-6pt}
    \caption{(Exp\#10) Cumulative network traffic after storing each snapshot.}
    \vspace{-6pt}
    \label{fig:expNetworkTraffic}
\end{figure}


Figure~\ref{fig:expNetworkTraffic} presents the results. Since \prototype performs pure source-based deduplication, it outperforms the other approaches. For example, after storing the last snapshot, \prototype reduces the network traffic of target-based deduplication by 98.9\% in FSL and 83.1\% in MS. Two-stage deduplication performs well in FSL (e.g., only 3.7\% bandwidth overhead over \prototype), since FSL includes a large volume of intra-user redundancies. However, in MS, the network traffic of two-stage deduplication is finally 3.1$\times$ of that of \prototype. Random-threshold deduplication incurs varying network traffic in FSL (e.g., 72.5$\times$ of \prototype) and MS (e.g., 2.6$\times$ of \prototype).


% reduces the network traffic of target-based deduplication, random-threshold deduplication and two-stage deduplication
% Compared to
% shows the accumulated network traffic according to the snapshot storage order. When snapshots are uploaded, compared with target-based deduplication, \prototype saved $98.92\%$, $83.13\%$, $74.53\%$, and $43.78\%$ network traffic for FSL, MS, Linux, and CouchDB dataset respectively. Two-stage deduplication achieves almost the same traffic savings in FSL as $98.88\%$, because the FSL dataset includes a large volume of intra-user redundancies. In contrast, in the MS datasets, it can only achieve network traffic savings of $47.81\%$ (compare with \prototype, reduced by $42.48\%$). Since there is almost no redundancy in each version of the Linux and CouchDB datasets, and it requires a large number of source-based deduplication queries, for Linux, it only brings $1.05\%$ traffic savings, and for CouchDB, it even causes $0.31\%$ extra traffic overhead. The traffic savings brought by the Randomized-threshold deduplication are quite different. In MS, it brought $55.75\%$ traffic savings, while in FSL and Linux, there is only $21.60\%$ and $26.68\%$. However, it even introduces the extra traffic overhead of $0.0016\%$ in CouchDB. In addition, we found that in the first 35 versions of CouchDB, the network traffic of several schemes is almost the same, and for the latter versions, \prototype's accumulative network traffic keeps almost level. This is because these 35 versions are common versions with almost no redundancy. And there is a lot of redundancy between these common versions and the community/enterprise versions, making \prototype no longer need to upload a lot of data.

\section{Related Work}
\label{sec:related-work}

\paragraph{Secure deduplication approaches.}
In \S\ref{sub:basics}, we reviewed two orthogonal threats against deduplication. MLE (and its variants \cite{bellare13a, bellare13b, douceur02, li15}) preserves data confidentiality against a compromised cloud, and is extensively studied from security theory \cite{bellare15, abadi13}, and evaluated in storage systems \cite{cox02, adya02, bellare13b, armknecht15, shah15, li15, li19, qin17, li20a, ren21}. \sysnameF proposes SPE (\S\ref{sub:spe}), which augments MLE with similarity preservation, so as  to  detect the learning-content attack based on ciphertext chunks.


PoW prevents a malicious  client from compromising data ownerships in source-based deduplication, but it cannot address the learning-content attack (\S\ref{sub:attack}). \sysnameF complements encrypted deduplication \cite{ren21} by proactively detecting the learning-content attack. Although previous studies can defeat against the learning-content attack by applying deduplication in different ways \cite{harnik10, li15}, they inject network traffic for transferring duplicate contents. \sysnameF performs  pure source-based deduplication and achieves bandwidth efficiency (Exp\#10).





\paragraph{SGX-based secure applications.}
SGX has been widely used to strengthen the security of different application systems, such as Bitcoin \cite{matetic19}, file systems \cite{ahmad18, shinde20}, outsourced databases \cite{eskandarian17, priebe18, sun21}, key-value stores \cite{mishra18, bailleu19, kim19, bailleu21}, and data analytic platforms \cite{schuster15, zheng17, bowe20}. This paper focuses on deduplicated storage systems. S2Dedup \cite{miranda21} manages a cloud-side enclave to perform secure deduplication, yet it relies on the assumption that the client is fully trusted. This paper extends SGXDedup \cite{ren21} to prevent a malicious client from launching the learning-content attack.

%SGXDedup \cite{ren21} implements SGX-based PoW to prevent a compromised client from , yet it is vulnerable to the learning-content attack.


% BITE , while protects the privacy of light client in Bitcoin.
% . PESOS implements policy-based object storage against an untrusted third-party commodity platform. EnclaveDB manages sensitive database states in a server-side enclave, and preserves confidentiality, integrity and freshness of data records.

\section{Conclusion}
\label{sec:conclusion}
This paper addresses the learning-content attack in encrypted deduplication. We present \sysnameF, which augments  encrypted deduplication by proactively detecting the learning-content attack in a client-side TEE. It
builds on the insight that a malicious client enumerates many similar data for attack, and reports an attack by detecting similar chunks among the processing contents. Also, it proposes SPE, and allows detection  to be directly performed on ciphertext chunks.
Evaluation shows that \sysnameF not only effectively detects the learning-content attack, but also incurs limited performance overhead when being deployed in SGXDedup \cite{ren21}.

\chapter{全文总结与展望}

\section{全文总结}

本文以可信执行环境在加密后重复数据删除中的应用为研究背景,主要对提升服务器辅助消息锁加密

本文基于可信执行环境(TEE)解决了加密后重复数据删除中的性能问题。本文提出了\sysnameS,它实现了一组安全区内部调用以在安全区内执行敏感操作,从而在保持安全性的同时加速加密重复数据删除。\sysnameS 包含三项关键设计:(1)安全高效的安全区管理;(2)自更新的会话密钥管理;以及(3)用于减轻在线加解密开销的基于TEE的推测性加密。此外,本文的实验分析表明\sysnameS 在合成数据集和真实世界数据集作为工作负载的情况下具有远超现有基于密码学机制的加密重复数据删除的性能表现。

本章解决了加密后重复数据删除中的推测内容攻击,提出了\sysnameF,它通过在客户端TEE中主动检测推测内容攻击来增强加密后重复数据删除。它建立在以下观察的基础之上:发起推测内容攻击的而已客户端往往会枚举大量相似的伪造文件进行攻击。因此,可通过检测较小时间段内处理数据中相似数据块出现频率来检测推测内容攻击。此外,本文提出了特征保留加密技术(SPE),并实现了基于密文的数据块相似性检查。实验分析表明\sysnameF 不仅可以有效的检测推测内容攻击,且在\sysnameS 中部署时仅产生有限的额外性能开销。


\section{后续工作展望}

高性能加密后重复数据删除及可信执行环境在存储系统中的应用等相关研究近几年发展迅速,在本文研究工作的基础上,仍有以下方向值得进一步研究:


\thesisacknowledgement
在攻读硕士学位期间,首先衷心感谢我的导师李经纬教授

\thesisbibliography[large]{reference}

\thesisappendix
\section*{实验6中的部分原始数据}
下表为FSL数据集中User004在基于分布的攻击中按文件类型区分的的实验数据。
\begin{table}[!htbp]
\centering
\caption{基于分布的攻击(安全隐患分析)对FSL数据集User4的实际攻击结果}
\label{tab:exp6-user4}
\begin{tabular}{@{}ccccccc@{}}
\toprule
 文件类型 & 正确块数 & 总块数   & 正确唯一块数 & 总唯一块数 & \tabincell{c}{正确块占比\\超10\%的文件} & 文件总数 \\ \midrule
doc  & 1      & 192     & 1       & 108    & 0               & 24   \\
docx & 0      & 30      & 0       & 22     & 0               & 14   \\
pptx & 355    & 19184   & 330     & 9484   & 0               & 38   \\
jpg  & 7      & 484     & 7       & 434    & 1               & 384  \\
png  & 0      & 1284    & 0       & 722    & 0               & 160  \\
img  & 0      & 4       & 0       & 4      & 0               & 4    \\
amr  & 0      & 10976   & 0       & 5496   & 0               & 16   \\
c    & 1896   & 11290   & 1896    & 9826   & 1579            & 8362 \\
h    & 70     & 816     & 70      & 770    & 69              & 724  \\
py   & 2      & 60      & 2       & 53     & 2               & 46   \\
js   & 0      & 24      & 0       & 17     & 0               & 10   \\
data & 0      & 108     & 0       & 86     & 0               & 64   \\
db   & 33     & 5300    & 27      & 2660   & 9               & 58   \\
json & 0      & 346     & 0       & 194    & 0               & 42   \\
po   & 0      & 2616    & 0       & 1333   & 0               & 50   \\
xml  & 151    & 888     & 151     & 841    & 145             & 794  \\
pb   & 0      & 160     & 0       & 159    & 0               & 158  \\
gz   & 4      & 24474   & 4       & 12274  & 4               & 74   \\
bz2  & 0      & 1028    & 0       & 570    & 0               & 112  \\ 
html & 1      & 112     & 1       & 103    & 1               & 94   \\
log  & 8      & 1486    & 8       & 1206   & 8               & 926  \\
sh   & 453    & 2614    & 453     & 2274   & 381             & 1934 \\
out  & 2122   & 84548   & 860     & 37431  & 49              & 2716 \\
txt  & 226    & 1268    & 226     & 1089   & 181             & 910  \\
vmdk & 307356 & 1657924 & 213180  & 640077 & 22              & 48   \\
\bottomrule
\end{tabular}
\end{table}


% \begin{thesistheaccomplish}
%     \section{学术论文}
%     \bibitem{SGXDedup} 第一作者. Accelerating Encrypted Deduplication via SGX[C]. Proc.of USENIX ATC, 2021, 957-971. \textbf{CCF-A}
%     \bibitem{TED} 第三作者. Balancing storage efficiency and data confidentiality with tunable encrypted deduplication[C]. Proc. of EuroSys, 2020, 1-15. \textbf{CCF-B}
%     \bibitem{Metadedup} 学生第一作者. Metadedup: Deduplicating metadata in encrypted deduplication via indirection[C]. Proc. of MSST, 2019, 269-281. \textbf{CCF-B}
%     \bibitem{TEDExt} 第三作者. Tunable Encrypted Deduplication with Attack Resilient Key Management[J]. ACM Transactions on Storage (TOS), 2022. \textbf{CCF-A}
%     \bibitem{MetadedupExt} 第三作者. Enabling Secure and Space Efficient Metadata Management in Encrypted Deduplication[J]. IEEE Transactions on Computers (TC), 2021. \textbf{CCF-A}
%     \bibitem{Attack} 第四作者. Revisiting Frequency Analysis against Encrypted Deduplication via Statistical Distribution[C]. Proc. of IEEE INFOCOM, 2021. \textbf{CCF-A}
%     \section{发明专利}
%     \bibitem{CN111338572B} 第三发明人. 一种可调节加密重复数据删除方法:CN111338572B[P]. 2021-09-14.
%     \bibitem{CN110109617B} 第三发明人. 一种加密重复数据删除系统中的高效元数据管理方法:CN110109617B[P]. 2020-05-12.
%     \bibitem{CN112947855A} 第二发明人. 一种基于硬件安全区的高效加密重复数据删除方法:CN112947855A[P]. 2021-06-11.
%     \bibitem{CN112152798A} 第三发明人. 基于加密数据去重的分布式密文共享密钥管理方法及系统:CN112152798A[P]. 2020-12-29.
% \end{thesistheaccomplish}


\begin{thesistheaccomplish}
    \section{学术论文}
    \bibitem{SGXDedup} \textbf{Ren, Yanjing} and Li, Jingwei and Yang, Zuoru and Lee, Patrick PC and Zhang, Xiaosong. Accelerating Encrypted Deduplication via SGX[C]. Proc.of USENIX ATC, 2021, 957-971. \textbf{CCF-A}
    \bibitem{TED} Li, Jingwei and Yang, Zuoru and \textbf{Ren, Yanjing} and Lee, Patrick PC and Zhang, Xiaosong. Balancing storage efficiency and data confidentiality with tunable encrypted deduplication[C]. Proc. of EuroSys, 2020, 1-15. \textbf{CCF-B}
    \bibitem{Metadedup} Li, Jingwei and Lee, Patrick PC and \textbf{Ren, Yanjing} and Zhang, Xiaosong. Metadedup: Deduplicating metadata in encrypted deduplication via indirection[C]. Proc. of MSST, 2019, 269-281. \textbf{CCF-B}
    \bibitem{TEDExt} Yang, Zuoru and Li, Jingwei and \textbf{Ren, Yanjing} and Lee, Patrick PC and Zhang, Xiaosong. Tunable Encrypted Deduplication with Attack Resilient Key Management[J]. ACM Transactions on Storage (TOS), 2022. \textbf{CCF-A}
    \bibitem{MetadedupExt} Li, Jingwei and Huang, Suyu and \textbf{Ren, Yanjing} and Yang, Zuoru and Lee, Patrick PC and Zhang, Xiao-song and Hao, Yao. Enabling Secure and Space Efficient Metadata Management in Encrypted Deduplication[J]. IEEE Transactions on Computers (TC), 2021. \textbf{CCF-A}
    \bibitem{Attack} Li, Jingwei and Wei, Guoli and Liang, Jiacheng and \textbf{Ren, Yanjing} and Lee, Patrick PC and Zhang, Xiaosong. Revisiting Frequency Analysis against Encrypted Deduplication via Statistical Distribution[C]. Proc. of IEEE INFOCOM, 2021. \textbf{CCF-A}
    \section{发明专利}
    \bibitem{CN111338572B} 李经纬, 杨祚儒, \textbf{任彦璟}, 李柏晴, 张小松. 一种可调节加密重复数据删除方法:CN111338572B[P]. 2021-09-14.
    \bibitem{CN110109617B} 李经纬, 李柏晴, \textbf{任彦璟}, 张小松. 一种加密重复数据删除系统中的高效元数据管理方法:CN110109617B[P]. 2020-05-12.
    \bibitem{CN112947855A} 李经纬, \textbf{任彦璟}, 杨祚儒, 李柏晴, 张小松. 一种基于硬件安全区的高效加密重复数据删除方法:CN112947855A[P]. 2021-06-11.
    \bibitem{CN112152798A} 李经纬, 黄苏豫, \textbf{任彦璟}, 杨祚儒, 李柏晴. 基于加密数据去重的分布式密文共享密钥管理方法及系统:CN112152798A[P]. 2020-12-29.
\end{thesistheaccomplish}

\end{document}