\subsection{安全区管理}
\label{subsec:sgxdedup-enclave-management}

\sysnameS 在首次初始化时通过云建立对所有安全区的信任。在部署 \sysnameS 之前,本文首先将安全区代码编译成共享对象 \cite{sgx},为每个共享对象附加签名(用于完整性验证),并将共享对象分发到密钥管理器和每个客户端。云还托管共享对象以供后续验证。密钥管理器创建密钥安全区,而每个客户端通过加载相应的共享对象来创建自己的 PoW 安全区。云通过远程证明 (\S\ref{subsec:sgxdedup-sgx}) 对每个安全区进行身份验证,以确保加载正确的代码。在这里,本文解决了两个特定的管理问题:(i)如何将全局机密(\S\ref{subsec:sgxdedup-arch})安全地引导到密钥安全区; (ii) 每个客户端在重启后如何有效地引导其 PoW 安全区。

\paragraph*{Key安全区management.} \sysnameS 不是完全引导全局密钥,而是根据云和密钥管理器分别拥有的两个 \textit{ sub-secrets} 在密钥安全区中生成全局密钥,以便阻止他们中的任何一个了解整个全球秘密。
为了生成全局密钥,本文将云的子密钥硬编码到密钥安全区代码中,并在 SGXDedup 初始化期间将代码(作为共享对象)传递给密钥管理器。本文还为密钥安全区实现了一个\textit{ secret generation ECall},以便让密钥管理器提供自己的子密钥。只有在云的子密钥被包含在密钥安全区中后,密钥管理器才能发出 ECall。它将密钥管理器的子密钥作为其单一输入,并对密钥管理器的子密钥和云的子密钥的串联进行哈希运算,形成全局密钥。请注意,密钥管理器无法访问安全区代码,因此无法了解在安全区内硬编码的云子秘密(假设逆向工程是不可能的)。因此,即使密钥管理器遭到破坏,全局机密仍然是安全的,因此服务器辅助 消息锁加密(MLE) 的安全性得以保留。如果密钥管理器和云同时受到威胁,\sysnameS 的安全性会降低到原始 消息锁加密(MLE) (\S\ref{subsec:sgxdedup-encrypted-dedup}) 的安全性。

\paragraph*{PoW安全区管理。} 当客户端启动其 PoW安全区时,它​​需要证明 PoW安全区的真实性。但是,远程证明通常会产生非常大的延迟(例如,大约 9\,s;请参阅 \S\ref{subsec:sgxdedup-synthetic})以连接到Intel服务。与 key安全区不同,其远程证明只在初始化期间完成一次,客户端每次加入和离开 \sysnameS 时都需要分别引导和终止 PoW enclave。如果每次客户端加入时都使用远程证明,其大量开销将损害可用性。

\sysnameS 在 PoW安全区的第一次引导后利用密封来避免远程证明。回想一下,PoW 安全区与云共享一个 PoW 密钥,这样云就可以验证指纹的真实性 (\S\ref{subsec:sgxdedup-arch})。本文的想法是根据 PoW安全区的测量哈希来密封 PoW 密钥。因此,当客户端再次引导其 PoW 安全区时,它会将 PoW 密钥解封到引导的 PoW 安全区中。只要成功恢复 PoW 密钥,就可以验证自举 PoW安全区的真实性。

具体来说,客户端首先检查其物理机中是否有任何密封的 PoW 密钥在本地可用。如果密封的 PoW 密钥不可用(第一个引导程序),客户端通过远程证明来证明 PoW 安全区并与云交换 PoW 密钥;否则,如果一个密封的 PoW 密钥可用(在第一次引导之后),客户端通过加载共享对象创建一个新的 PoW enclave,并调用新 PoW安全区的 \textit{ key unsealing ECall} 来解封 PoW 密钥。解封 ECall 的密钥以被密封的 PoW 密钥的地址作为输入。它根据新 PoW安全区的测量散列推导出密封密钥,解密密封的 PoW 密钥,并将其保存在新的 PoW安全区中。

当客户端离开 \sysnameS 时,它​​的 PoW安全区需要被终止。客户端发出 \textit{ key seal ECall} 来密封 PoW 密钥。密钥密封 ECall 根据 PoW安全区的测量哈希对 PoW 密钥进行加密,并将结果存储在客户端提供的地址中。