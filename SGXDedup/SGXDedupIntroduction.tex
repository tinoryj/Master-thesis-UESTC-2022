\section{简介}
\label{sec:sgxdedup-introduction}

现有的加密后重复数据删除方法通常会产生很高的性能开销来实现安全保证(参见\S\ref{subsubsec:intro-problem-performance})。可信执行环境\cite{trustzone,sgx,MK-TME,AMDSEV}的发展为提高加密后重复数据删除的性能提供了新的机会。本文通过在安全区内运行敏感操作来替代传统加密后重复数据删除中昂贵的密码学操作:即在客户端安全区执行所有权证明以取代传统基于Merkle树的所有权证明技术;在密钥服务器中的安全区执行密钥生成以取代传统基于OPRF协议的服务器辅助密钥生成技术。从而提高加密后重复数据删除的性能,同时保持其安全性、带宽效率和存储效率。

本文提出了\sysnameS,这是一个基于TEE的高性能加密后重复数据删除系统。\sysnameS 建立在DupLESS\cite{bellare2013DupLESS}中的服务器辅助密钥管理之上,但在安全区内执行高效的加密操作。实现\sysnameS 的设计面临着严峻的挑战。首先,安全的创建并启动安全区以托管受信任的代码和数据至关重要,但对安全区进行远程证明(参见\S\ref{subsec:background-tee-sgx})会导致显著的启动延迟;其次,每个客户端都需要通过安全信道与密钥服务器内部的安全区进行通信,但安全信道的管理开销会随着客户端数量的增加而增加;最后,客户端可能随时续订或撤销云存储服务的订阅,因此允许对客户端身份进行高效验证以遍动态管理至关重要。为此,本文为\sysnameS 设计了三项核心技术:

\begin{itemize}[leftmargin=0em]
    \item \textbf{安全高效的安全区管理}:
          它可以防止密钥服务器被攻击时泄露密钥生成过程中的隐私信息,并允许客户端在重新启动后快速启动安全区。
    \item \textbf{自更新会话密钥管理}:
          它基于密钥回归\textit{(Key Regression)}\cite{fu06}生成会话密钥,用于保护密钥服务器内的安全区和每个客户端之间的通信,并通过更新会话密钥的方式对客户端访问权限进行动态管理。
    \item \textbf{基于TEE的推测性加密}:
          它通过推测性加密\textit{(Speculative Encryption)}\cite{eduardo2019Speculative}减轻了密钥安全区管理其与各个客户端之间的安全信道传输会话内容时的在线数据加密/解密开销。
\end{itemize}

本文使用随机数据和真实世界数据集\cite{fsl,meyer2011deduplication}作为工作负载评估了本文提出的\sysnameS 原型。与DupLESS\cite{bellare2013DupLESS}中基于OPRF协议的服务器辅助密钥生成方案相比,密钥生成加速了131.9倍;与基于Merkle树的所有权证明方案\cite{halevi11}相比,数据块所有权证明效率提升了8.2倍。此外,相较于DupLESS\cite{bellare2013DupLESS},\sysnameS 对于上传非重复或重复数据实现了8.1倍和9.6倍的性能提升,并且在真实世界数据集作为工作负载的测试中节省了高达99.2\%的带宽与存储开销。

