\section{实验分析}
\label{sec:sgxdedup-evaluation}

本文为多个客户端、密钥服务器和云配置了一个 LAN 机器集群。每台机器都有一个四核 3.0\,GHz Intel Core i5-7400 CPU,一个 1\,TB 7200 RPM SATA 硬盘和 8\,GB RAM。所有机器都运行 Ubuntu 18.04 并通过 10\,GbE 连接。本文使用合成 (\S\ref{subsec:sgxdedup-synthetic}) 和真实世界 (\S\ref{subsec:sgxdedup-real-world}) 工作负载来评估 \sysnameS。本文将主要结果总结如下。

\begin{itemize}[leftmargin=*]
\item \sysnameS 在单客户端 (Exp\#1) 和多客户端 (Exp\#2) 情况下均实现了高 消息锁加密(MLE) 密钥生成性能。例如,在单客户端情况下,它比 {\em DupLESS} 的 消息锁加密(MLE) 密钥生成 \cite{bellare2013DupLESS} 采用的 OPRF-RSA 实现了 131.9$\times$ 的加速。
\item \sysnameS 与基于通用哈希的 PoW \cite{xu2013weak}(仅实现较弱的安全性)和基于 Merkle 树的 PoW \cite{halevi11} 相比,具有 2.2$\times$ 和 8.2$\times$ 的计算 PoW 加速 (Exp\#3)。
\item \sysnameS 在单客户端(Exp\#4 和 Exp\#5)和多客户端(Exp\#6)情况下具有较高的整体性能。本文还提供了上传中 \sysnameS 的时间细分 (Exp\#7)。例如,在 10\,GbE LAN 测试平台中,与没有任何安全保护的普通重复数据删除系统相比,\sysnameS 的上传速度仅降低了 17.5\%;在真实云部署(Exp\#5)中,\sysnameS 会导致 13.2\% 的减速,并且其性能受 Internet 带宽的限制。
\item 在上传时间细分中,\sysnameS 将第二次上传的客户端初始化时间减少了 91.5\%,并将第二次上传的密钥生成时间减少了 41.9\% (Exp\#7)。
\item \sysnameS 对于处理实际工作负载(Exp\#8 和 Exp\#9)非常有效。例如其上传性能开销相对于普通重复数据删除(无安全保护)在22.0\%以内;与现有方法 \cite{li15,harnik2010side} 相比,它还实现了高带宽节省,绝对差异高达 91.4\%。
\end{itemize}


\begin{table}
  \centering
  \small
    % \begin{tabular}{|@{\hspace{.1em}}c@{\hspace{.2em}}|c@{\hspace{.2em}}|c@{\hspace{.2em}}|@{\hspace{.2em}}c@{\hspace{.2em}}|}
  \begin{tabular}{cccc}
    \toprule
    \multicolumn{2}{c}{\bf 对比对象} & {\bf 基础方案/系统} & {\bf 优势} \\
    \midrule
    \multirow{8}{*}{\bf 性能提升} & \multirow{4}{*}{\shortstack{密钥生成}} & OPRF-BLS \cite{armknecht2015transparent} & 1,583$\times\;\uparrow$  \\
               &            & OPRF-RSA \cite{bellare2013DupLESS} & 131.9$\times\;\uparrow$ \\
               &            & MinHash encryption \cite{li2020Info} & 9.4$\times\;\uparrow$ \\
               &            & TED \cite{li2020TED} & 3.7$\times\;\uparrow$ \\
    \cline{2-4}
    & \multirow{2}{*}{所有权证明} & PoW-MT \cite{halevi11} & 8.2$\times\;\uparrow$ \\
      &                     & PoW-UH \cite{xu2013weak} & 2.2$\times\;\uparrow$ \\
    \cline{2-4}
    & \multirow{2}{*}{\shortstack{原型系统}} & {\em DupLESS} \cite{bellare2013DupLESS} & 8.1$\times\;\uparrow$ \\
    & & PlainDedup & 17.5\% $\downarrow$ \\
    \hline
    \multicolumn{2}{c}{\multirow{2}{*}{\shortstack{\bf 网络资源节省}}} & Two-stage dedup \cite{li15} & 35.3\% $\uparrow$ \\
\multicolumn{2}{c}{} & Randomized-threshold dedup \cite{harnik2010side} & 91.4\% $\uparrow$ \\
    \bottomrule
  \end{tabular}
    \vspace{-6pt}
    \caption{Summary of main results.}
  \label{tab:sgxdedup-summary}
\end{table}

