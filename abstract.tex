\begin{chineseabstract}
    云服务商通过重复数据删除来保持较低的存储维护成本,重复数据删除仅存储来自相同或不同客户端的冗余数据的单个副本。加密重复数据删除建立在密码原语的基础上,以增强重复数据删除的数据机密性,并同时保留存储空间节省能力,对云服务商具有较大吸引力。然而,现有的加密重复数据删除方法建立在昂贵的密码原语之上,这会导致性能大幅下降。同时,现有加密重复数据删除不能完全解决恶意客户端枚举数据内容以发起学习内容攻击的问题。

    针对加密重复数据删除中的性能问题,本文提出了\sysnameS,它利用可信执行环境来加速基于服务器辅助消息锁加密,以及基于数据所有权证明的源端重复数据删除,同时通过可信安全区保持加密重复数据删除的安全性。\sysnameS 通过一组安全接口在安全区内执行消息锁加密密钥生成和数据所有权证明操作,并提出了三项关键设计来支持高效且安全的安全区操作。对于随机和实际工作负载的评估表明\sysnameS 实现了显著的性能提升并保持了高带宽/存储节省。

    针对推测内容攻击的安全性问题,本文提出了\sysnameF,它通过在安全区内主动执行攻击检测来增强加密重复数据删除。它建立在恶意客户端枚举许多相似数据进行攻击的观察的基础上,并通过检测数据之间的相似性来发现推测内容攻击。此外,它还提出了一种新的加密原语来保留加密后的相似性,并根据加密的内容进行相似性检测,从而使恶意客户端无法绕过检测过程。对随机和实际工作负载的评估表明,\sysnameF 不仅可以检测高概率且低误判率的检测到推测内容攻击,而且部署在\sysnameS 中时仅产生有限的性能开销。

    \chinesekeyword{加密后重复数据删除,可信执行环境,密钥生成,所有权证明,推测内容攻击}
\end{chineseabstract}

\begin{englishabstract}
    Cloud providers keep low maintenance costs via deduplication, which stores only a single copy of redundant data from the same or different clients.
    Encrypted deduplication builds on cryptographic primitives to augment deduplication with data confidentiality, which preserves the deduplication effectiveness on encrypted data and is attractive for outsourced storage. However, existing encrypted deduplication approaches build on expensive cryptographic primitives that incur substantial performance slowdown. Meanwhile, existing encrypted deduplication has a security problem, as it cannot fully address a malicious client, which enumerates data contents to launch the learning-content attack.

    To address the performance overhead in encrypted deduplication. This paper presents \sysnameS, which leverages the trusted execution environment (TEE) to speed up encrypted deduplication based on server-aided message-locked encryption (MLE) while preserving security. In addition, it also uses TEE to boost proof-of-ownership (PoW) in source-based deduplication. \sysnameS implements a suite of secure interfaces to execute MLE key generation and PoW operations in enclaves. It also proposes assorted designs to support secure and efficient enclave operations. Evaluation on synthetic and real-world workloads shows that \sysnameS achieves significant speedups and maintains high bandwidth and storage savings.

    To address the security limitation, this paper presents \sysnameF, which augments encrypted deduplication by proactively performing attack detection. It builds on the insight that a malicious client enumerates many similar data for the attack, and reports the learning-content attack by detecting similarity among the processing contents. Also, it proposes a new primitive to preserve similarity after encryption and performs similarity detection based on the encrypted contents, such that the malicious client cannot bypass the detection procedure. Evaluation on synthetic and real-world workloads shows that \sysnameF not only detects the learning-content attack with high probabilities and low misjudgments, but also incurs limited performance overhead when being deployed in \sysnameS.

    \englishkeyword{Encrypted Deduplication, Trusted Execution Environment(TEE), Key Generation, Proof-of-Ownership, Learning-content Attack}
\end{englishabstract}