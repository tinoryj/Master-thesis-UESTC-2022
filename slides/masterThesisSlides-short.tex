\documentclass{beamer}
\usefonttheme[onlymath]{serif}
\usepackage{ctex, hyperref}
\usepackage[T1]{fontenc}

% other packages
\usepackage{latexsym,amsmath,xcolor,multicol,booktabs,calligra}
\usepackage{graphicx,pstricks,listings,stackengine}
\usepackage{multirow}

\author{匿名}
\title{基于可信执行环境的高性能加密重复数据删除研究}
\subtitle{硕士学位论文答辩}
\institute{电子科技大学计算机科学与工程学院(网络空间安全学院)}
\date{2022年5月30日}
\usepackage{UESTC}

% defs
\def\cmd#1{\texttt{\color{red}\footnotesize $\backslash$#1}}
\def\env#1{\texttt{\color{blue}\footnotesize #1}}
\definecolor{deepblue}{rgb}{0,0,0.5}
\definecolor{deepred}{rgb}{0.6,0,0}
\definecolor{deepgreen}{rgb}{0,0.5,0}
\definecolor{halfgray}{gray}{0.55}

\lstset{
    basicstyle=\ttfamily\small,
    keywordstyle=\bfseries\color{deepblue},
    emphstyle=\ttfamily\color{deepred},    % Custom highlighting style
    stringstyle=\color{deepgreen},
    numbers=left,
    numberstyle=\small\color{halfgray},
    rulesepcolor=\color{red!20!green!20!blue!20},
    frame=shadowbox,
}

\newcommand{\sysnameS}{TEEDedup }
\newcommand{\sysnameF}{FeatureSpy }
\newcommand{\prototype}{TEEDedup+ }

\begin{document}

\kaishu
\begin{frame}
    \titlepage
\end{frame}

% \begin{frame}
%     \tableofcontents[sectionstyle=show,subsectionstyle=show/shaded/hide,subsubsectionstyle=show/shaded/hide]
% \end{frame}


\section{研究背景}

% \begin{frame}{外包数据存储}
%     \begin{itemize}[<+-| alert@+>] % 当然,除了alert,手动在里面插 \pause 也行
%         \item Outsourcing data management to cloud is common in practice
%         \item 22% business data are stored in the cloud[*]
%         \item  Outsourcing storage should fulfill security and storage efficiency
%         \item  Security: protect outsourced data against unauthorized access
%         \item  Storage efficiency: reduce storage footprints
%     \end{itemize}
% \end{frame}

\begin{frame}{外包数据存储}
    \begin{itemize}
        \item 外包数据存储管理的数据规模快速增长
              \begin{itemize}
                  \item 根据IDC发布于2020年的预测,2025年时单年新创建数据量将超过175.8\,ZB,且其中超过70\%的数据将集中存储于云端
              \end{itemize}
    \end{itemize}
    \vspace{-1em}
    \begin{figure}[!htb]
        \centering
        \includegraphics[width=0.8\linewidth]{./pic/dataStore.png}
        \vspace{-5pt}
        \caption{数据产生与存储关系的统计与预测}
        \label{fig:dataCreateAndStore}
    \end{figure}
    \vspace{-1.5em}
    \begin{itemize}
        \item  外包数据存储需同时满足安全性和存储经济性需求
              \begin{itemize}
                  \item 安全性:保护外包数据免受未授权访问
                  \item 存储经济性:尽可能减少存储设备需求
              \end{itemize}
    \end{itemize}
\end{frame}

\begin{frame}{重复数据删除}
    \begin{figure}[!htb]
        \small
        \centering
        \includegraphics[width=\linewidth]{../pic/background/chunk-based-dedup-arch.pdf}
        \caption{基于数据块的重复数据删除的工作流程概览}
        \label{fig:chunk-based-dedup-flow}
    \end{figure}
    \vspace{-1em}
    \begin{itemize}
        \item 重复数据删除是一种粗粒度的数据压缩技术
              \begin{itemize}
                  \item 作用单位为数据块(可变大小或固定大小)
              \end{itemize}
        \item  其仅保存具有相同内容的数据块的一个副本,并将其他数据块通过索引指向该副本
    \end{itemize}
\end{frame}

\begin{frame}{加密重复数据删除}
    \begin{figure}[!htb]
        \small
        \centering
        \includegraphics[width=\textwidth]{../pic/background/chunk-based-enc-dedup-arch.pdf}
        \caption{基于数据块的加密重复数据删除的工作流程概览}
        \label{fig:chunk-based-enc-dedup-flow}
    \end{figure}
    \vspace{-1em}
    \begin{itemize}
        \item 加密重复数据删除基于密文数据块执行重复数据删除
              \begin{itemize}
                  \item 传统加密方案无法兼容跨用户的重复数据删除
                  \item 消息锁加密基于明文数据块导出加密密钥(例如数据块哈希)
              \end{itemize}
    \end{itemize}
\end{frame}

\begin{frame}{加密重复数据删除的安全性保障}
    \begin{figure}[!htb]
        \small
        \centering
        \includegraphics[width=\textwidth]{../pic/background/Cloud-encrypted-deduplication-logic.pdf}
        \caption{面向云环境的加密重复数据删除系统框架}
        \label{fig:Cloud-based-encrypted-deduplication-storage-logic}
    \end{figure}
    \vspace{-1em}
    \begin{itemize}
        \item 依赖基于盲签名的服务器辅助密钥生成抵御离线暴力破解攻击
        \item 依赖数据所有权证明保护源端重复数据删除(即仅传输非重复数据块)
              \begin{itemize}
                  \item 目标端重复数据删除需传输所有数据块,产生高额网络资源开销
              \end{itemize}
    \end{itemize}
\end{frame}

\begin{frame}{性能瓶颈问题}
    \begin{table}[!htb]
        \small
        \centering
        \caption{基础原型每个步骤处理1MiB数据的时间开销:括号内为可供备选的服务器辅助密钥生成方案和所有权证明方案,以及相应处理时间}
        \label{tab:intro-bottleneck}
        \begin{tabular}{@{}ccc@{}}
            \toprule
                                        & 处理阶段/可选方案                    & 处理时间(ms/MiB)          \\ \midrule
                                        & 数据分块                             & 3.8                       \\
            服务器辅助密钥生成          & {\color{red} RSA(BLS)盲签名方案}     & {\color{red} 40.7(488.3)} \\
            \multirow{3}{*}{所有权证明} & 数据加密                             & 2.5                       \\
                                        & {\color{red} Merkel树(通用哈希)方案} & {\color{red} 27.0(7.2)}   \\
                                        & 重复数据删除                         & 1.7                       \\ \bottomrule
        \end{tabular}
    \end{table}
    \begin{itemize}
        \item 盲签名(OPRF)算法成为加密重复数据删除的系统性能瓶颈
        \item 基于Merkel树的所有权证明方案安全性高但计算开销极大
        \item 基于通用哈希的方案以降低安全性为代价且仍为性能瓶颈之一
    \end{itemize}
\end{frame}

\begin{frame}{安全性问题—推测内容攻击}
    \begin{itemize}
        \item 通过枚举可能的目标内容进行重复数据删除查询,基于重复数据删除结果判断攻击成功与否
    \end{itemize}
    \begin{figure}[!htb]
        \small
        \centering
        \includegraphics[width=\textwidth]{./pic/LRI.pdf}
        \caption{推测内容攻击模式}
        \label{fig:LRI-mode}
    \end{figure}
    \vspace{-1em}
    \begin{itemize}
        \item 现有防御机制存在高额网络资源开销或安全性不足等问题
              \begin{itemize}
                  \item 目标端重复数据删除(Target-based deduplication)
                  \item 随机阈值重复数据删除(Randomized-threshold deduplication)
                  \item 两阶段重复数据删除(Two-stage deduplication)
                  \item 随机冗余数据块传输方案(Randomized Redundant Chunk Scheme)
              \end{itemize}
    \end{itemize}
\end{frame}

\begin{frame}{可信执行环境}
    \begin{itemize}
        \item 可信执行环境(TEE)是一种在计算平台上由软硬件协同构建的安全区域,可确保其中的程序按照预期执行,保证程序初始状态和运行时的机密性、完整性
              \begin{itemize}
                  \item {\bf 隔离性}:
                        TEE将其中的目标程序、数据等与外部环境隔离,使得TEE外软硬件均无法获得其内部的机密信息。
                  \item {\bf 软硬协同性}:
                        TEE产品包含物理隔离的专用硬件或运行权限以及相应的软件SDK。
                  \item {\bf 富表达性}:
                        与传统安全芯片或纯软件的密码学隐私保护机制相比,TEE支持更丰富的上层应用。
              \end{itemize}
        \item  主流可信执行环境
              %   \vspace{-1em}
              \begin{itemize}
                  \item  Intel SGX:拥有隔离、密封、远程认证能力,已被广泛运用于安全虚拟机等场景
                  \item  ARM TrustZone:通过CPU状态切换实现隔离,并支持专用安全外设,已被广泛运用于移动设备生物识别、安全存储等场景
              \end{itemize}
    \end{itemize}
\end{frame}

\section{研究内容}

\begin{frame}{研究内容概述}
    \begin{itemize}
        \item 针对现有面向云环境的加密重复数据删除存在密钥生成效率低、所有权证明计算复杂、易于遭受推测内容攻击等不足,提出:
              \begin{itemize}
                  \item 基于TEE的轻量级密钥生成技术,相较基于RSA盲签名的方案实现{\color{red}131倍}性能提升
                  \item 基于TEE的高效所有权证明技术,相较基于Merkel树的方案实现{\color{red}8.2倍}性能提升
                  \item 基于TEE的扩展所有权证明技术,在测试数据集中实现{\color{red}98\%}以上的检测成功率
              \end{itemize}
    \end{itemize}
    \begin{itemize}
        \item 设计并分别实现基于Intel SGX和ARM TrustZone的\prototype 原型系统:
              \begin{itemize}
                  \item 相较于基于密码学方案的DupLESS实现{\color{red}8.1$\sim$9.6倍}的性能提升
                  \item 相较于不采用任何安全措施的明文重复数据删除系统仅产生{\color{red}17.5\%}的额外开销
              \end{itemize}
    \end{itemize}
\end{frame}

\subsection{\sysnameS: 基于TEE加速加密重复数据删除}

\begin{frame}{\sysnameS: 基于TEE加速加密重复数据删除}
    \begin{figure}[!htb]
        \centering
        \includegraphics[width=\linewidth]{../pic/sgxdedup/sgxdedup-arch.pdf}
        \caption{\sysnameS 系统框架,在密钥服务器和每个客户端中分别部署了密钥安全区和所有权证明安全区}
        \label{fig:sgxdedup-overview}
    \end{figure}
    \vspace{-1em}
    \begin{itemize}
        \item  {\bf 核心思想:使用TEE取代加密重复数据删除中计算开销高昂的密码学算法}
    \end{itemize}
\end{frame}

% \begin{frame}{\sysnameS:设计目标与关键问题}
%     \begin{itemize}
%         \item 设计目标:
%               \begin{itemize}
%                   \item 数据机密性保障:即使密钥服务器或任何客户端受到攻击,\sysnameS 仍可避免外包数据块和密钥受到未授权访问。
%                   \item 低带宽开销/高存储效率:与源端重复数据删除一致,\sysnameS 在上传前将进行跨客户端的重复数据删除,并仅上传非重复的密文数据块到云服务提供商。
%                   \item 高计算效率: 相较于现有基于软件的加密重复数据删除方案,\sysnameS 可显著降低数据块加密、所有权证明等密码学操开销。
%               \end{itemize}
%         \item 部署安全区存在的关键问题:
%               \begin{itemize}
%                   \item 如何安全有效的引导安全区?
%                   \item 密钥安全区与每个客户端如何建立安全信道?
%                   \item 密钥安全区如何减少安全信道的在线开销?
%               \end{itemize}
%     \end{itemize}
% \end{frame}

\begin{frame}{基于TEE的数据所有权证明}
    \begin{figure}[!htb]
        \centering
        \includegraphics[width=\textwidth]{../pic/sgxdedup/pow.pdf}
        \caption{\sysnameS 所有权证明安全区}
        \label{fig:sgxdedup-overview-pow}
    \end{figure}
    \vspace{-1em}
    \begin{itemize}
        \item  基于安全区与云服务端共享的PoW密钥对数据块进行指纹计算并签名,取代昂贵的Merkel树构建和验证
    \end{itemize}
\end{frame}

\begin{frame}{基于TEE的密钥生成(基于推测性加密)}
    \begin{figure}[!htb]
        \centering
        \includegraphics[width=\textwidth]{../pic/sgxdedup/key-enclave-arch.pdf}
        \caption{基于TEE的密钥生成(基于推测性加密)概述}
        \label{fig:sgxdedup-SpecEnc}
    \end{figure}
    \vspace{-1em}
    \begin{itemize}
        \item 基于密钥安全区和客户端共享的会话密钥保护密钥生成过程
        \item 推测性加密:基于AES CTR模式进行会话内容加解密,密钥安全区离线生成CTR模式所需的加解密掩码以提升在线服务效率
    \end{itemize}
\end{frame}

\subsection{\sysnameF: 基于TEE在线检测推测内容攻击}

\begin{frame}{\sysnameF: 基于TEE在线检测推测内容攻击}
    \begin{figure}[!htb]
        \centering
        \includegraphics[width=\textwidth]{../pic/featurespy/architecture.pdf}
        \caption{基于密文的攻击检测方案工作流程。\sysnameF 将攻击检测程序与安全区中的所有权证明(\sysnameS )结合起来,以防止恶意客户端绕过检测}
        \label{fig:featurespy-architecture-secure}
    \end{figure}
    \vspace{-1em}
    \begin{itemize}
        \item 检测原理:执行推测内容攻击的客户端需枚举伪造文件,产生大量相似数据块
        \item {\bf 关键问题:无法基于密文检测明文数据块相似性}
        \item {\bf 核心思想:在安全区中对采用相似性保留加密的数据块进行统计,以判断攻击是否发生}
    \end{itemize}
\end{frame}

\begin{frame}{相似性保留加密}
    \begin{figure}[!htb]
        \centering
        \includegraphics[width=0.7\textwidth]{../pic/featurespy/spe.pdf}
        \caption{相似性保留加密方法}
        \label{fig:featurespy-design-spe}
    \end{figure}
    \vspace{-1.5em}
    \begin{itemize}
        \item
              基于N-Transform方案提取多个数据块特征(默认为3个)
              \begin{itemize}
                  \item 采用所有特征的哈希作为数据块特征(allFeature方案)或从3个特征中选取代表特征(minFeature、firstFeature方案)
              \end{itemize}
        \item  密文数据块前缀(约占0.4\%)相同则认为两数据块相似
    \end{itemize}
\end{frame}

\section{原型系统实现}

\begin{frame}{原型系统实现}
    \begin{textbox}{基于Intel SGX}
        \begin{itemize}
            \item \prototype 及其底层\sysnameS 基于Intel SGX SDK 2.15以及OpenSSL、LevelDB等开发,包含16.5K行C++代码
            \item 开源链接:\href{https://github.com/tinoryj/masterGraduation/artifactsSGX}{https://github.com/tinoryj/masterGraduation/artifactsSGX}
        \end{itemize}
    \end{textbox}

    \begin{textbox}{基于ARM TrustZone}
        \begin{itemize}
            \item \prototype 及其底层\sysnameS 基于OP-TEE 3.16以及OpenSSL、LevelDB等开发,包含13.5K行C++代码
            \item 开源链接:\href{https://github.com/tinoryj/masterGraduation/artifactsARM}{https://github.com/tinoryj/masterGraduation/artifactsARM}
        \end{itemize}
    \end{textbox}
\end{frame}

\section{实验分析}

\subsection{测试环境}

\begin{frame}{测试环境与数据集}
    \begin{textbox}{测试数据集}
        \vspace{-1em}
        \begin{table}[!htb]
            \small
            \centering
            \caption{真实世界数据集的特征}
            \label{tab:featurespy-datasets}
            \begin{tabular}{cccc}
                \toprule
                {\bf 数据集} & {\bf 快照总数} & {\bf 去重前总数据量} & {\bf 重复数据删除系数} \\
                \midrule
                FSL          & 795            & 56.2\,TiB            & 140.4                  \\
                MS           & 143            & 14.4\,TiB            & 6.0                    \\
                Linux        & 84             & 44.9\,GiB            & 1.3                    \\
                CouchDB      & 83             & 22.9\,GiB            & 1.5                    \\
                \bottomrule
            \end{tabular}
        \end{table}
    \end{textbox}

    \begin{textbox}{测试平台}
        \begin{itemize}
            \item 本地集群(LAN)。万兆局域网内多台Intel SGX设备,每台设备均采用Intel Core i7-10700 CPU,4\,TB SATA机械硬盘,32\,GB DDR4内存
            \item 云环境(Cloud)。在两个不同区域的阿里云部署了多台规格为\textit{ecs.g7t.3xlarge}的虚拟机(VM)来分别运行云服务端、密钥服务器和多个客户端。
        \end{itemize}
    \end{textbox}
\end{frame}

\subsection{\sysnameS 原型系统}

\begin{frame}{\sysnameS 原型系统实验结果概览}
    \vspace{-1em}
    \begin{table}[!htb]
        \small
        \centering
        \caption{\sysnameS 原型系统主要实验结果汇总}
        \label{tab:sgxdedup-summary}
        \begin{tabular}{cccc}
            \toprule
            \multicolumn{2}{c}{\bf 对比对象}                              & {\bf 基础方案/系统}                                     & {\bf 优势}                                                 \\
            \midrule
            \multirow{9}{*}{\bf 性能提升}                                 & \multirow{4}{*}{\shortstack{密钥生成}}                  & OPRF-BLS                 & 1,583$\times\;\uparrow$         \\
                                                                          &                                                         & OPRF-RSA                 & 131.9$\times\;\uparrow$         \\
                                                                          &                                                         & MinHash encryption       & 9.4$\times\;\uparrow$           \\
                                                                          &                                                         & TED                      & 3.7$\times\;\uparrow$           \\
            \cline{2-4}
                                                                          & \multirow{2}{*}{所有权证明}                             & PoW-MT                   & 8.2$\times\;\uparrow$           \\
                                                                          &                                                         & PoW-UH                   & 2.2$\times\;\uparrow$           \\
            \cline{2-4}
                                                                          & \multirow{3}{*}{\shortstack{原型系统}}                  & DupLESS                  & 8.1$\times\;\uparrow$           \\
                                                                          &                                                         & PlainDedup               & 17.5\% $\downarrow$             \\
                                                                          &                                                         & {\color{blue}\prototype} & {\color{blue}11.9\% $\uparrow$} \\
            \hline
            \multirow{2}{*}{\shortstack{\bf 网络资源节省}}                & \multirow{2}{*}{\shortstack{重复数据删除}}              & Two-stage dedup          & 35.3\% $\uparrow$               \\
            \multicolumn{2}{c}{}                                          & Randomized-threshold dedup                              & 91.4\% $\uparrow$                                          \\
            \hline
            \multirow{2}{*}{\shortstack{\bf {\color{blue} 攻击检测效果}}} & \multirow{2}{*}{\shortstack{{\color{blue} \prototype}}} & {\color{blue}检测率}     & {\color{blue} >98.8\%}          \\
                                                                          &                                                         & {\color{blue}误报率}     & {\color{blue}<2.4\%}            \\
            \bottomrule
        \end{tabular}
    \end{table}
\end{frame}

\section{总结}

\begin{frame}{}
    \begin{textbox}{针对性能瓶颈问题:}
        \begin{itemize}
            \item \sysnameS :基于可信执行环境替代昂贵的密码学计算显著降低了加密重复数据删除中的性能瓶颈
                  \begin{itemize}
                      \item 安全高效的安全区管理机制
                      \item 自更新的会话密钥管理
                      \item 基于TEE的轻量级推测性加密
                  \end{itemize}
        \end{itemize}
    \end{textbox}
    \begin{textbox}{针对推测内容攻击问题:}
        \begin{itemize}
            \item \prototype :基于\sysnameS 以较低额外性能开销主动防御推测内容攻击
                  \begin{itemize}
                      \item 相似性保留加密
                      \item 安全区内基于密文的相似性检查
                  \end{itemize}
        \end{itemize}
    \end{textbox}
\end{frame}

\section{个人成果}

\begin{frame}[allowframebreaks]{攻读硕士学位期间取得的成果}
    \newcommand{\bstlabelmark}{lo}
    \nocite{*}
    \bibliographystyle{thesis-uestc}
    \bibliography{publication}
    % \begin{textbox}{其他成果及荣誉}
    %     \begin{itemize}
    %         \item 2022年四川省优秀毕业生
    %         \item 2021年中国电子科技集团-电子科技大学奖学金
    %         \item 2019$\sim$2021年研究生学业奖学金一等奖(三次)
    %         \item 2020年全国大学生集成电路创新创业大赛总决赛三等奖
    %     \end{itemize}
    % \end{textbox}
\end{frame}

\begin{frame}
    \begin{center}
        {\Huge 衷心感谢老师倾听}

        {\Huge 请各位老师指正!}
    \end{center}
\end{frame}

\end{document}