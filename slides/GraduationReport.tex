\documentclass[aspectratio=43]{beamer}
\usepackage[utf8]{inputenc}
\usepackage{multicol}
\usepackage{fontspec}
\setsansfont{Songti SC} % font name is case-sensitive

\title{密文重复数据删除机制的\\频率分析攻击}
\date{\today}
\author[Tinoryj]{任彦璟}

\usetheme{material}

\useLightTheme
\usePrimaryBlue
\useAccentOrange

\begin{document}

\begin{frame}
\titlepage
\end{frame}

\begin{frame}
\begin{card}
现有频率分析攻击在加密重复数据删除中效果不佳,\\[2mm]如何有针对性的提高攻击效果?
\end{card}
\end{frame}


\begin{frame}{目录}
\begin{card}
\tableofcontents
\end{card}
\end{frame}

\section{对选题的分析}

\begin{frame}{对选题的分析}
\begin{card}
\begin{itemize}
    \item 背景知识
    \item 两大核心问题
    \item 研究的意义
    \item 研究的目的
\end{itemize}
\end{card}
\end{frame}

\begin{frame}{背景知识-重复数据删除}
\centering
\cardImg{img/DedupSystemStorageMode}{\textwidth}
\begin{card}
重复数据删除系统的存储方式
\end{card}   
\end{frame}

\begin{frame}{背景知识-重复数据删除}
\centering
\cardImg{img/DedupSystemView}{\textwidth}
\begin{card}
重复数据删除系统的运作流程
\end{card}   
\end{frame}

\begin{frame}{背景知识-密文重复数据删除}
\centering
\cardImg{img/EncryptDedupSystemLogic}{\textwidth}
\begin{card}
密文重复数据删除的运作流程
\end{card}   
\end{frame}

\begin{frame}{背景知识-频率分析攻击}
\begin{card}
频率分析是一种针对确定性加密的密码分析技术,被应用\\[2mm]于破解匿名查询日志、破坏关键词隐私、重构密文数据库\\[2mm]记录等实际攻击。
\end{card}   
\end{frame}

\begin{frame}{背景知识-MLE加密}
\begin{card}
消息锁定加密确立了加密重复数据删除的密码学基础:基\\[2mm]于数据内容产生密钥,从而将相同明文加密为相同密文。
\end{card}   
\end{frame}

\begin{frame}{两大核心问题}
\begin{card}
数据块频率泄漏问题:加密重复数据删除广泛应用MLE,\\[2mm]导致数据块的频率信息泄漏。
\end{card}
\begin{card}
针对数据块的频率分析攻击,与现有攻击目标(查询日志\\[2mm]条目、关键词、数据库记录等)相比,数据块数量极其庞\\[2mm]大(呈千万级),并且大量数据块具有相同频率,致使当\\[2mm]前的频率分析攻击算法难以适用。
\end{card}
\end{frame}

\begin{frame}{研究的意义}
\begin{card}
填补频率分析攻击研究空白。
\end{card}
\begin{card}
对理解加密重复数据删除的实际安全性,并降低其在非适\\[2mm]合场景下的误用风险具有重要作用。
\end{card}
\end{frame}


\begin{frame}{研究的目的}
\begin{card}
在理论上:构造针对加密重复数据删除的频率分析攻击,\\[2mm]揭示实践中的安全隐患。
\end{card}
\begin{card}
在技术上:以理论研究为支撑,设计并实现针对加密重复\\[2mm]数据删除的频率分析攻击工具,并在真实系统中进行理论\\[2mm]验证和攻击效果测试。 
\end{card}
\end{frame}

\section{国内外研究现状}

\begin{frame}{国内外研究现状}
\begin{card}
\begin{itemize}
    \item 加密重复数据删除
    \item 频率分析攻击
    \item 密文重复数据删除的其他攻击
\end{itemize}
\end{card}
\end{frame}

\begin{frame}{加密重复数据删除}
\begin{card}
密码学理论基础:\textbf{消息锁定加密(MLE)}
\end{card}

\begin{itemize}
    \item 收敛加密(CE)使用明文的哈希值作为MLE密钥,并基于\\[2mm]密文哈希值计算指纹,以识别重复数据。
    \item 哈希收敛加密(HCE)与CE具有相同的MLE密钥产生规则\\[2mm],但基于明文哈希值计算指纹。
    \item 随机收敛加密(RCE)使用随机密钥加密以产生非确定的密\\[2mm]文,同时基于明文哈希值来进行重复检查。
    \item 收敛扩散(CD)使用明文哈希值作为秘密共享的输入种子\\[2mm],提高了密文存储的可靠性。
\end{itemize}

\end{frame}

\begin{frame}{频率分析攻击}
\begin{card}
针对确定性加密的密码分析技术。
\end{card}
\begin{card}
面向加密重复数据删除,已有工作提出了基于数据块局部\\[2mm]性(chunk locality)的频率分析攻击。
\end{card}
\end{frame}

\begin{frame}{密文重复数据删除的其他攻击}
\begin{card}
加密重复数据删除可能遭受边信道攻击、副本伪造攻击、\\[2mm]
基于数据块长度的攻击等威胁,但这些攻击可通过所有权\\[2mm]
证明、守卫解密(guarded decryption)、固定长度分块等措\\[2mm]施进行防御。
\end{card}
\begin{card}
本研究的频率分析攻击超出了现有保护措施的防御范畴。
\end{card}
\end{frame}

\section{主要研究内容}

\begin{frame}{对选题的分析}
\begin{card}
\begin{itemize}
    \item 三个主要研究内容.
    \item 研究的技术路线.
\end{itemize}
\end{card}
\end{frame}

\begin{frame}{三个主要研究内容}
\begin{card}
研究基于数据特征的新型频率分析攻击技术,提高传统频\\[2mm]率分析的攻击效果。
\end{card}
\begin{card}
分别从抵抗频率排序干扰和降低攻击发生条件两方面改进\\[2mm]攻击技术。
\end{card}
\begin{card}
实现针对真实系统的频率分析攻击原型,并分析该攻击对\\[2mm]各类数据安全性的影响。
\end{card}
\end{frame}

\begin{frame}{研究的技术路线}
\centering
\cardImg{img/TechnicalRoute}{\textwidth}
\begin{card}
本研究的技术路线
\end{card}   
\end{frame}


\section{攻击方案的设计}

\begin{frame}{攻击方案的设计}
\begin{card}
\begin{itemize}
    \item 已有工作:基于数据块局部性的攻击
    \item 本课题研究内容:基于分布的攻击
    \item 本课题研究内容:基于聚类的攻击
\end{itemize}
\end{card}
\end{frame}

\begin{frame}{基于数据块局部性的攻击}
\centering
\cardImg{img/BaseFrequencyAttack}{\textwidth}
\begin{card}
已有工作提出的基于数据块局部性的攻击方案
\end{card}   
\end{frame}

\begin{frame}{基于分布的攻击}
\centering
\cardImg{img/DistributionAttack}{\textwidth}
\begin{card}
设计的基于分布的攻击的攻击方案
\end{card}   
\end{frame}

\begin{frame}{基于聚类的攻击}
\centering
\cardImg{img/ClusteringAttack}{\textwidth}
\begin{card}
设计的基于聚类的攻击方案
\end{card}   
\end{frame}

\section{研究结果与分析}

\begin{frame}{研究结果与分析}
\begin{card}
\begin{itemize}
    \item 基于分布的攻击-部分结果
    \item 基于聚类的攻击-部分结果
\end{itemize}
\end{card}
\end{frame}

\begin{frame}{基于分布的攻击结果}
\begin{figure}[!htbp]
    \centering
    \begin{tabular}{p{.48\linewidth}p{.48\linewidth}}
        \multicolumn{2}{c}{\includegraphics[width=.35\textwidth]{img/legend-effectiveness.pdf}}  \\
        \includegraphics[width=\linewidth]{img/distribution-effectiveness-wo-size.pdf} &
        \includegraphics[width=\linewidth]{img/distribution-effectiveness-w-size.pdf}\\
    \end{tabular}
\end{figure}
\begin{card}
设计的基于分布的攻击方案在FSL数据集中的攻击效果\\[2mm](左无数据块大小信息辅助,右有数据块大小信息辅助)
\end{card}  
\end{frame}

\begin{frame}{基于聚类的攻击结果}
\begin{figure}[!htbp]
    \centering
    \begin{tabular}{p{.48\linewidth}p{.48\linewidth}}
        \multicolumn{2}{c}{\includegraphics[width=.35\textwidth]{img/clu-effect-bar.pdf}}  \\
        \includegraphics[width=\linewidth]{img/clu-effect-rate.pdf} &
        \includegraphics[width=\linewidth]{img/clu-effect-pre.pdf}\\
    \end{tabular}
\end{figure}
\begin{card}
设计的基于聚类的攻击方案在VM数据集中的攻击效果。
\end{card}  
\end{frame}

\section{总结建议与参考文献}
\begin{frame}{总结建议与参考文献}
\begin{card}
\begin{itemize}
    \item 关于研究的总结
    \item 针对本文提出的攻击的建议
    \item 参考文献
\end{itemize}
\end{card}
\end{frame}

\begin{frame}{关于研究的总结}
\begin{card}
加密重复数据删除应用确定性加密,并由此泄漏了明文的\\[2mm]
频率。研究重新审视了频率分析引起的安全漏洞,并证明\\[2mm]
加密重复数据删除更容易受到推理攻击。
\end{card}

\begin{card}
研究提出了两种新的频率分析攻击方法,它们在攻击者所\\[2mm]
具有的条件的不同假设下都能实现高推理率和高推理精度。
\end{card}
\end{frame}

\begin{frame}{关于研究的总结}
\begin{card}
利用三个真实世界的数据集来验证评估这两种攻击方法,\\[2mm]
提出关于其性质的各种新观察,并进一步分析它们如何带
来实际性的损害。
\end{card}
\begin{card}
研究还讨论了加密重复数据删除应对频率分析攻击的可能\\[2mm]
的对策及其相应的优缺点,以建议从业者安全地实现和部\\[2mm]
署加密重复数据删除存储系统。
\end{card}
\end{frame}

\begin{frame}{针对本文提出的攻击的建议}
\begin{card}
\begin{itemize}
    \item \textbf{防止频率泄漏}:MinHash加密、加入冗余数据块。
    \item \textbf{防止顺序泄漏}:添加扰动。
    \item \textbf{防止大小泄漏}:数据填充、固定大小分块。
\end{itemize}
\end{card}
\end{frame}

\begin{frame}{参考文献}
\begin{itemize}
    \item M. Bellare, S. Keelveedhi, T. Ristenpart. Message-locked encryption and secure deduplication[C]. Annual International Conference on the Theory and Applications of Cryptographic Techniques, 2013, 296-312
    \item M. Naveed, S. Kamara, C. V. Wright. Inference attacks on property-preserving encrypted
databases[C]. Proceedings of the 22nd ACM SIGSAC Conference on Computer and Communications Security, 2015, 644-655
    \item J. Li, C. Qin, P. P. Lee, et al. Rekeying for encrypted deduplication storage[C]. Dependable
Systems and Networks (DSN), 2016 46th Annual IEEE/IFIP International Conference on, 2016, 618-629
    \item ...
\end{itemize}
\end{frame}
\end{document}
