%!TEX options = --shell-escape


\documentclass[master]{thesis-uestc}

\title{Accelerating Encrypted Deduplication via SGX}{}
\author{任彦璟}{}
\advisor{李经纬\chinesespace 副教授}{}
\school{计算机科学与工程学院(网络空间安全学院)}{}
\major{计算机科学与技术}{}
\studentnumber{201921080334}{}

\begin{document}
\makecover

\begin{chineseabstract}
为了适应日益增长的宽带信号和非线性系统的工程应用,用于分析瞬态电磁散射问题的时域积分方程方法研究日趋活跃。本文以时域积分方程时间步进算法及其快速算法为研究课题,重点研究了时间步进算法的数值实现技术、后时稳定性问题以及两层平面波算法加速计算等,主要研究内容分为四部分。

……

\chinesekeyword{时域电磁散射,时域积分方程,时间步进算法,后时不稳定性,时域平面波算法}
\end{chineseabstract}

\begin{englishabstract}
With the widespread engineering applications ranging from broadband signals and non-linear systems, time-domain integral equations (TDIE) methods for analyzing transient electromagnetic scattering problems are becoming widely used nowadays. TDIE-based marching-on-in-time (MOT) scheme and its fast algorithm are researched in this dissertation, including the numerical techniques of MOT scheme, late-time stability of MOT scheme, and two-level PWTD-enhanced MOT scheme. The contents are divided into four parts shown as follows.

\englishkeyword{time-domain electromagnetic scattering, time-domain integral equation (TDIE), marching-on in-time (MOT) scheme, late-time instability, plane wave time-domain (PWTD) algorithm}
\end{englishabstract}

\thesistableofcontents

\thesischapterexordium

% \chapter{绪\hspace{6pt}论}

\section{研究工作的背景与意义}

\cite{ren2021accelerating}
\section{国内外研究历史与现状}
\section{本文的主要贡献与创新}
\section{本论文的结构安排}

\nocite{*}
\bibliographystyle{thesis-uestc}
\bibliography{reference}

\end{document}
