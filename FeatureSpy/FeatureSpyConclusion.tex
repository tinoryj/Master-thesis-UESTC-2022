\section{本章小结}
\label{sec:featurespy-conclusion}
本章解决了加密后重复数据删除中的推测内容攻击,提出了\sysnameF,通过在客户端安全区中主动检测推测内容攻击来增强加密后重复数据删除的安全性。它建立在以下观察的基础之上:发起推测内容攻击的恶意户端通过枚举大量相似的伪造文件进行攻击。因此,可通过检测较小时间段内处理数据中相似数据块出现频率来检测推测内容攻击。此外,本文提出了特征保留加密技术(SPE),并实现了基于密文的数据块相似性检查。实验分析表明\sysnameF 不仅可以有效地检测推测内容攻击,且在\sysnameS 中部署时仅产生有限的额外性能开销。
