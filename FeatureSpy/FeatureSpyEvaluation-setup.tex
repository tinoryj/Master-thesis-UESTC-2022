\section{实验分析}
\label{sec:featurespy-evaluation}


本文进行了广泛的评估来研究 \sysnameF (\S\ref{subsec:featurespy-evaluation-detection}) 的攻击检测有效性,以及 \prototype 在合成 (\S\ref{subsec:featurespy-syn}) 和现实世界中的性能(\S\ref{subsec:featurespy-real}) 工作负载。本文将主要结果总结如下:

\begin{itemize}[leftmargin=*]
\item \sysnameF 根据相应的密文数据块有效地找到相似的明文数据块。例如,当它们彼此进行中等修改时,它会检测到多达 80.2\% 的相似块。
\item \sysnameF 有效地检测到学习内容攻击,即使对手在普通内容快照中提交枚举内容。同时,它为现实世界的数据引入了低误判(例如,默认配置中的零)。
\item \prototype 在处理大规模真实世界数据时,在 SGXDedup(即没有 \sysnameF)的情况下会产生有限的上传(例如,8.8\%)和下载(例如,0.8\%)的性能开销。
\item 与现有的 \cite{harnik10, li15} 方法相比,可以安全抵御学习内容攻击,\prototype 具有带宽效率(例如,将网络流量节省高达 98.9\%)。
\end{itemize}


\subsection{数据集}
\label{subsec:featurespy-datasets}

\paragraph*{合成数据集。}

本文考虑三种类型的合成数据集进行评估。通过以受控方式修改基本块来生成第一数据集。具体来说,本文首先创建一个 8\,KiB 基本块 $M$。每次,本文在 $M$ 中随机选择 $x$ 个位置,并在每个位置更改 $y$ 个字节,以创建一个修改过的块。经过 10\,K 次修改后,本文得到一个合成数据集 {\em SYNChunk}$(x, y)$,其中包括从同一个基本块 $M$ 修改的 10\,K 个块。本文使用 SYNChunk$(x, y)$ 作为(相似块的)基本事实来研究相似性检测的有效性(Exp\#1 和 Exp\#2)。

此外,本文将 {\em SYNFile}$(x, y)$ 视为大小为 4\,KiB 的文件。本文假设 SYNFile$(x, y)$ 在文件内容中包含 $x$ 未知变量(其值将被推断),并且每个变量的值是从大小为 $y$ 的空间中选择的。本文使用 SYNFile$(x, y)$ 作为目标文件来研究 \sysnameF 在不同情况下的检测有效性(Exp\#4)。本文不考虑更大的目标文件,因为对手需要枚举更多相似的块(用于攻击)并且更有可能被 \sysnameF 捕获。

此外,本文将 {\em SYNUnique} 生成为一组 2\,GiB 文件,每个文件都包含全局非重复块。本文使用 SYNUnique 对 \prototype (\S\ref{subsec:featurespy-syn}) 的性能进行压力测试。

\paragraph*{Real-world datasets.} 本文考虑四个真实世界数据集,其特征总结在 Table~\ref{tab:featurespy-datasets} 中: (i) {\em FSL} \cite{fsl},其中包括 795 个 home 2013 年 1 月 22 日至 6 月 17 日 9 名学生的目录快照; (ii) {\em MS} \cite{meyer11},其中包括 143 个 windows 文件系统快照,每个快照的逻辑大小约为 100\,GiB; (iii) {\em Linux} \cite{linux},其中包括来自 Linux 源代码稳定版本(v2.6.11 和 v5.13 之间)的 84 个快照; (iv) {\em CouchDB} \cite{couchdb},其中包括 83 个 CouchDB 的 docker 镜像,版本介于 v2.5.2 和 v6.6.2 之间,在一般、社区和企业发行版下。请注意,FSL 和 MS 仅包括块元数据(例如,指纹和大小)而不是实际数据;本文将重放两条轨迹以评估 \prototype 在处理大规模工作负载(Exp\#9)中的性能。

\begin{table}
  \centering
  \small
  \begin{tabular}{|l|l|l|l|}
    \hline
    {\bf Datasets} & {\bf Snapshots} & {\bf Raw Size} & {\bf Deduplication Ratio} \\
    \hline
    \hline
    FSL & 795 & 56.2\,TiB & 140.4 \\
    \hline
    MS & 143 & 14.4\,TiB & 6.0 \\
    \hline
    Linux & 84 & 44.9\,GiB & 1.3 \\
    \hline
    CouchDB & 83 & 22.9\,GiB & 1.5 \\
    \hline
  \end{tabular}
  \caption{真实世界数据集的特征。 重复数据删除率定义为重复数据删除前数据大小与重复数据删除后数据大小之比。 更高的重复数据删除率意味着相应的数据集包含更多的冗余。}
  \vspace{-6pt}
  \label{tab:featurespy-datasets}
\end{table}
