\section{推测内容攻击}
\label{sec:featurespy-attack}

除了伪造所有权攻击(参见\S\ref{subsec:background-encrypted-deduplication-pow}),源端加密后重复数据删除还面临另一种侧信道攻击(仍然由恶意客户端发起),称为推测内容攻击(\textit{Learning-content attack})\cite{harnik2010side, zuo2018mitigating},它利用了源端重复数据删除泄露的重复数据删除结果(即,数据块是否已由任何其他客户端上传)。推测内容攻击假设攻击者先验地知道某些受害客户端拥有某个文件,其内容遵循公开的内容模板(即文件大部分内容已知)。攻击者的目标是识别文件中的私有部分。具体来说,攻击者会枚举私有部分的所有可能值,并生成许多伪造的文件(每个文件包含一种可能的值)。攻击者上传每个伪造的文件,如果被告知不需要传输某个文件的任意数据块(即伪造的文件不包含任何未被上传的数据块),则推断目标文件与当前伪造的文件相同。

本文认为数据所有权证明\cite{halevi11}(参见\S\ref{subsec:background-encrypted-deduplication-pow})不足以防范推测内容攻击,因为攻击者枚举了数据块的全部内容并且能够说服云服务端其对这些数据块的所有权。换句话说,数据所有权证明无法检测到这些数据块是完全由客户真实拥有的还是刻意伪造的。

\paragraph*{案例研究。}
本文扩展现有工作\cite{harnik2010side,zuo2018mitigating}对推测内容攻击的场景分析,设计案例验证推测内容攻击的可行性。考虑Alice和Bob为应届毕业学生,同时收到某公司的雇佣合同。他们将各自合同备份至云服务端。假设Alice为攻击者,通过推测内容攻击推断Bob合同中的薪水和签字费。

\begin{table}[!htb]
    \small
    \centering
    \begin{tabular}{@{}cccc@{}}
        \toprule
        测试环境 & 上传次数                           & 上传流量(MiB)                  & 攻击时间(秒)      \\ \midrule
        局域网   & \multirow{2}{*}{841.0 $\pm$ 608.3} & \multirow{2}{*}{7.4 $\pm$ 5.4} & 105.0 $\pm$ 76.1  \\
        阿里云   &                                    &                                & 475.5 $\pm$ 339.8 \\
        \bottomrule
    \end{tabular}
    \caption{推测内容攻击开销:以上结果包含十次测试的平均值和基于T分布(Student's t-Distribution)的95\%置信区间}
    \label{tab:LRI-verify}
\end{table}

为了实现以上推测内容攻击案例,本文基于Google合同模版\cite{GoogleOffer},更改其中姓名、年薪(假设为6K的倍数\cite{harnik2010side},介于204K和804K之间)和签字费(假设为10K的倍数,介于300K和600K之间)以生成Alice和Bob的合同,每个合同约占18.5KiB。本文随机生成Bob的薪水和签字费,并通过基于图~\ref{fig:Cloud-based-encrypted-deduplication-storage-logic}设计实现的基础原型将Bob的合同存储至服务端。令Alice基于自身合同作为内容模版,枚举Bob所有可能的薪水和签字费,以实施推测内容攻击。本文分别在本地局域网(客户端、密钥服务器和服务端均部署在本地测试平台中,请参阅\S\ref{subsec:featurespy-evaluation-performance}以了解测试平台详细配置)和阿里云(将客户端和密钥服务器部署在本地测试平台并将云服务端部署在阿里云\cite{Alibaba},请参阅 \S\ref{subsec:featurespy-evaluation-performance}以了解测试环境详细配置)实现上述攻击。如表~\ref{tab:LRI-verify}所示,Alice需上传大约841份伪造合同,消耗7.4MiB网络流量(包括传输非重复密文数据块和元数据),在本地局域网和阿里云环境中推断Bob的隐私信息分别只需105.0秒和475.5秒。
