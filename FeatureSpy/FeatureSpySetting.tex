\section{场景及假设}
\label{sec:featurespy-setting}
本文提出 \sysnameF 通过在客户端 {\em 可信执行环境 (TEE)} 中通过 {\em 主动检测学习内容攻击} 来增强加密后重复数据删除,从而完全击败恶意客户端。

\paragraph*{Trusted execution.} 本文建立在 {\em Intel Software Guarded Extensions (SGX)} \cite{sgx} 上来实现 TEE,因为 SGX 得到了当今商品计算机的广泛支持。除了 SGX,本文的设计 (\S\ref{sec:featurespy-design}) 可以扩展到其他支持 TEE 的可信计算技术 \cite{AMDSEV, pinto19}。

SGX 使用一组与安全相关的指令扩展 Intel CPU 以实现 TEE。它在硬件保护的内存区域(称为 {\em 安全区页面缓存 (EPC)})中分配一个 {\em 安全区},以便在具有机密性和完整性保证的情况下托管(in-enclave)内容。创建安全区后,SGX 提供 {\em 远程认证} 以通过远程实体(例如,云)对安全区进行身份验证,以及 {\em 安全密封} 在(经过身份验证的)enclave 和云端。此外,SGX 为安全区和未受保护的内存之间的交互提供了两个接口。程序可以通过 {\em 安全区调用 (ECalls)} 进入安全区执行安全区内部函数。在 ECall 中,它可以暂时退出安全区并通过 {\em 安全区外部调用 (OCalls)} 在不受保护的内存中调用不受信任的函数。



\paragraph*{部署场景。}图~\ref{fig:featurespy-model}展示了加密后重复数据删除的场景(\S\ref{subsec:featurespy-basics})。为了部署 \sysnameF,本文首先将安全区代码编译成共享对象 \cite{sgx},并将共享对象连同用于完整性验证的签名分发给每个客户端。云托管共享对象以验证每个客户端的安全区。具体来说,客户端初始化\sysnameF,通过加载共享对象创建对应的enclave,云端通过远程证明\cite{sgx}对每个enclave进行认证,以确保将正确的代码加载到enclave中。

在上传过程中,\sysnameF 处理明文数据块(由客户端生成),并为源端重复数据删除计算相应的密文数据块和指纹。未受损的客户端仅将不重复的密文数据块传输到云,而 \sysnameF 如果客户端被捕获以发起学习内容攻击,则会报告恶意客户端。

\begin{figure}
    \centering
    \includegraphics[width=\textwidth]{pic/featurespy/deployment.pdf}
    \vspace{-6pt}
    \caption{部署 \sysnameF.}
    \label{fig:featurespy-model}
    \vspace{-6pt}
\end{figure}

\paragraph*{威胁模型。} 本文的主要安全目标是增强加密后重复数据删除 (\S\ref{subsec:featurespy-basics}) 以防止 {\em 恶意} 客户端的安全性。与加密后重复数据删除 \cite{bellare2013MLE} 一样,本文考虑了一个旨在从任何存储的密文数据块中窃听原始内容的受损云。此外,本文考虑一个恶意客户端,旨在学习其他未受损客户端的原始明文数据块。具体来说,恶意客户端可以访问其受损的明文数据块和密钥,并任意伪造新的明文数据块以发起学习内容攻击(\S\ref{subsec:featurespy-basics})。此外,它可以篡改未受保护的内存中的内容和操作,以逃避 \sysnameF 的捕获。

本文的威胁模型做出以下假设。
\begin{itemize}[leftmargin=*]
    \item
          每个客户端与云之间的通信通道受 SSL/TLS 保护以防窃听。
    \item
          SGX安全区是受信任和经过身份验证的(例如,在首次引导时通过远程证明),以便诚实地执行攻击检测以防止篡改。此外,与之前的作品 \cite{shinde20, ren21} 一样,SGX安全区只为加密密钥(即 \S\ref{sec:featurespy-implementation} 中的证明密钥)而不是所有安全区内的内容保留机密性;鉴于针对 SGX 的侧信道攻击,这一点很重要(例如,请参阅 \cite{fei21} 进行调查)。
    \item
          恶意云可能会破坏外包数据以损害完整性。 \sysnameF 没有解决威胁,但它与 \textit{  (proof data prosession,PDP)} \cite{ateniese2007provable} 和 \textit{  (proof of retrievability,PoR)} \cite{juels07} 兼容,以执行定期的完整性验证外包数据,以及容错存储,以从损坏 \cite{li15} 中恢复数据。
\end{itemize}