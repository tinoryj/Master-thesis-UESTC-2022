\section{Implementation}
\label{sec:implementation}
我们使用 OpenSSL 1.1.1l \cite{openssl}、SGX SDK 2.15 \cite{sgxsdk} 和 SGX SSL \cite{sgxssl} 实现 \sysnameF,并将其部署到现有的基于 SGX 的加密重复数据删除系统 {\em SGXDedup} \cite{ren21} 来提高其针对学习内容攻击的安全性。具体来说,SGXDedup除了每个客户端和云端之外,还维护一个{\em key server}来管理一个全局secret,并根据chunk指纹和全局secret生成每个明文chunk的MLE key,从而对 {\em 离线暴力攻击} 具有鲁棒性(即解决了不可预测的假设)\cite{bellare13b}。为了加速基于源的加密重复数据删除,它在密钥服务器中部署了一个 SGX 安全区,并执行 {\em 推测加密} \cite{eduardo19} 以减轻(服务器辅助)密钥生成的在线计算开销。此外,它还部署了一个客户端安全区来执行基于 SGX 的高效 PoW (\S\ref{sub:secure_design})。请注意,SGXDedup 无法抵御学习内容攻击(\S\ref{sub:attack}),我们的新原型(称为 \prototype)是为了增强 SGXDedup 以抵御学习内容攻击的安全性。目前,\prototype(包括底层的 SGXDedup)由 C++ 中的 16\,K LoC 组成。下面,我们重点介绍与 \prototype 相关的实现细节。


\paragraph{设置。}
初始化后,\prototype 遵循 SGXDedup,通过 NIST P-256 椭圆曲线中的 {\em Diffie-Hellman 密钥交换 (DHKE)} 在云和每个客户端 enclave 之间共享一个 {\em proof key}。证明密钥在基于 SGX 的 PoW 中用于生成和验证签名(见下文)。


\paragraph{密钥生成。}
\prototype 适用于通过 Rabin 指纹 \cite{rabin81} 生成的可变大小明文块,最小、平均和最大大小分别为 4\,KiB, 8\,KiB 和 16\,KiB。为了在特征键 (\S\ref{sub:spe}) 上提供足够的熵,它将滑动窗口大小和 N 变换中的模数分别配置为 64 字节和 $2^{64}$。它生成 12 个子特征(每个 8 个字节),并将四个子特征的串联散列以形成每个特征(32 个字节)。它根据来自密钥服务器(如 SGXDedup \cite{ren21})的每个明文块的指纹请求 MLE 密钥,以及基于采样特征的特征密钥(根据 {\tt firstFeature},{\ tt minFeature} 或 {\tt allFeature})。


\paragraph{安全区操作。}
加密后(通过 AES-CFB-256 实现),\prototype 将 4,096 个密文块分批到 enclave 中进行处理,以减轻 SGX 上下文切换开销 \cite{arnautov16}。它跟踪哈希表中相似性指标的出现,并将窗口大小 $W$ 和比率阈值 $T$ 的默认值分别配置为 5\,K 和 3\%。

\prototype 遵循 SGXDedup 将基于源的重复数据删除与基于 SGX 的 PoW 相结合。它通过证明密钥基于 4,096 个指纹(密文块)的串联生成签名(通过 AES-CMAC 实现),以便云使用相同的密钥验证这些指纹的真实性。云通过LevelDB\cite{leveldb}实现指纹索引(\S\ref{sub:basics}),并通知客户端只传输非重复密文块。

\paragraph{存储管理。}
云端以 8\,MiB {\em 容器} 为单位管理非重复密文块,以减轻磁盘 I/O 开销。对于每个文件,它管理一个 {\em file recipe},其中列出了密文块的指纹,以及相应的 MLE 密钥和特征密钥。为保护机密性,每个客户端在将其外包到云端之前,都会使用单独的主密钥对文件配方进行加密。

要下载文件,客户端首先从云端检索文件配方,并使用相应的主密钥对其进行解密。然后,客户端根据文件配方检索密文块,并根据 MLE 密钥和功能密钥对其进行解密。

\paragraph{优化。}
我们应用标准方法来提高 \prototype 的性能。每个客户端在不同线程中提取多个明文块的内容特征,并在管道中并行处理分块、密钥生成、加密、PoW和上传。此外,为了提高下载性能,云会在内存中维护一个最近最少使用的缓存 (1\,GiB) 来保存最近恢复的容器。对于每个下载请求,它首先在缓存中搜索容器,并且仅当容器不在缓存中时才从磁盘中检索容器。