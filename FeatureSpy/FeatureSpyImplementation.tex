\section{\prototype 实现}
\label{sec:featurespy-implementation}
本文基于OpenSSL 1.1.1l\cite{openssl}、SGX SDK 2.15\cite{sgxsdk}和Intel SGX SSL\cite{sgxssl}实现了\sysnameF,并将其部署到本文提出的基于TEE的加密后重复数据删除系统\sysnameS (参见\S\ref{sec:sgxdedup-implementation})中(新原型系统称为\prototype)以提高其针对推测内容攻击的安全性。\prototype(包括底层的\sysnameS )共计包含16,400行C++代码。

\paragraph*{原型系统启动。}
初始化后,\prototype 遵循\sysnameS,对每个客户端持有的所有全证明安全区基于NIST P-256椭圆曲线\cite{nist}实现DHKE,以与云服务端共享PoW密钥。

\paragraph*{密钥生成。}
\prototype 采用基于Rabin指纹\cite{rabin81}的内容定义的数据块分块方案生成可变大小的数据块,其中最小、平均和最大数据块大小分别固定为4\,KiB,8\,KiB和16\,KiB。为了给特征密钥(\S\ref{subsec:featurespy-spe})提供足够的信息熵,将Rabin指纹计算的滑动窗口大小和N-transform特征提取中的模数分别配置为64字节和$2^{64}$。在特征提取过程中,生成12个子特征(每个子特征大小为8个字节),并计算相临四个子特征的串联结果的SHA-256安全哈希作为特征(共计3个,每个32个字节)。除消息锁加密密钥之外,本文提出的特征密钥也采用服务器辅助密钥生成方案产生。服务器辅助特征密钥生成中,将数据块的{\tt firstFeature},{\tt minFeature}或{\tt allFeature}作为密钥生成的输入。

\paragraph*{安全区操作。}
在数据块加密(AES-CFB-256)后,\prototype 将整合至多4,096个密文数据块为一批送入到安全区中进行处理,以减轻安全区上下文切换开销\cite{arnautov2016SCONE}。在安全区中,\prototype 使用哈希表记录数据块相似性指标的频率,并将检查窗口大小$W$和推测内容攻击检测阈值$T$的默认值分别配置为5\,K和3\%。

此外,\prototype 遵循\sysnameS 将源端重复数据删除与基于TEE的数据所有权证明相结合。它基于PoW密钥和每批进入安全区的密文数据块指纹(至多4,096个)的串联结果生成签名(使用AES-CMAC签名算法),以便云服务端使用相同的PoW密钥验证这些指纹的真实性。云服务端基于LevelDB\cite{leveldb}实现数据块指纹索引(\S\ref{sec:background-enc-deduplication}),并通知客户端只传输非重复的密文数据块。

\paragraph*{存储管理。}
云服务端以8\,MiB大小的容器为基本单位管理非重复密文数据块,以减轻磁盘I/O开销。对于每个文件,它管理一个文件元数据,其中列出了密文数据块的指纹,以及相应的消息锁加密密钥和特征密钥。为保护文件机密性,每个客户端在将该文件元数据外包存储到云服务端之前,都会使用各自的主密钥对其进行加密。

若要下载文件,客户端首先从云服务端检索目标文件的文件元数据,并使用相应的主密钥对其进行解密。然后,客户端根据文件元数据在服务端检索密文数据块,并根据文件元数据中的消息锁加密密钥和特征密钥对其进行解密,最终整合为原始文件。

\paragraph*{系统优化。}
本文采用常规多线程与缓存方式优化\prototype 的性能。每个客户端并行处理分块、明文数据块内容特征提取(在不同线程中同时提取多个明文数据块的内容特征)、密钥生成、加密、所有权证明和非重复数据块及文件元数据的上传。此外,为了提高下载性能,云服务端在内存中维护一个最近最少使用(LRU)缓存,大小为1\,GiB,用来保存最近访问的容器。对于每个下载请求,云服务端首先在缓存中搜索目标容器,并且仅当目标容器不在缓存中时才从磁盘中检索容器。