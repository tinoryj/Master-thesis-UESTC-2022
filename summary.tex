\chapter{全文总结与展望}

\section{全文总结}

本文以可信执行环境在加密后重复数据删除中的应用为研究背景,主要对提升服务器辅助消息锁加密

本文基于可信执行环境(TEE)解决了加密后重复数据删除中的性能问题。本文提出了\sysnameS,它实现了一组安全区内部调用以在安全区内执行敏感操作,从而在保持安全性的同时加速加密重复数据删除。\sysnameS 包含三项关键设计:(1)安全高效的安全区管理;(2)自更新的会话密钥管理;以及(3)用于减轻在线加解密开销的基于TEE的推测性加密。此外,本文的实验分析表明\sysnameS 在合成数据集和真实世界数据集作为工作负载的情况下具有远超现有基于密码学机制的加密重复数据删除的性能表现。

本章解决了加密后重复数据删除中的推测内容攻击,提出了\sysnameF,它通过在客户端TEE中主动检测推测内容攻击来增强加密后重复数据删除。它建立在以下观察的基础之上:发起推测内容攻击的而已客户端往往会枚举大量相似的伪造文件进行攻击。因此,可通过检测较小时间段内处理数据中相似数据块出现频率来检测推测内容攻击。此外,本文提出了特征保留加密技术(SPE),并实现了基于密文的数据块相似性检查。实验分析表明\sysnameF 不仅可以有效的检测推测内容攻击,且在\sysnameS 中部署时仅产生有限的额外性能开销。


\section{后续工作展望}

高性能加密后重复数据删除及可信执行环境在存储系统中的应用等相关研究近几年发展迅速,在本文研究工作的基础上,仍有以下方向值得进一步研究:
