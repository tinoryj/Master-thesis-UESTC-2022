\chapter{威胁模型}
\label{sec:ThreatModel}
本章为针对加密重复数据删除的频率分析攻击制定了威胁模型。

\section{威胁模型定义}
\label{sec:ThreatModel-Definitions}

本文分别使用$M$和$C$来表示明文数据块(即加密前的逻辑数据块)及其对应的密文数据块(即加密后的逻辑数据块);$|M|$和$|C|$来表示明文数据块(即加密前的逻辑数据块)的大小及其对应的密文数据块(即加密后的逻辑数据块)的大小。

首先将普通文件建模为由$n$个逻辑明文数据块构成的有序列表(即重复数据删除之前的逻辑块),该列表记为$\mathbf{M} = \langle \hat{M}^{(1)}, \hat{M}^{(2)}, \ldots, \hat{M}^{(n)}\rangle$。 每个逻辑明文数据块$\hat{M}^{(i)}$(其中$1\le i\le n$)通过MLE加密得到相应的密文数据块$\hat{C}^{(i)}$。由$\mathbf{M}$的所有加密结果形成由$n$个逻辑密文数据块组成的有序列表记为$\mathbf{C} = \langle \hat{C}^{(1)}, \hat{C}^{(2)}, \ldots, \hat{C}^{(n)} \rangle$。在$\mathbf{C}$中的逻辑密文数据块也会被排序以反映加密重复数据删除存储系统中重复数据删除处理过程的顺序。

相同的逻辑明文和密文可能分别出现在$\mathbf{M}$和$\mathbf{C}$中的不同位置。 本文将一个唯一的明文数据块表示为$M$(通过其指纹的唯一性确定),通过MLE加密得到相应的唯一密文数据块$C$。每个$M$对应于$\hat{M}^{(i)}$的一个或多个相同副本,同理,每个$C$对应于$\hat{C}^{(i)}$的一个或多个相同副本。


\section{对手的目标和相关假设}
\label{sec:ThreatModel-Assumptions}

本文考虑存在一个对手,它准备以以下两个明确的目标为基准推理一组密文数据块-明文数据块对(用\{$(C, M)$\}表示)。


\begin{itemize}
    \item \textbf{推理率高:} 
    
    在所有的正确密文数据块-明文数据块对中,推理出大部分正确的密文数据块-明文数据块对(即统计学上的高召回率或低阴性率)。

    \item \textbf{推理精度高:} 
    
    在所有推理得到的密文数据块-明文数据块对中大部分的密文数据块-明文数据块对的配对是正确的(即统计术语中的高精度或低误报率)。
\end{itemize}
  
  
本文假设该对手是诚实但好奇的,它可以被动地监视密文数据块流$\mathbf{C}$被写入存储系统中的过程,并利用来自$\mathbf{C}$的不同类型的泄漏信息(参见章节:\ref{sec:ThreatModel-Leakage})。鉴于可用的泄漏信息种类,对手仅可通过唯密文攻击的方法推理$\mathbf{C}$中每个密文数据块对应的原始明文数据块。当然,对手也有可能知道有限的密文数据块-明文数据块对的集合来发起已知明文攻击,这种情况进一步加深了攻击的严重性\citing{li2017information}。但在本文中不会考虑前述的已知明文攻击方式。因此,本文将$\mathbf{C}$视为对手所能观察到的的视图,并为该视图的不同属性建模。

本文假设对手无法访问任何包含有关如何操作和存储数据块的信息的元数据(由于不对元数据应用重复数据删除,因此可以通过传统的对称加密来保护这些元数据,这种操作可使得对手无法获知该类型的信息)。此外,本文假设对手没有主动对重复数据删除系统进行攻击的能力,因为这种问题可以通过现有方法予以阻止。例如是恶意客户端可以在客户端重复数据删除中声明对未授权文件拥有所有权\citing{harnik2010side,halevi2011proofs,mulazzani2011dark}; 该问题可以通过所有权证明\citing{halevi2011proofs,xu2013weak,di2012boosting}或服务器端重复数据删除\citing{harnik2010side,li2015cdstore}予以解决。另一个例子是恶意存储系统可以修其改存储的数据;该问题可以通过远程完整性检查来进行检测和解决\citing{juels2007pors,ateniese2007provable}。

\section{三种信息的泄漏}
\label{sec:ThreatModel-Leakage}

本文在加密重复数据删除存储中考虑三种类型的泄漏方式,这些泄漏使得对手能够基于它们推理信息:

\begin{itemize}
    \item \textbf{频率:}  

    由于加密重复数据删除的加密过程的确定性,重复数据删除前$\mathbf{C}$中每个密文数据块的频率(即重复副本数量)可以映射得到$\mathbf{M}$中相应明文数据块的频率。

    \item \textbf{顺序:} 
 
    某些加密重复数据删除存储系统\citing{xia2011silo,lillibridge2009sparse,zhu2008avoiding}为了更高的运行性能,在存储密文数据块是保持了其对应的明文数据块的原始顺序。因此,$\mathbf{C}$中的密文的顺序可以映射到$\mathbf{M}$中的明文的顺序。

    \item \textbf{大小:} 

    可变大小的数据块分块方法(参见章节:\ref{sec:background})产生会产生各种不同大小的明文数据块,如果在对明文数据块进行加密前没有进行数据填充(为了避免增大存储开销\citing{ritzdorf2016information,douceur2002reclaiming,wilcox2008tahoe,keelveedhi2013dupless}),则产生的密文数据块大小与其对应的原始明文数据块大小一致($\mathbf{C}$中的密文大小可以映射到$\mathbf{M}$中相应明文的大小)且密文数据块集合中的密文数据块也具有众多不同的大小。当然,该泄漏可以通过使用块密码算法使得明密文数据块大小不一致来避免。
\end{itemize}

除上述三种泄漏以外,对手还可以获得一些辅助信息,这些信息提供了与$\mathbf{M}$相关的数据特征的基本情况(任何推理攻击都必须提供辅助信息\citing{kumar2007anonymizing,li2017information,grubbs2016breaking,zhang2016all,kellaris2016generic,ritzdorf2016information,naveed2015inference,cash2015leakage,islam2012access})。在这项工作中,将辅助信息视为先前已知的明文数据块(例如,通过旧用户备份或VM磁盘映像得到)构成的有序列表,由$\mathbf{A}$表示。显然,攻击有效性与严重性取决于$\mathbf{A}$(即先前已知的明文数据块)和$\mathbf{M}$(即要推理的明文数据块)之间的相关性。本文的重点不是解决对手如何获取辅助信息的问题。例如,可能是由于粗心的数据发布\citing{careless-release}、被的盗存储设备\citing{stolen-device}和云存储泄漏\citing{cloud-leakage}。相反,根据这些信息,本文研究了可以获得的的辅助信息如何与各种泄漏渠道相结合,给加密重复数据删除带来信息泄漏。

本文将频率分析\citing{al1992origins}作为攻击方法。经典频率分析对$\mathbf{C}$中的唯一密文数据块和$\mathbf{A}$中的唯一明文数据块按频率(即对应于每个唯一密文数据块或唯一明文数据块的相同副本的数量)进行排序。然后,简单的将$\mathbf{C}$中的每个唯一密文数据块与$\mathbf{A}$中具有相同频率等级的唯一明文数据块相关联。基本的频率分析攻击实际效果非常糟糕,在接下来的章节中,本文针对加密重复数据删除设计了较为复杂同时效性极高的频率分析攻击。

\section{本章小结}

本章介绍了针对重复数据删除的威胁模型的定义、本课题研究中将要用到的各种信息来源,以及本文攻击方案的假设的相关说明。





