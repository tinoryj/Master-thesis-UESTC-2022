\chapter{结束语}
\label{sec:Conclusion}

\section{本文总结}

加密重复数据删除应用确定性加密,并由此泄漏了明文的频率。本文重新审视了频率分析引起的安全漏洞,并证明加密重复数据删除更容易受到推理攻击。本文提出了两种新的频率分析攻击方法,它们在攻击者所具有的条件的不同假设下都能实现高推理率和高推理精度。本文用三个真实世界的数据集来验证评估这两种攻击方法,提出关于其性质的各种新观察,并进一步分析它们如何带来实际性的损害。本文还讨论了加密重复数据删除应对频率分析攻击的可能的对策及其相应的优缺点,以建议从业者安全地实现和部署加密重复数据删除存储系统。
     
\section{下一步学习工作方向}

本文为未来的工作提出三个方向。


首先,本文将完整的旧备份视为辅助信息,如果攻击者仅对旧备份的部分有了解(没有完整的旧备份用作辅助信息),则不研究攻击者如何发起攻击。可能的是,攻击者仍然可以应用频率分析攻击从可用的部分备份中提取特征,并通过比较特征与目标备份中的特征来推理密文数据块-明文数据块对。未来研究的第一个方向是设计更高级的推理攻击方法,并比较在前述的部分知识案例中新方法是否比直接应用本文提出的两种攻击更好。
      
      
\par 其次,本文根据频率分析攻击的有效性调整频率分析攻击的参数,这种调整只有在攻击发生后才能学习。本文没有研究如何预先从辅助信息中推导出最佳参数。未来研究的第二个方向是在预先通过辅助信息推导最佳攻击参数方面尝试作出改进。
 


第三,本文不对实际的加密重复数据删除存储系统实现攻击原型。未来研究的第三个方向是在实际系统中部署本文提出的攻击设计以及报告加密重复数据删除在实践中存在的漏洞。

