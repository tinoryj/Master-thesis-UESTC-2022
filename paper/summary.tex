\chapter{全文总结与展望}

\section{全文总结}

本文以可信执行环境在加密重复数据删除中的应用为研究背景,主要研究提升服务器辅助消息锁加密和数据所有权证明性能,并进一步解决数据所有权证明技术难以应对的推测内容攻击的检测问题。

本文基于可信执行环境解决了加密重复数据删除中的性能及安全性问题。本文提出了\sysnameS 和\sysnameF 方案,通过一组安全区内部调用在安全区内执行敏感操作,从而在保持安全性的前提下加速源端加密重复数据删除。其中,\sysnameS 包含三项关键设计,即安全高效的安全区管理、自更新的会话密钥管理、用于减轻在线加解密开销的基于TEE的推测性加密。而\sysnameF 则根据发起推测内容攻击的恶意户端往往会枚举大量相似的伪造文件进行攻击的重要观察提出了相似性保留加密(SPE),进而实现了基于密文数据块的推测内容攻击检测。

本文最终将\sysnameS 和\sysnameF 整合形成\prototype 原型系统。实现了完整对抗加密重复数据删除中的侧信道攻击,同时保留源端重复数据删除的低网络资源开销,且实际性能远超现有基于软件(密码学机制)的加密重复数据删除原型系统。

\section{后续工作展望}

高性能加密重复数据删除及可信执行环境在存储系统中的应用等相关研究近几年发展迅速,在本文研究工作的基础上,仍有以下方向值得进一步研究:

\begin{itemize}[leftmargin=0em]
    \item 本文提出的\sysnameF 尚不足以应对大量客户端联合发起的合作式推测内容攻击(\S\ref{subsec:featurespy-security}),防御合作式的推测内容攻击是未来亟需解决的难点。
    \item 本文提出的\sysnameF 方案目前采用固定的推测内容攻击检测阈值及检测窗口大小参数,而这些参数将直接影响对推测内容攻击的检测率和误判率。如何自动平衡检测对低信息熵文件的攻击和在处理不同工作负载时最大限度地减少误判之间的权衡称为亟需解决的关键问题。
    \item 现有研究集中于加密重复数据删除(\S\ref{sec:background-enc-deduplication}),尚未探索加密前重复数据删除。理论上,在加密前执行重复数据删除将具有密钥管理、存储效率和安全性优势,TEE的广泛应用为加密前重复数据删除提供了可能。
\end{itemize}
