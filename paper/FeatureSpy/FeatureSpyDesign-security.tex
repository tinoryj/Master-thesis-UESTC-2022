
\subsection{安全性分析}
\label{subsec:featurespy-security}
本节讨论了\sysnameF 的机密性保障,特别是作为底层加密原语的相似性保留加密(SPE)的安全性,并讨论了\sysnameF 应对恶意客户端的鲁棒性。

\paragraph*{对不可信云服务端的保密性。}

本文已经讨论了当密钥被泄露时相似性保留加密(SPE)相对于特征加密(FBE)的安全性提升(\S\ref{subsec:featurespy-spe}),这里关注密钥未被泄露(保持安全)时的情况。本文的目标是证明相似性保留加密(SPE)可以保护加密重复数据删除面临云服务端不可信的安全问题。具体来说,由于相似性保留加密(SPE)通过消息锁加密和特征加密(FBE) 共同执行数据块加密,因此其安全性可规约为消息锁加密和特征加密的安全性。现有工作\citing{bellare2013MLE}表明,当明文数据块取自一个较大的信息空间,并且任意数据块的选取难以从信息空间中进行预测(即,明文数据块不可预测)。在下文中,本文(非正式地)说明,如果数据块特征同样不可预测,则特征加密可达到与消息锁加密同等地安全性。

本文将特征加密视为消息锁加密的一种扩展形式。如果本文将采样的特征(例如,\textit{firstFeature}和\textit{minFeature}等代指的代表特征,以及\textit{allFeature}代指的全部特征)视为消息$M$,那么消息锁加密实际上使用$M$的安全哈希来加密对应的扩展$f(M)$(即明文数据块),其中$f(\cdot)$是一个扩展函数。由于每个$M$都是不可预测的,因此加密密钥对多项式时间的攻击者保持机密性。因此,如果底层对称加密是安全的,则攻击者无法区分产生的密文和随机值。本文的非正式分析依赖于特征不可预测的安全假设,且本文可以通过服务器辅助密钥生成\citing{bellare2013DupLESS}来增强特征密钥不可预测的假设。

\paragraph*{对恶意客户端的鲁棒性。}
本文已经讨论过攻击者无法绕过检测程序(\S\ref{subsec:featurespy-secure_design}),这里关注\sysnameF 对其他恶意行为的鲁棒性(\S\ref{sec:featurespy-setting})。

\begin{itemize}[leftmargin=0em]
    \item \textbf{案例1:篡改不受保护的程序。}
          \sysnameF 通过TEE保护检测程序,但恶意客户端可以篡改未受保护的程序。它可能会操纵数据块特征、相似性指标和数据块密钥,以欺骗\sysnameF。本文认为,这些操作无助于从遵循本文设计的其他诚实客户端处获得目标内容,因为这些操作导致即使原始明文数据块是相同的,恶意客户端与相似性保留加密(由正常客户应用)产生的密文数据块不同。进而导致该密文数据块无法执行重复数据删除(被认为是非重复数据块)。最终使得依赖重复数据删除结果泄漏的推测内容攻击不可行。
    \item \textbf{案例2:篡改数据处理流程。}
          由于\sysnameF 以处理窗口为粒度检查数据块相似性,恶意客户端可能会篡改正在处理的数据块流,在正常数据块流中小心的插入相似数据块(用于攻击),使得每个窗口内只包含极少数相似数据块。尽管\sysnameF 在这种情况下无法检测到攻击,但本文认为它显著缓解了推测内容攻击。例如,在没有\sysnameF 的情况下,恶意客户端可以全部使用相似的数据块(即,数据量为$W$)填充每个窗口以发动攻击,但\sysnameF 确保攻击者在每个窗口内至多可以填充$W\times T$ 的相似块(否则会被检测到),使得攻击成本提升$1/T$倍。通过配置一个较小的阈值$T$,本文可以使推测内容攻击成本显著增高而变得不切实际。使用较小阈值的代价使增加了误判的可能性(即,将正常情况误报为检测到攻击)。
\end{itemize}

\paragraph*{安全性限制。}\sysnameF 基于每个独立的客户端检测推测内容攻击。但是,攻击者可能会控制多个客户端以合作发起推测内容攻击,使得每个客户端上传的相似数据块的数量大大减少,\sysnameF 可能无法有效检测到此类推测内容攻击的实施。本文将防御合作的推测内容攻击作为未来亟需解决的工作。