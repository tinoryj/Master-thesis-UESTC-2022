\chapter{应对频率分析攻击的对策讨论}
\label{sec:Countermeasure}

在本章中,将讨论应对本文攻击方案使用的三种泄漏的对策及其优缺点。但在三中泄漏中,单纯的防范数据块大小信息泄漏并不足以使本文提出的攻击手段失效(大小信息的利用只是攻击用于进一步提高推理攻击准确性的可选条件)。

\section{防止频率泄漏}

MinHash加密\citing{qin2017design,li2017information}使用从一组相邻数据块上的最小块哈希导出的密钥加密该组内每个明文数据块,因此可能将相同的明文数据块加密映射到不同的密文数据块。防御频率信息泄漏的基本原理是通过MinHash加密改变各个密文数据块的频率,并由此扰乱不同数据块的频率排名。

与此同时,MinHash加密还可以用于防止本文中的频率分析攻击。MinHash加密是非确定性的加密方法,使用该方法会改变数据块的频率分布,因此本文的的攻击目标是确定性加密(例如,MLE\citing{bellare2013message})。但MinHash加密因为打破了明文数据块-密文数据块之间的一一映射关系,所以在加密重复数据删除中导致了存储效率的下降(重复数据删除所能节省的存储空间变小)。此外,因为MinHash加密的随机性主要取决于目标工作负载中的最小块哈希,所以其不是一种扰乱数据块频率的主动对策,在不同的工作负载中能实现的扰乱效果具有较大的差异。         

最近的一项工作\citing{zuo2018mitigating}提出通过有计划的添加冗余(重复)的数据块来防止针对客户端加密重复数据删除的流量分析攻击。该方法\citing{zuo2018mitigating}在运作中会改变数据块的频率,因此可以用于防止频率分析攻击。与MinHash加密相比,该方法\citing{zuo2018mitigating}只是添加了重复的数据块,因此不会降低存储效率。但另一方面,它需要使用在特定的假设之下。首先它要求特定的数据块是重复的,其次它只适用于客户端重复数据删除。因此使用这种方法可能会引入额外的泄漏通道(参见\ref{sec:RelatedWork})。

现有工作提出了几个扩展的MLE实例基于强加密原语构建,以防止加密重复数据删除的频率泄漏,例如支持等式测试的随机加密\citing{abadi2013message},混合加密\citing{stanek2014secure},以及与完全同态加密的交互\citing{bellare2015interactive}。它们都提供了可证明的安全性,但它们如何在实践中实施和部署仍未有进一步的研究。

\section{防止顺序泄漏} 

一个简单的对策是干扰数据块在重复数据删除处理中顺序。例如,加密重复数据删除可以在加密之前对明文数据块流进行额外的顺序扰动添加处理,以隐藏每个明文数据块的真实逻辑顺序。这种方法可以有效的防止基于分布的频率分析攻击(参见\ref{sec:DistributionAttack}),因为攻击者无法正确识别到数据块邻居信息。但它对于基于聚类的频率分析攻击(参见\ref{sec:ClusteringAttack})并不是完全有效的。如果添加扰动仅在很小的范围内进行(例如,每个数据段中的明文数据块的顺序被添加扰动),则基于聚类的频率分析攻击方法仍然有效。

顺序扰乱策略的最大缺陷在于它破坏了数据块的局部性并且在大规模的重复数据删除中会导致性能的严重下降\citing{xia2011silo,zhu2008avoiding,lillibridge2009sparse}。

\section{防止大小泄漏}

正如已有的工作\citing{ritzdorf2016information}所建议的那样,加密重复数据删除可以在每个明文数据块种填充额外的数据来混淆这个明文数据块的实际大小。但是,这种填充方案的实现存在很大的难度,因为它需要在相同的明文数据块中添加相同的冗余数据;否则加密重复数据删除存储系统无法正常检测数据块是否重复。一种可能的解决方案建立在MLE\citing{keelveedhi2013dupless,bellare2013message}的范例之上,考虑将每个明文数据块的加密哈希作为计算种子,并使用它来生成可变大小的伪随机数据用于明文数据块的填充。与MLE一样,这种解决方案的代价是需要使用服务器辅助方法\citing{keelveedhi2013dupless}来抵御暴力攻击(参见\ref{sec:RelatedWork})。   


另一种思路是使用固定大小的数据块分块方法来取代可变大小的数据块分块方法。在这种情况下,由于所有的数据块都具有相同的大小,因此攻击者无法利用大小信息来区分它们。虽然固定大小的分块会受到边界偏移的影响,但它在VM磁盘映像这一特定数据类型中实现了与可变大小数据块分块方案几乎相同的重复数据删除存储空间节省效果\citing{jin2009effectiveness}。 因此,本文建议在某些特定的加密重复数据删除工作负载(例如,VM磁盘映像)中应用固定大小的数据块分块方法,以防止数据块大小信息泄漏。

\section{本章小结}

本章给出了应对三种类型泄漏带来的风险的方案,并比较了现有方案的优缺点。
